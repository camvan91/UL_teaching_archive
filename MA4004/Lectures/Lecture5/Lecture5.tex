\documentclass[compress]{beamer}        % [compress] (written before {beamer} <=> navigation bar one line, all subsections in 1 line instead of 2

% Setup appearance:
\usetheme{CambridgeUS}
%	AnnArbor | Antibes | Bergen |
%	Berkeley | Berlin | Boadilla |
%	boxes | CambridgeUS | Copenhagen |
%	Darmstadt | default | Dresden |
%	Frankfurt | Goettingen |Hannover |
%	Ilmenau | JuanLesPins | Luebeck |
%	Madrid | Malmoe | Marburg |
%	Montpellier | PaloAlto | Pittsburgh |
%	Rochester | Singapore | Szeged |
%	Warsaw
%

\useoutertheme[footline=authorinstitute,subsection=false]{miniframes}
\usecolortheme{whale}

%	albatross | beaver | beetle |
%	crane | default | dolphin |
%	dove | fly | lily | orchid |
%	rose |seagull | seahorse |
%	sidebartab | structure |
%	whale | wolverine


\setbeamertemplate{footline}
{
  \hbox{%
  \begin{beamercolorbox}[wd=.25\paperwidth,ht=2.25ex,dp=1ex,center]{title in head/foot}%
    \usebeamerfont{date in head/foot}\insertshortauthor
  \end{beamercolorbox}%
  \begin{beamercolorbox}[wd=.5\paperwidth,ht=2.25ex,dp=1ex,center]{date in head/foot}%
    \usebeamerfont{title in head/foot}\insertshortinstitute
  \end{beamercolorbox}%
  \begin{beamercolorbox}[wd=.25\paperwidth,ht=2.25ex,dp=1ex,center]{title in head/foot}%
    \usebeamerfont{date in head/foot}
    \insertframenumber{} / \inserttotalframenumber
    %\insertframenumber{} / \insertpresentationendpage
  \end{beamercolorbox}}%
  \vskip0pt%
}

%\setbeamercolor{titlelike}{parent=structure}
%\setbeamercolor{structure}{fg=beamer@blendedblue}
%% \useinnertheme{rounded}
%\setbeamerfont{block title}{size={}}
%\usefonttheme[onlylarge]{structurebold}   % title and words in the table of contents bold
%\setbeamerfont*{frametitle}{size=\normalsize,series=\bfseries}
\setbeamertemplate{navigation symbols}{}
\setbeamercolor{frametitle}{parent=boxes, bg=white}
{ % only on titlepage


\usepackage{times}
\usepackage{amsmath,amssymb,amsthm}
\usepackage{color}
\usepackage{changepage}
\usepackage{multirow}
\usepackage[absolute,overlay]{textpos}
\usepackage{enumerate}
%\usepackage{pgfpages}
\usepackage[all]{xy}
\usepackage{textcomp}
\usepackage{etex}
\usepackage{tikz}
\usetikzlibrary{shapes}
%\usepackage{handoutWithNotes}
%\pgfpagesuselayout{4 on 1}[border shrink=1mm]




\definecolor{camblue}{RGB}{26,26,89}
\definecolor{Rblue}{RGB}{0,255,255}
\definecolor{Rdarkblue}{RGB}{0,0,255}
\definecolor{Rgreen}{RGB}{0,205,0}
\definecolor{green2}{RGB}{51,204,51}
\newcommand{\tcb}{\textcolor{beamer@blendedblue}}
\newcommand{\tcbb}{\textcolor{camblue}}
\newcommand{\tcr}{\textcolor{red}}
\newcommand{\tcg}{\textcolor{gray}}
\newcommand{\tcgr}{\textcolor{green2}}
\newcommand{\tcblk}{\textcolor{black}}
\newcommand{\tcRg}{\textcolor{Rgreen}}
\newcommand{\tcRdb}{\textcolor{Rdarkblue}}
\newcommand{\tcRb}{\textcolor{Rblue}}
\newcommand{\tcw}{\textcolor{white}}
\newcommand{\m}{\phantom{-}}
\newcommand{\bp}{\tcbb{$\bullet$}\:}


\title{{\huge Statistics for Computing\\[0.1cm]MA4413}}
\author[Kevin Burke]{{\bf\\[0.5cm]{\huge Lecture 5}\\[0.2cm]\emph{Counting Techniques:\\Basics, Permutations and Combinations}\\[1.4cm]Kevin Burke}\\[0.3cm]\tcb{kevin.burke@ul.ie}}

\institute[University of Limerick, Maths \& Stats Dept]{}
\date{}

%\TPGrid[5mm,5mm]{1}{1}

\begin{document}


\begin{frame}[t]
\titlepage
\end{frame}

\section{Introduction}
\subsection{Introduction}
\begin{frame}{\bf \tcb{Introduction}}

In this lecture, we are concerned with counting the number of ways of selecting objects.\\[0.4cm]

\begin{itemize}\itemsep0.4cm
\item Basics of counting.
\item Factorial operator.
\item Permutations.
\item Combinations.\\[0.6cm]
\end{itemize}

This content is most easily understood by considering various examples.

\end{frame}

\subsection{Example: Fast Food Stall}
\begin{frame}{\bf \tcb{Example: Fast Food Stall}}

A fast food stall serves two types of drink and three types of food.\\[0.2cm]
How many meal options are there in total?\\[0.2cm]
\begin{itemize}\itemsep0.2cm
\item Let $D_1$ and $D_2$ represent the two drinks.
\item Let $F_1$, $F_2$ and $F_3$ represent the three meals.\\[0.5cm]
\end{itemize}
Thus, the total \emph{set} of options is
\begin{align*}
\{(D_1, F_1), (D_1, F_2), (D_1, F_3), (D_2, F_1), (D_2, F_2), (D_2, F_3)\}
\end{align*}
$\Rightarrow$ 6 options altogether. {\bf Note that {\boldmath$6 = 2\text{ drinks} \times 3 \text{ foods}$}}.\\[0.5cm]

The purpose of this section is to learn how to count \emph{without} listing the whole set of choices.

\end{frame}


\subsection{Example: Fast Food Stall}
\begin{frame}{\bf \tcb{Example: Fast Food Stall}}

What if we plan to get \emph{two} of the food options?\\[0.3cm]

\begin{tabular}{l|c|c|c|c}
\cline{2-4}
&&&&\\[-0.4cm]
&Drink & 1st Food & 2nd Food &\\
\cline{2-4}
&&&&\\[-0.4cm]
Choices: & 2 & 3 & 3 & $= 2(3)(3) = 18$. \\
\cline{2-4}
\end{tabular}\\[1cm]


What if our second food choice \emph{must be different} to the first?\\[0.3cm]

In this case we have three choices for the first food, but for the second food we only have two choices since we have already chosen one food type.\\[0.3cm]

\begin{tabular}{l|c|c|c|c}
\cline{2-4}
&&&&\\[-0.4cm]
&Drink & 1st Food & 2nd Food &\\
\cline{2-4}
&&&&\\[-0.4cm]
Choices: & 2 & 3 & 2 & $= 2(3)(2) = 12$. \\
\cline{2-4}
\end{tabular}

\end{frame}

\subsection{Example: Fast Food Stall}
\begin{frame}{\bf \tcb{Example: Fast Food Stall}}

Let's assume there are also five types of dessert and we plan to get a drink, food and dessert. How many choices have we?\\[0.3cm]

\begin{tabular}{l|c|c|c|c}
\cline{2-4}
&&&&\\[-0.4cm]
&Drink & Food & Dessert & \\
\cline{2-4}
&&&&\\[-0.4cm]
Choices: & 2 & 3 & 5 & $= 2(3)(5) = 30$. \\
\cline{2-4}
\end{tabular}\\[1cm]


What if there is a possibility that we don't get dessert?\\[0.3cm]

In this case there are \emph{six} dessert options - the five options on the menu \emph{plus} the option of \emph{not getting} dessert.\\[0.3cm]

$\Rightarrow$ $2(3)(6) = 36$.

\end{frame}


\subsection{Example: Fast Food Stall}
\begin{frame}{\bf \tcb{Example: Fast Food Stall}}

What if we will \emph{either} get a drink \& food \emph{or} a drink \& dessert?\\[0.3cm]

Drink \& food: $2(3) = 6$.\\
Drink \& dessert: $2(5) = 10$.\\[0.2cm]
In total there are $6 + 10 = 16$ possible options.\\[1cm]

Let's assume we will get a drink \& food. However, there is only \emph{one} drink that we like. If the stall has it, we will get it. If the stall doesn't have it, we won't get a drink.\\[0.3cm]

Drink \& food: $1(3) = 3$.\\
No drink \& food: $1(3) = 3$.\\[0.2cm]
In total there are $3 + 3 = 6$ possible options.\\[0.3cm]

%Note that in either case we only have one option for the drink. If the stall has our drink we get it - one choice - whereas, if the stall doesn't have it, we get no drink - also one choice.

\end{frame}


\subsection{Example: Outcomes in a Sample Space}
\begin{frame}{\bf \tcb{Example: Outcomes in a Sample Space}}
Previously we enumerated all possible outcomes in a \emph{sample space}, $S$, using a manual approach.\\[0.4cm]

Now, consider the following experiments:\\[0.2cm]
\begin{enumerate}[i)]\itemsep0.3cm
\item Flipping three coins $\Rightarrow$ $\#(S) = 2 (2)(2) = 8$.
\item Flipping four coins $\Rightarrow$ $\#(S) = 2(2)(2)(2) = 16$.
\item Flipping a coin and a rolling a die $\Rightarrow$ $\#(S) = 2 (6) = 12$.
\item Rolling two dice $\Rightarrow$ $\#(S) = 6(6) = 36$.
\item Flipping a coin and rolling two dice $\Rightarrow$ $\#(S) = 2(6)(6) = 72$.
\item Rolling three dice $\Rightarrow$ $\#(S) = 6(6)(6) = 216$.
\end{enumerate}

\end{frame}


\subsection{Example: Probabilities}
\begin{frame}{\bf \tcb{Example: Probabilities}}\label{probslide}

Experiment (i): Flipping three coins.\\[0.1cm]
Let $A =$ ``one tail'' $= \{HHT, HTH, THH\}$\\[0.1cm]
$\Rightarrow$ $\Pr(A) = \frac{\#(A)}{\#(S)} = \frac{3}{8} = 0.375$.\\[0.4cm]

Experiment (ii): Flipping four coins.\\[0.1cm]
Let $A =$ ``two heads'' $= \{HHTT, HTHT, HTTH, THHT, THTH, TTHH\}$\\[0.1cm]
$\Rightarrow$ $\Pr(A) = \frac{\#(A)}{\#(S)} = \frac{6}{16} = 0.375$.\\[0.4cm]


Experiment (vi): Rolling three dice.\\[0.1cm]
Let $A =$ ``the sum of the no.s is five'' $= \{221, 212, 122,311,131,113\}$\\[0.1cm]
$\Rightarrow$ $\Pr(A) = \frac{\#(A)}{\#(S)} = \frac{6}{216} \approx 0.0278$.\\[0.4cm]

{\footnotesize Note: The only difficulty is calculating $\#(A)$. Sometimes this can be simplified using \emph{permutation / combination} \,theory - more on this later.}
\end{frame}


\subsection{Question 1}
\begin{frame}{\bf \tcb{Question 1}}
Let's assume you have four shirts (green, red, brown, black), two jackets (blue, black) and two pairs of trousers (brown, black).\\[0.3cm]
\begin{enumerate}[a)]\itemsep0.3cm
\item How many outfits have you got altogether?
\item What if the shirt must be red?
\item What if the shirt must be green or black?
\item What if the shirt must be green and the jacket must be blue?
\item What if no item is to be black?
\item What if at least one item must be black?
\item What if the shirt and jacket can be the same colour but the trousers must be a different colour?
\end{enumerate}

\end{frame}



\section{Factorial Operator}
\subsection{Factorial Operator}
\begin{frame}{\bf \tcb{Factorial Operator}}
The {\bf factorial} operation is given by:\\
\begin{align*}
\boxed{n\,! = n \times (n - 1) \times (n - 2) \cdots \times 3 \times 2 \times 1}.\\[-0.3cm]
\end{align*}

The above is pronounced ``$n$-factorial''.\\[0.5cm]

\begin{minipage}[t]{0.45\textwidth}
Examples
\begin{itemize}
\item $1! = 1$.
\item $2! = 2(1) = 2$.
\item $3! = 3(2)(1) = 6$.
\item $4! = 4(3)(2)(1) = 24$.
\item $5! = 5(4)(3)(2)(1) = 120$.
\end{itemize}
\end{minipage}\hspace{0.08\textwidth}
\begin{minipage}[t]{0.45\textwidth}
We can see that
\begin{align*}
5! &= 5 \times 4!\\[0.3cm]
&= 5 \times 4 \times 3!\\[0.3cm]
&\quad\ldots \text{ etc}.
\end{align*}
\end{minipage}
\vspace{0.2cm}
\begin{center}
{\bf Note:} \emph{by definition} we have that $\boxed{0\,! = 1}.$
\end{center}

\end{frame}


\subsection{Example}
\begin{frame}{\bf \tcb{Example}}

Sometimes we get expressions such as\\[-0.1cm]
\begin{align*}
\frac{10!}{6!} = \frac{10\times9\times8\times7\times6!}{6!} &= \frac{10\times9\times8\times7\times \not6!}{\not6!} \\
&= \frac{10\times9\times8\times7\times 1}{1} \\
&= 10\times9\times8\times7 = 5040.\\[-0.3cm]
\end{align*}
We could just plug $\frac{10!}{6!}$ into a calculator but the above manipulation is often very useful.\\[0.6cm]

Example: $\frac{100!}{98!}$ may lead to an error on your calculator (model dependent) but it is easy to show that $\frac{100!}{98!} = 100(99) = 9900$.


\end{frame}







\section{Permutations}
\subsection{Permutations}
\begin{frame}{\bf \tcb{Permutations}}

A {\bf permutation} is an \emph{arrangement} of objects $\Rightarrow$ \emph{order matters}.\\[1cm]

For example, consider the PIN code 7162. This is obviously \emph{not} the same as 6172.\\[1cm]

As another example, consider three paintings on a wall labelled 1, 2 and 3. The arrangement 123 does \emph{not} look the same as 213.\\
{\footnotesize\phantom{w}}

\end{frame}


\subsection{Permutations: Repetition}
\begin{frame}{\bf \tcb{Permutations: Repetition}}

\emph{Repetition} of objects may or may not be allowed.\\[1cm]

In the PIN code example, clearly we can have repeated digits, e.g., codes such as 7718, 3633 and 1111 are allowed.\\[0.5cm]

Since there are 10 numbers to choose from, $\{0,1,\ldots,9\}$, there are $10000 = 10(10)(10)(10) = 10^4$ possible four digit PIN codes.\\[1cm]

With $n$ objects and repetitions allowed, there are $\boxed{n^k}$ permutations of length $k$.

\end{frame}

\subsection{Permutations: No Repetition}
\begin{frame}{\bf \tcb{Permutations: No Repetition}}

Now consider the task of arranging three paintings on a wall. If we only have one of each then 221 is \emph{not} allowed {\footnotesize(unless we scan copies!).}\\[0.5cm]

Since repetitions are not allowed, we have 3 choices for the first position, 2 for the second and 1 for the third $\Rightarrow$ $6 = 3(2)(1) = 3!$.\\[0.5cm]

If we were arranging five paintings then there are $5!$ possibilities. Six paintings $\Rightarrow$ $6!$ possibilities.\\[1cm]

Thus, we see that $\boxed{n\,!}$ is the number of arrangements of $n$ objects.

\end{frame}






\subsection{Permutations: Constraints}
\begin{frame}{\bf \tcb{Permutations: Constraints}}
Often there are other \emph{constraints} - we must deal with these on a case-by-case basis.\\[0.8cm]

For example, consider four digit PIN codes with the following constraint: the first digit cannot be zero.\\[0.8cm]

In this case we have 9 choices for the first digit, i.e., $\{1,\ldots,9\}$, and 10 choices for the remaining three digits , i.e., $\{0,1,\ldots,9\}$.\\[0.4cm]

$\Rightarrow$ There are $9(10)(10)(10) = 9000$ PIN codes that satisfy this constraint.


\end{frame}



\subsection{Example: Password}
\begin{frame}{\bf \tcb{Example: Password}}

Let's assume that you have to create a password of length 4 using the following set of characters:
\begin{align*}
\{\,a\,,\,b\,,\,c\,,\,A\,,\,B\,,\,C\,,\,1\,,\,2\,,\,3\,,\,.\,\}\\
\end{align*}

How many possible passwords are there?\\[0.2cm]

There are 10 character choices for each of four positions \\$\Rightarrow$ $10(10)(10)(10) = 10,\!000$.\\[0.5cm]

What if the first character \emph{cannot} be a full stop?\\[0.2cm]

In this case we only have 9 choices for the first position \\$\Rightarrow$ $9(10)(10)(10) = 9,\!000$.

\end{frame}



\subsection{Question 2}
\begin{frame}{\bf \tcb{Question 2}}

Continuing with the example of creating a 4 character password using:
\begin{align*}
\{\,a\,,\,b\,,\,c\,,\,A\,,\,B\,,\,C\,,\,1\,,\,2\,,\,3\,,\,.\,\}
\end{align*}

How many passwords are there with:\\[0.2cm]
\begin{enumerate}[a)]\itemsep0.2cm
\item No repetitions.
\item Exactly one full stop - \emph{not in the first position}.
\item No upper case letters.
\item At least one upper case letter.
\item Using lower case letters and numbers only.
\item Assuming a hacker knows that your first character is either ``a'' or ``A'' and that your last is a number, how many possibilities are there?
\item In addition to the above, now assume that the hacker also knows that the password contains a full stop.
%\item How many passwords are there if they can be 4 - 6 characters in length?
\end{enumerate}


\end{frame}




\section{Combinations}
\subsection{Combinations}
\begin{frame}{\bf \tcb{Combinations}}

A {\bf combination} is a \emph{selection} of objects $\Rightarrow$ \emph{order does not matter}.\\[1cm]

For example, you say you have a pen, a packet of sweets and a euro in your pocket.\\[0.4cm]
It does not matter if you say ``pen, sweets, euro'', ``euro, pen, sweets'', ``sweets, euro, pen'' etc. - in any case it is the same selection of objects.

\end{frame}



\subsection{Examples}
\begin{frame}{\bf \tcb{Examples}}

You have a group assignment and need to choose 3 others to form a team. You have 6 friends who you would choose to be on your team.\\[0.2cm]
$\Rightarrow$ From a group of 6 friends, you must \emph{choose} 3.\\[0.8cm]

You are packing a bag for your holidays, you only have enough room for 4 t-shirts and you own 10 altogether.\\[0.2cm]
$\Rightarrow$  From a group of 10 t-shirts, you must \emph{choose} 4.\\[0.8cm]

There are 5 computer games that you want but you only have enough money with you to purchase 2 of them.\\[0.2cm]
$\Rightarrow$  From a group of 5 games, you must \emph{choose} 2.

\end{frame}






\subsection{Choose Operator}
\begin{frame}{\bf \tcb{Choose Operator}}

The {\bf choose} operator allows us to calculate the number of ways we can \emph{choose} $k$ objects from a group of $n$ objects:\\[-0.1cm]

\begin{align*}
\boxed{\binom{n}{k} = \frac{n\,!}{k\,! \, (n-k)\,!}}.\\
\end{align*}

The choose operator $\binom{n}{k}$ is pronounced ``$n$-choose-$k$'' and can also be written  $^n C _k$. \\[0.6cm]
Note: you have the $\boxed{^n C _k}$ button on your calculator.

\end{frame}


\subsection{Examples}
\begin{frame}{\bf \tcb{Examples}}

From a group of 6 friends, you must choose 3 $\Rightarrow$ \emph{6-choose-3:}
\begin{align*}
\binom{6}{3} = \frac{6!}{3! \, 3!} = \frac{6\times5\times4\times\not3!}{3! \, \not3!} = \frac{6\times5\times4}{3!} = \frac{6\times5\times4}{3\times2\times1} = 20 \text{ choices}.\\
\end{align*}

From a group of 10 t-shirts, you must choose 4 $\Rightarrow$ \emph{10-choose-4:}
\begin{align*}
\binom{10}{4} = \frac{10!}{4! \, 6!} = \frac{10\times9\times8\times7\times\not6!}{4! \, \not6!} =
%\frac{10\times9\times8\times7}{4!} =
\frac{10\times9\times8\times7}{4\times3\times2\times1} = 210 \text{ choices}.\\
\end{align*}

From a group of 5 games, you must choose 2 $\Rightarrow$ \emph{5-choose-2:}
\begin{align*}
\binom{5}{2} = \frac{5!}{2! \, 3!}  = \frac{5\times4\times\not3!}{2!\not3!} = \frac{5\times4}{2\times1} = 10 \text{ choices}.
\end{align*}

\end{frame}
%
%\subsection{Choose Operator - Alternative Formula}
%\begin{frame}{\bf \tcb{Choose Operator - Alternative Formula}}
%The trick used on the previous slide shows that there is an alternative formula for the choose operator:
%
%\begin{align*}
%\boxed{\binom{n}{k} = \frac{n \times (n-1) \times \cdots \times (n-k+1)}{k\,!}}.\\
%\end{align*}
%The top of the fraction is similar to a factorial except we only reduce $n$ as far as $(n-k+1)$, giving $k$ numbers on top.
%
%\end{frame}


\subsection{Examples}
\begin{frame}{\bf \tcb{Examples}}

Consider $\binom{10}{4}$. This leads to a fraction with $4!$ on the bottom and 10 reduced successively by 4 places on the top:\\[-0.3cm]
\begin{align*}
\binom{10}{4} = \frac{10\times9\times8\times7}{4\times3\times2\times1}.\\
\end{align*}

We can often simplify further - by cancelling various numbers.\\[-0.3cm]
\begin{align*}
\frac{10\times9\times\not8\times7}{\not4\times3\times\not2\times1} = \frac{10\times9\times7}{3\times1}
= \frac{10\times\not9\,\,^3\times7}{\not3\times1} &= \frac{10\times3\times7}{1} \\ &= 30 \times 7 = 210.\\
\end{align*}

This is useful if we wish to calculate things by hand.

\end{frame}




\subsection{Example: Probability}
\begin{frame}{\bf \tcb{Example: Probability}}
Earlier (slide \pageref{probslide}) we had the experiment of flipping four coins where\\[0.2cm]
$\#(S) = 2^4 = 16$ and\\[0.2cm]
$A =$ ``two heads'' $= \{HHTT, HTHT, HTTH, THHT, THTH, TTHH\}$\\[0.4cm]

It is possible to list, and then count, all outcomes in $A$, but there is an easier way to count.\\[0.3cm]

Let's think about how we construct the set $A$ above:\\[0.1cm]
\begin{enumerate}[1.]\itemsep0.2cm
\item All outcomes in $A$ are composed using 4 objects: $H$, $H$, $T$ and $T$.
\item From the 4 available slots, we \emph{choose} 2 slots to put the $H$s. The tails go in the remaining slots.\\[0.5cm]
\end{enumerate}


$\Rightarrow$ $\# (A) = \binom{4}{2} =  6$.

\end{frame}


\subsection{Example: Probability}
\begin{frame}{\bf \tcb{Example: Probability}}
Consider a more difficult example of flipping 10 coins where $\#(S) = 2^{10} = 1024$.\\[0.2cm]
Now consider the event $A =$ ``getting four heads''. It is \emph{very} tedious trying to list all outcomes contained in $A$.\\[0.6cm]

We know that each outcome in $A$ is constructed using 4 $H$s and 6 $T$s.\\[0.2cm]
From the 10 slots available, we \emph{choose} 4 of them for the $H$s\\[0.1cm] $\Rightarrow$ $\#(A) = \binom{10}{4} = 210$.\\[0.4cm]

We can then calculate $\Pr(A) = \frac{\#(A)}{\#(S)} = \frac{210}{1024} \approx 0.205$.\\[0.6cm]

{\footnotesize(Note: the \emph{Binomial} distribution which we will encounter later is specifically designed to calculate probabilities like the one above)}

\end{frame}



\subsection{Question 3}
\begin{frame}{\bf \tcb{Question 3}}

Using the choose formula, evaluate (without a calculator!) each of the following and explain their meaning:\\
\begin{align*}
\binom{8}{6}, \quad \binom{8}{2}, \quad \binom{n}{1}, \quad \binom{n}{n}, \quad \binom{n}{0}.\\
\end{align*}

\begin{align*}
\text{What value would you assign to } \binom{8}{11} \text{?}\\[0.1cm]
\text{{\footnotesize(hint: don't attempt to use the choose formula here)}}
\end{align*}

\end{frame}



\subsection{Question 4}
\begin{frame}{\bf \tcb{Question 4}}

A team of 5 people is required to perform a particular task. We are selecting from a group of 7 women and 3 men.\\[0.2cm]
How many selections are there:
\begin{enumerate}[a)]\itemsep0.2cm
\item Altogether?
\item If one of the men is an expert and must be on the team?
\item If two of the individuals do not get along and cannot be on the team together?
\item If the group must contain 3 women and 2 men?
\item If the group must contain more women than men?
\item If the group must contain more men than women?
\end{enumerate}



\end{frame}








\end{document} 