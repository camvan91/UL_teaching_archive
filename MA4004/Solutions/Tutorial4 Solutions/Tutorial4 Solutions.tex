\documentclass[12pt]{article}
\usepackage{amsmath}
%\usepackage[paperwidth=21cm, paperheight=29.8cm]{geometry}
\usepackage[angle=0,scale=1,color=black,hshift=-0.3cm,vshift=15cm]{background}
\usepackage{multirow}
\usepackage{enumerate}
\usepackage[gen]{eurosym}

%\SetBgScale{1}
%\SetBgAngle{0}
%\SetBgColor{black}
%\SetBgContents{\rule{1pt}{30cm}}
%\SetBgHshift{-8.4cm}
%
%\backgroundsetup{contents={
%\begin{tabular}{c|c}
%\hspace{2cm} & \\[0.7cm]
%& {\bf Statistics for Computing ------ Lecture 1 ------ Solutions} \\[0.3cm]
%%\hline
%\hspace{2cm} & \hspace{18.5cm} \\ [28cm]
%\end{tabular}}}

\backgroundsetup{contents={
{\bf \centering Statistics for Computing ------------------------ Tutorial 4 ------------------------------------------ Solutions} }}


\setlength{\voffset}{-3cm}
\setlength{\hoffset}{-3.45cm}
\setlength{\parindent}{0cm}
\setlength{\textheight}{27cm}
\setlength{\textwidth}{19.7cm}

\pagestyle{empty}



\begin{document}


\framebox[1.02\textwidth]{
\begin{minipage}[t]{0.98\textwidth}
\begin{minipage}[t]{0.47\textwidth}
\subsection*{Question 1}
\begin{enumerate}[a)]
\item \quad \\[-1.45cm]
\begin{align*}
&p(0) + p(0.5) + p(1) + p(2) \\
&\qquad= 0.4 + 0.2 + 0.15 + 0.15 \\
&\qquad= 0.9.\\[0.3cm]
\Rightarrow& p(3) = 1 - 0.9 = 0.1.
\end{align*}
Since the probabilities must add to one in a probability distribution.
\item \quad\\[-1.45cm]
\begin{align*}
E(X) &= 0(0.4) + 0.5(0.2) + 1(0.15) +&\\
 &\qquad\qquad2(0.15) + 3(0.1) \\
 &= 0.85.\\[0.3cm]
E(X^2) &= 0^2(0.4) + 0.5^2(0.2) + 1^2(0.15) + &\\
&\qquad\qquad2^2(0.15) + 3^2(0.1) \\
&= 1.7.\\[0.3cm]
Var(X) &= E(X^2) - [E(X)]^2 \\
&= 1.7 - (0.85^2) =0.9775. \\[0.3cm]
Sd(X) &= \sqrt{Var(X)} = \sqrt{0.9775} = 0.9887. \\
\end{align*}
\end{enumerate}
\end{minipage}\hspace{0.055\textwidth}
\begin{minipage}[t]{0.47\textwidth}
\begin{enumerate}
\item[c)] Since $E(X) = 0.85$, the player will win \euro{0.85} on average per game. We should charge \euro{0.95} for a play of this game so that the average amount won is then \euro{\,-0.10}, i.e., a \emph{loss} of \euro{0.10}.
\item[d)] We are charging \euro{0.95} for a play of the game. If the player wins less than this, we make a profit. Thus, if the player wins \euro{0.00} or \euro{0.50} we make a profit.
    \begin{align*}
    \Pr(\text{we profit}) &= p(0) + p(0.5) \\&= 0.4 + 0.2 = 0.6.
    \end{align*}
$\Rightarrow$ If somebody plays our game, there is a $0.6$ probability that we profit.
\item[e)] Each play of the game is a Bernoulli trial where $p=0.6$ is the probability that we profit. 10 games $\Rightarrow$ 10 Bernoulli trials $\Rightarrow$ Binomial$(n=10,p=0.6)$.\\[0.2cm]
    Let $Y =$ ``the number of times we profit'':
    \begin{align*}
    \Pr(Y = 8) &= \binom{10}{8} \, (0.6^8) \, (0.4^2) \\&= 0.1209.
    \end{align*}
\end{enumerate}
\end{minipage}
\end{minipage}}\vspace{0.03\textwidth}


\framebox[1.02\textwidth]{
\begin{minipage}[t]{0.98\textwidth}
\begin{minipage}[t]{1\textwidth}
\subsection*{Question 2}
\begin{enumerate}[a)]
\item The sample space is $S = \{HHH, THH, HTH, TTH, HHT, THT, HTT, TTT\}$.
\begin{center}
\begin{tabular}{|c|c|c|c|c|c|c|c|c|}
\hline
&&&&&&&&\\[-0.4cm]
Outcome & $HHH$ & $THH$ & $HTH$ & $TTH$ & $HHT$ & $THT$ & $HTT$ & $TTT$ \\
\hline
&&&&&&&&\\[-0.4cm]
$X =$ ``no. of heads'' & $3$ & $2$ & $2$ & $1$ & $2$ & $1$ & $1$ & $0$ \\
\hline
&&&&&&&&\\[-0.4cm]
$Y =$ ``no. of unique faces'' & $1$ & $2$ & $2$ & $2$ & $2$ & $2$ & $2$ & $1$ \\[0.1cm]
\hline
\end{tabular}
\end{center}
\end{enumerate}
\end{minipage}\vspace{0.03\textwidth}
\begin{minipage}[t]{0.47\textwidth}
\begin{enumerate}[a)]
\item[b)] From the above we get the joint distribution:
\begin{center}
\begin{tabular}{c|c|cccc|c|}
\multicolumn{2}{c}{} & \multicolumn{4}{c}{$X$} & \multicolumn{1}{c}{}\\
\cline{2-7}
&&&&&&\\[-0.4cm]
&&                          0 & 1 & 2 & 3 & $p(y)$\\
\cline{2-7}
&&&&&&\\[-0.3cm]
\multirow{2}{*}{$Y$} & 1 & $\frac{1}{8}$ &  $0$ & $0$ & $\frac{1}{8}$ & $\frac{2}{8}$ \\[0.1cm]
                     & 2 & $0$           &  $\frac{3}{8}$  & $\frac{3}{8}$ & $0$ & $\frac{6}{8}$ \\[0.1cm]
\cline{2-7}
&&&&&&\\[-0.3cm]
& $p(x)$ & $\frac{1}{8}$ & $\frac{3}{8}$ & $\frac{3}{8}$ & $\frac{1}{8}$ & 1 \\[0.1cm]
\cline{2-7}
\multicolumn{7}{c}{}
\end{tabular}
\end{center}
\item[c)] The marginal distributions for $X$ and $Y$ are shown in the margins of the table above.
\end{enumerate}
\end{minipage}\hspace{0.055\textwidth}
\begin{minipage}[t]{0.47\textwidth}
\begin{enumerate}[a)]
\item[d)] If $X$ and $Y$ were independent then:
\begin{center}
\begin{tabular}{c|c|cccc|c|}
\multicolumn{2}{c}{} & \multicolumn{4}{c}{$X$} & \multicolumn{1}{c}{}\\
\cline{2-7}
&&&&&&\\[-0.4cm]
&&                          0 & 1 & 2 & 3 & $p(y)$\\
\cline{2-7}
&&&&&&\\[-0.3cm]
\multirow{2}{*}{$Y$} & 1 & $\frac{2}{64}$ &  $\frac{6}{64}$ & $\frac{6}{64}$ & $\frac{2}{64}$ & $\frac{2}{8}$ \\[0.1cm]
                     & 2 &  $\frac{6}{64}$ &  $\frac{18}{64}$   & $\frac{18}{64}$  & $\frac{6}{64}$  & $\frac{6}{8}$ \\[0.1cm]
\cline{2-7}
&&&&&&\\[-0.3cm]
& $p(x)$ & $\frac{1}{8}$ & $\frac{3}{8}$ & $\frac{3}{8}$ & $\frac{1}{8}$ & 1 \\[0.1cm]
\cline{2-7}
\multicolumn{7}{c}{}
\end{tabular}
\end{center}
Clearly this is \emph{not} the joint distribution as in part (b) $\Rightarrow$ $X$ and $Y$ are dependent.
\end{enumerate}
\end{minipage}
\end{minipage}}\vspace{0.03\textwidth}

\framebox[1.02\textwidth]{
\begin{minipage}[t]{0.98\textwidth}
\begin{minipage}[t]{0.47\textwidth}
\subsection*{Question 2 continued}
\begin{enumerate}[a)]
\item[e)] \quad\\[-1.45cm]
\begin{align*}
E(Y) &= 1\left(\frac{2}{8}\right) + 2\left(\frac{6}{8}\right) \\
 &= 1.75 \text{ unique faces}.\\[0.3cm]
E(Y^2) &= 1^2\left(\frac{2}{8}\right) + 2^2\left(\frac{6}{8}\right) \\
 &= 3.25.\\[0.3cm]
Var(Y) &= 3.25 - (1.75^2) \\ &= 0.1875 \text{ unique faces$^2$}. \\[0.3cm]
Sd(X) &= \sqrt{0.1875} \\&= 0.433 \text{ unique faces}.
\end{align*}
\end{enumerate}
\end{minipage}\hspace{0.055\textwidth}
\begin{minipage}[t]{0.47\textwidth}
\quad
\begin{enumerate}
\item[f)] \quad\\[-1.45cm]
\begin{align*}
\Pr(Y=2\,|\,X=2) &= \frac{\Pr(Y=2\cap X=2)}{\Pr(X=2)} \\
&= \frac{3/ 8}{3 / 8} =1.
\end{align*}
Given that there are two heads ($X=2$), we are certain that there are two unique faces ($Y=2$). Hence $\Pr(Y=2\,|\,X=2)=1$.\\[0.2cm]
Note that $\Pr(Y=2) = \frac{6}{8}$ when we do not have any information.
\end{enumerate}
\end{minipage}
\end{minipage}}\vspace{0.03\textwidth}


\framebox[1.02\textwidth]{
\begin{minipage}[t]{0.98\textwidth}
\begin{minipage}[t]{0.47\textwidth}
\subsection*{Question 3}
\begin{enumerate}[a)]
\item \quad\\[-1.45cm]
\begin{align*}
E(X) &= 0(0.2) + 100(0.75) + 300(0.05) \\
&= 90. \\[0.2cm]
E(Y) &= 0(0.1) + 80(0.6) + 200(0.3) \\
&= 108.
\end{align*}
\item P1 has an average attack power of $E(X) = 90$ and P2 has 1000 hit points.\\
    $\Rightarrow$ On average it takes $\frac{1000}{90} = 11.11$ attacks to defeat P2.\\[0.3cm]
    P2 has an average attack power of $E(Y) = 108$ and P1 has 1000 hit points.\\
    $\Rightarrow$ On average it takes $\frac{1000}{108} = 9.25$ attacks to defeat P1.\\[0.3cm]
    Thus, P2 will win on average (note: this does \emph{not} mean that P1 cannot win - it is just less likely).
\item We split this into two parts: the first turn and then the battle after the first turn.\\[0.3cm]
    On the first turn, P1 casts a spell and does not attack $\Rightarrow$ after turn 1, P2's life is still 1000.\\[0.2cm]
    On the first turn, P2 has an average attack power of 108 $\Rightarrow$ P1's life is 1000 - 108 = 892 after the first turn (on average).\\
\end{enumerate}
\end{minipage}\hspace{0.055\textwidth}
\begin{minipage}[t]{0.47\textwidth}
\begin{enumerate}[a)]
\item[] From turn two onwards, P2 can no longer deal a critical blow, i.e., $p(200) = 0$. We distribute the remaining 0.3 probability evenly across the other outcomes $\Rightarrow$ $p(0) = 0.1+0.15 = 0.25$ and $p(80) = 0.6 + 0.15 = 0.75$.\\[0.2cm]
    Thus, P2's average attack is $E(Y) = 0(0.25) + 80(0.75) = 60$ from turn 2 onwards.
    \\[0.4cm]
    We can think of the rest of the battle as a new battle. P1 has 892 hit points and average attack $E(X) = 90$. P2 has 1000 hit points and average attack $E(Y) = 60$.\\[0.4cm]
    On average it takes $\frac{1000}{90} = 11.11$ attacks for P1 to defeat P2 (after the first turn).\\[0.2cm]
    On average it takes $\frac{892}{60} = 14.87$ attacks for P2 to defeat P1 (after the first turn).\\[0.2cm]
    Therefore, P1 is more likely to win the fight in this scenario.
\end{enumerate}
\end{minipage}
\end{minipage}}\vspace{0.03\textwidth}



\framebox[1.02\textwidth]{
\begin{minipage}[t]{0.98\textwidth}
\begin{minipage}[t]{0.47\textwidth}
\subsection*{Question 4}
$X \sim \text{Binomial}(n=10, p=0.5)$.\\[0.5cm]
$\Rightarrow$ The \emph{probability function} is:\\$p(x) = \binom{10}{x}\,\,0.5^x\,\,0.5^{10-x}$,\\[0.2cm]
and, since $0.5^x\,0.5^{10-x} = 0.5^{x+10-x} = 0.5^{10}$, we can simplify:
$p(x) = \binom{10}{x}\,\,0.5^{10} $.\\[0.2cm]
Note: this only possible when $p=0.5$.
\begin{enumerate}[a)]
\item \quad\\[-1.45cm]
\begin{align*}
\Pr(X=2) = \binom{10}{2}\,\,0.5^{10} = 0.0439.\\[-0.2cm]
\end{align*}
\item \quad\\[-1.45cm]
\begin{align*}
\Pr(X=0) = \binom{10}{0}\,\,0.5^{10} = 0.00098.\\[-0.5cm]
\end{align*}
{\bf To save space I will write $p(x)$ from now on. In the exam, please show your work as in parts (a) and (b) above.}
\item \quad\\[-1.45cm]
\begin{align*}
\Pr(X>2) &= 1 - \Pr(X \le 2) \\
&= 1 - [p(0) + p(1) + p(2)] \\[0.2cm]
%&= 1 - \left[\binom{10}{0}\,\,0.5^{10} + \binom{10}{1}\,\,0.5^{10}\right. +\\ &\hspace{4cm}\left.\binom{10}{2}\,\,0.5^{10}\right] \\[0.2cm]
&= 1 - [0.00098 + 0.0098 + 0.0439] \\[0.2cm]
&= 1 - 0.0547 = 0.9453. \\[-0.2cm]
\end{align*}
\item \quad\\[-1.45cm]
\begin{align*}
\Pr(X \le 3) &= p(0) + p(1) + p(2) + p(3) \\[0.2cm]
&= 0.00098 + 0.0098 + 0.0439 + 0.1172 \\[0.2cm]
&= 0.1719. \\[0.2cm]
\end{align*}
\end{enumerate}
\end{minipage}\hspace{0.04\textwidth}
\begin{minipage}[t]{0.47\textwidth}
\begin{enumerate}
\item[e)] \quad\\[-1.45cm]
\begin{align*}
\Pr(5 \le X \le 7) &= p(5) + p(6) + p(7) \\[0.2cm]
&=  0.2461 + 0.2051 + 0.1172 \\[0.2cm]
&= 0.5684. \\[-0.2cm]
\end{align*}
\item[f)] \quad\\[-1.45cm]
\begin{align*}
E(X) &= n\,p = 10(0.5) = 5 \text{ heads}.\\[0.3cm]
Var(X) &= n\,p\,(1-p) \\&= 10(0.5)(0.5) = 2.5 \text{ heads$^2$}.\\[0.3cm]
Sd(X) &= \sqrt{2.5} = 1.58 \text{ heads}.\\[-0.3cm]
\end{align*}
\item[g)] Here $X$ is now Binomial$(n=20,p=0.5)$. Calculating $\Pr(X \le 10) = p(0) + p(1) + \ldots + p(9) + p(10)$ is tedious using the probability function and $1 - \Pr(X > 10)$ is no easier using this approach.\\[0.2cm]
    With the tables we must rewrite the problem in terms of ``$\ge$'' probabilities.
\begin{align*}
\Pr(X \le 10) &= 1 - \Pr(X > 10) \\
&= 1 - \Pr(X \ge 11) \\
&= 1 - 0.4119 = 0.5581.
\end{align*}
\item[h)] $E(X) = n\,p = 50(0.5) = 25$ heads.
\end{enumerate}
\end{minipage}
\end{minipage}}\vspace{0.03\textwidth}


\framebox[1.02\textwidth]{
\begin{minipage}[t]{0.98\textwidth}
\begin{minipage}[t]{0.47\textwidth}
\subsection*{Question 5}
Same as above but now using the binomial tables. Go to column $p=0.5$ and row $n=10$.
\begin{enumerate}[a)]
\item \quad\\[-1.45cm]
\begin{align*}
\Pr(X=2) &= \Pr(X\ge2) - \Pr(X\ge3) \\
&= 0.9893 - 0.9453 \\
&= 0.0440.\\[-0.2cm]
\end{align*}
\item \quad\\[-1.45cm]
\begin{align*}
\Pr(X=0) &= \Pr(X\ge0) - \Pr(X\ge1) \\
&= 1.0000 - 0.9990 \\
&= 0.001.\\[-0.2cm]
\end{align*}
\item \quad\\[-1.45cm]
\begin{align*}
\Pr(X>2) = \Pr(X \ge 3) = 0.9453.\\
\end{align*}
\end{enumerate}
\end{minipage}\hspace{0.055\textwidth}
\begin{minipage}[t]{0.47\textwidth}
\begin{enumerate}[a)]
\item[d)] \quad\\[-1.45cm]
\begin{align*}
\Pr(X\le3) &= 1 - \Pr(X>3) \\
&=  1 - \Pr(X\ge4) \\
&= 1-0.8281 = 0.1719.\\[-0.2cm]
\end{align*}
\item[e)] \quad\\[-1.45cm]
\begin{align*}
\Pr(5\le X\le7) &= \Pr(X\ge5)-\Pr(X\ge8) \\
&=  0.6230 - 0.0547 \\
&= 0.5683.\\[-0.2cm]
\end{align*}
We can see that these are the same as before apart from small differences due to rounding.
\end{enumerate}
\end{minipage}
\end{minipage}}\vspace{0.03\textwidth}



\framebox[1.02\textwidth]{
\begin{minipage}[t]{0.98\textwidth}
\begin{minipage}[t]{0.47\textwidth}
\subsection*{Question 6}
Let $X =$ ``the number of errors in a string of bits''.
\begin{enumerate}[a)]
\item $X \sim \text{Binomial}(n=20,p=0.1)$.
\begin{align*}
\Pr(X = 0) = \binom{20}{0}\,0.1^0\,\,0.9^{20} = 0.1216.
\end{align*}
\item $X \sim \text{Binomial}(n=10,p=0.1)$.
\begin{align*}
\Pr(X < 3) &= \Pr(X \le 2) \\
 &= p(0) + p(1) + p(2) \\[0.2cm]
 &= \binom{10}{0}\,0.1^0\,\,0.9^{10} + \binom{10}{1}\,0.1^1\,\,0.9^{9} + \\ &\qquad\qquad\binom{10}{2}\,0.1^2\,\,0.9^{8}\\[0.2cm]
 &= 0.3487 + 0.3874 + 0.1937 = 0.9298.
\end{align*}
Note: (a) and (b) can also be done using tables.
\end{enumerate}
\end{minipage}\hspace{0.055\textwidth}
\begin{minipage}[t]{0.47\textwidth}
It is essential to use the tables for part (c) since there is far too much work involved if we do not use the tables.
\begin{enumerate}[a)]
\item[c)] (i) $X \sim \text{Binomial}(n=50,p=0.1)$\\
$\Rightarrow$ tables: column $p=0.1$, row $n=50$.
\begin{align*}
\Pr(X>10) &= \Pr(X\ge11) = 0.0094.\\[-0.2cm]
\end{align*}
(ii) $X \sim \text{Binomial}(n=100,p=0.1)$\\
$\Rightarrow$ tables: column $p=0.1$, row $n=100$.
\begin{align*}
\Pr(X>10) &= \Pr(X\ge11) = 0.4168.\\[-0.2cm]
\end{align*}
\item[d)] \quad\\[-1.45cm]
\begin{align*}
E(X) &= n\,p = 100(0.1) = 10 \text{ errors}.\\[0.3cm]
Var(X) &= n\,p\,(1-p) \\&= 100(0.1)(0.9) = 9 \text{ errors$^2$}.\\[0.3cm]
Sd(X) &= \sqrt{9} = 3 \text{ errors}.\\[-0.3cm]
\end{align*}
\end{enumerate}
\end{minipage}
\end{minipage}}\vspace{0.03\textwidth}



\framebox[1.02\textwidth]{
\begin{minipage}[t]{0.98\textwidth}
\begin{minipage}[t]{0.47\textwidth}
\subsection*{Question 7}
Let $Y =$ ``the number of errors in three replicates''.
$\Rightarrow$ $Y \sim \text{Binomial}(n=3,p=0.1)$
\begin{enumerate}[a)]
\item The bit will be assigned the wrong value if there are two or three errors:
\begin{align*}
\Pr(\text{error}) &= \Pr(Y\ge2) \\ &= p(2) + p(3)\\
&= \binom{3}{2}\,0.1^2\,\,0.9^1 + \binom{3}{3}\,0.1^3\,\,0.9^0\\
&= 0.027 + 0.001 \\
&= 0.028.
\end{align*}
\item $X =$ ``the number of errors in a string of bits'' as before. However, since each bit is replicated three times, the probability of error is $p = 0.028$.\\[0.2cm]
    $\Rightarrow$ $X \sim \text{Binomial}(n=20,p=0.028)$.
\begin{align*}
\Pr(X = 0) = \binom{20}{0}\,0.028^0\,\,0.972^{20} = 0.5667.
\end{align*}
\end{enumerate}
\end{minipage}\hspace{0.055\textwidth}
\begin{minipage}[t]{0.47\textwidth}
\begin{enumerate}[a)]
\item[c)] Here $Y \sim \text{Binomial}(n=5,p=0.1)$ and a bit will be assigned the wrong value if there are \emph{three or more} errors:
\begin{align*}
\Pr(\text{error}) &= \Pr(Y\ge3) = p(3) + p(4) + p(5)\\[0.2cm]
&= \binom{5}{3}\,0.1^3\,\,0.9^2 + \binom{5}{4}\,0.1^4\,\,0.9^1 +\\
&\hspace{2.5cm} \binom{5}{5}\,0.1^5\,\,0.9^0\\[0.2cm]
&= 0.0081 + 0.00045 + 0.00001 \\
&= 0.00856.\\[-0.8cm]
\end{align*}
{\footnotesize(it should be clear that by increasing replication we can reduce the error probability to any desired level)}\\[0.4cm]
Thus, with this error probability, the number of errors in a 20-bit string is has the following distribution:\\[0.2cm]
$X \sim \text{Binomial}(n=20,p=0.00856)$
\begin{align*}
\Rightarrow \Pr(X = 0) &= \binom{20}{0}\,0.00856^0\,\,0.99144^{20} \\[0.2cm]
&= 0.8420.
\end{align*}
\end{enumerate}
\end{minipage}
\end{minipage}}\vspace{0.03\textwidth}









\end{document} 