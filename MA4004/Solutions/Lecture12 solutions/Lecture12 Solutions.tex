\documentclass[12pt]{article}
\usepackage{amsmath}
%\usepackage[paperwidth=21cm, paperheight=29.8cm]{geometry}
\usepackage[angle=0,scale=1,color=black,hshift=-0.3cm,vshift=15cm]{background}
\usepackage{multirow}
\usepackage{enumerate}
\usepackage{changepage}

\backgroundsetup{contents={
{\bf \centering Statistics for Computing ------------------------ Lecture 12 ------------------------------------------ Solutions} }}


\setlength{\voffset}{-3cm}
\setlength{\hoffset}{-3.45cm}
\setlength{\parindent}{0cm}
\setlength{\textheight}{27cm}
\setlength{\textwidth}{19.7cm}

\pagestyle{empty}



\begin{document}



\framebox[1.02\textwidth]{
\begin{minipage}[t]{0.98\textwidth}
\begin{minipage}[t]{0.47\textwidth}
\subsection*{Question 1}
\begin{enumerate}[a)]
\item \quad\\[-1.45cm]
\begin{align*}
X &\sim \text{Normal}(\mu=60,\sigma=10)\\[0.5cm]
\Rightarrow \,\overline{\!X} &\sim \text{Normal}\left(\mu=60,\,\sigma(\,\overline{\!X})=\frac{\sigma}{\sqrt{n}}=\frac{10}{\sqrt{n}}\right).
\end{align*} 
\item Here $\,\overline{\!X} \sim \text{Normal}(60, \frac{10}{\sqrt{30}}=1.826)$.
\begin{align*}
\Pr(\,\overline{\!X} > 66) &= \Pr(Z > \tfrac{66-60}{1.826}) \\
&= \Pr(Z > 3.29) \\
&=0.0005.
\end{align*}
\item 99\% limits $\Rightarrow$ $\alpha=0.1$ $\Rightarrow$ $\alpha/2=0.005$.
\begin{align*}
60 &\pm z_{\,0.005} (1.826) \\
60 &\pm 2.58 (1.826) \\
60 &\pm 4.71 \\[0.2cm]
[55.29&,\,64.71].
\end{align*}
99\% of the time, the sample mean for 30 individuals will lie in the above interval.
\end{enumerate}
\end{minipage}\hspace{0.055\textwidth}
\begin{minipage}[t]{0.47\textwidth}
\begin{enumerate}[c)]
\item[d)] Here $\sigma(\,\overline{\!X}) = \frac{10}{\sqrt{50}}=1.414$
\begin{align*}
60 &\pm 2.58 (1.414) \\
60 &\pm 3.65 \\[0.2cm]
[56.35&,\,63.65].
\end{align*}
Note that for $n =50$ the interval is tighter (around $\mu$) than for $n=30$, i.e., $\,\overline{\!X}$ varies less in the bigger sample as we would expect.
\end{enumerate}
\end{minipage}
\end{minipage}}\vspace{0.03\textwidth}



\end{document}



