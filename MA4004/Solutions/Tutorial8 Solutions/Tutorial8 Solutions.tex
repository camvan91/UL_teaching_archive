\documentclass[12pt]{article}
\usepackage{amsmath}
%\usepackage[paperwidth=21cm, paperheight=29.8cm]{geometry}
\usepackage[angle=0,scale=1,color=black,hshift=-0.3cm,vshift=15cm]{background}
\usepackage{multirow}
\usepackage{enumerate}
\usepackage[gen]{eurosym}
\usepackage{tikz}
\usetikzlibrary{shapes}
\usepackage[all]{xy}

%\SetBgScale{1}
%\SetBgAngle{0}
%\SetBgColor{black}
%\SetBgContents{\rule{1pt}{30cm}}
%\SetBgHshift{-8.4cm}
%
%\backgroundsetup{contents={
%\begin{tabular}{c|c}
%\hspace{2cm} & \\[0.7cm]
%& {\bf Statistics for Computing ------ Lecture 1 ------ Solutions} \\[0.3cm]
%%\hline
%\hspace{2cm} & \hspace{18.5cm} \\ [28cm]
%\end{tabular}}}

\backgroundsetup{contents={
{\bf \centering Statistics for Computing ------------------------ Tutorial 8 ------------------------------------------ Solutions} }}


\setlength{\voffset}{-3cm}
\setlength{\hoffset}{-3.45cm}
\setlength{\parindent}{0cm}
\setlength{\textheight}{27cm}
\setlength{\textwidth}{19.7cm}

\pagestyle{empty}



\begin{document}





\framebox[1.02\textwidth]{
\begin{minipage}[t]{0.98\textwidth}
\begin{minipage}[t]{0.47\textwidth}
\subsection*{Question 1}
\begin{enumerate}[a)]
\item The data is categorical, i.e., people who use Android and people who don't.\\[0.2cm]
    For categorical data we calculate a proportion.
\item The parameter here is the true proportion of individuals who use an Android device.\\[0.2cm]
    It's value is unknown, i.e., $p = $ unknown.
\item The statistic is the proportion calculated in our sample. It provides us with an estimate of $p$ which is unknown.\\[0.2cm]
    Thus, $\hat p = \frac{359}{500} = 0.718$.
\item 95\% confidence interval $\Rightarrow$ $\alpha = 0.05$ remaining $\Rightarrow$ $\alpha / 2 = 0.025$ in each tail.\\[0.3cm]
    The 95\% confidence interval for $p$ is:
\begin{align*}
\hat p &\pm z_{\,0.025} \, \sqrt{\frac{\hat p(1- \hat p)}{n}} \\[0.2cm]
0.718 &\pm 1.96 \, \sqrt{\frac{0.718(0.282)}{500}} \\[0.2cm]
0.718 &\pm 1.96 \, (0.0201) \\[0.1cm]
0.718 &\pm 0.0394 \\[0.2cm]
[0.6786,&\,\,0.7574]
\end{align*}
Thus, we are 95\% confident that the true proportion, $p$, is between 0.6786 and 0.7574, i.e., that the true market share is between 67.86\% and 75.74\%.
\end{enumerate}
\end{minipage}\hspace{0.04\textwidth}
\begin{minipage}[t]{0.47\textwidth}
\quad\\[-1cm]
\begin{enumerate}
\item[e)] In part (d) note that the margin of error was $\pm 0.0394$. We want to find the value of $n$ that makes this equal to $\pm 0.02$.
\begin{align*}
\Rightarrow 1.96 \, \sqrt{\frac{0.718(0.282)}{n}} &= 0.02\\[0.3cm]
1.96 \, \frac{\sqrt{0.718(0.282)}}{\sqrt{n}} &= 0.02\\[0.3cm]
\frac{1.96 \, \sqrt{0.718(0.282)}}{\sqrt{n}} &= 0.02
\end{align*}
\begin{align*}
\Rightarrow \frac{1}{\sqrt{n}} &= \frac{0.02}{1.96 \, \sqrt{0.718(0.282)}}\\[0.3cm]
\sqrt{n} &= \frac{1.96 \, \sqrt{0.718(0.282)}}{0.02}\\[0.3cm]
n &= \left(\frac{1.96 \, \sqrt{0.718(0.282)}}{0.02}\right)^2\\[0.3cm]
&= \frac{(1.96^2) \, [0.718(0.282)]}{(0.02)^2}\\[0.3cm]
&= 1944.57.
\end{align*}
Thus, we need about 1945 individuals to reduce the margin of error to $\pm 0.02$ (assuming that $\hat p$ will still be approximately 0.718 in the next sample).
\end{enumerate}
\end{minipage}
\end{minipage}}\vspace{0.03\textwidth}


\framebox[1.02\textwidth]{
\begin{minipage}[t]{0.98\textwidth}
\begin{minipage}[t]{0.47\textwidth}
\subsection*{Question 2}
\begin{enumerate}[a)]
\item The data type is numeric continuous, i.e., money takes decimal values and is generally considered continuous.\\[0.2cm]
    In particular, since it is a numeric measurement we calculate the mean.
\item There is no $p$ or $\hat p$ in this case since we are not dealing with categorical data.\\[0.2cm]
    We do have the following quantities however:\\[0.2cm]
    $\bullet$ $\mu =$ unknown.\\[0.2cm]
    $\bullet$ $\bar x = 42.38$.\\[0.2cm]
    $\bullet$ $\sigma =$ unknown.\\[0.2cm]
    $\bullet$ $s = 16.80$.\\
\end{enumerate}
\end{minipage}\hspace{0.04\textwidth}
\begin{minipage}[t]{0.47\textwidth}
\quad\\[-1cm]
\begin{enumerate}
\item[c)] 99\% confidence interval $\Rightarrow$ $\alpha = 0.01$ remaining $\Rightarrow$ $\alpha / 2 = 0.005$ in each tail.\\[0.3cm]
    The 99\% confidence interval for $\mu$ is:
\begin{align*}
\bar x &\pm z_{\,0.005} \, \frac{s}{\sqrt{n}} \\[0.2cm]
42.38 &\pm 2.58 \, \frac{16.80}{\sqrt{1000}} \\[0.2cm]
42.38 &\pm 2.58 \, (0.5313) \\[0.2cm]
42.38 &\pm 1.3707 \\[0.2cm]
[41.01,&\,\,43.75]
\end{align*}
We are 99\% confident that $\mu$ is in the above interval, i.e., that people who click on this Ad go on to spend between \$41.01 and \$43.75.
\end{enumerate}
\end{minipage}
\end{minipage}}\vspace{0.03\textwidth}


\framebox[0.5\textwidth]{
\begin{minipage}[t]{0.46\textwidth}
\subsection*{Question 2 (continued)}
\begin{enumerate}[a)]
\item[d)] We wish to be 99\% confident that the margin of error is $\pm 0.50$ (rather than $\pm 1.3707)$.
    \begin{align*}
    z_{\,0.005} \, \frac{s}{\sqrt{n}} &= 0.5 \\[0.3cm]
    2.58 \, \frac{16.80}{\sqrt{n}} &= 0.5 \\[0.3cm]
    \frac{2.58(16.80)}{\sqrt{n}} &= 0.5 \\[0.3cm]
    \frac{1}{\sqrt{n}} &= \frac{0.5}{2.58(16.80)} \\[0.3cm]
    \sqrt{n} &= \frac{2.58(16.80)}{0.5} \\[0.3cm]
    n &= \left(\frac{2.58(16.80)}{0.5}\right)^2 \\[0.3cm]
    &=7514.81.
    \end{align*}
Thus, we need about 7515 individuals to ensure a margin of error of $\pm 0.5$, i.e., to be within $\pm$ 50c of the true average spend (assuming that $s$ is still approximately equal to $16.80$ in the next sample)
\end{enumerate}
\end{minipage}}\hspace{0.015\textwidth}
\framebox[0.5\textwidth]{
\begin{minipage}[t]{0.46\textwidth}
\subsection*{Question 3}
\begin{enumerate}[a)]
\item Time taken $\Rightarrow$ numeric continuous. Thus we will be dealing with a mean.
\item $\mu =$ unknown.
\item $\bar x =$ 671.23 hours.
\item We are sample variance $s^2 = 400 $ hours$^2$.\\
The standard deviation is $s = \sqrt{400} = 20$ hours.\\[0.3cm]
99.9\%, i.e., $1-\alpha = 0.999$ $\Rightarrow$ $\alpha = 0.001$ remaining $\Rightarrow$ $\alpha/2 = 0.0005$ in each tail.
\begin{align*}
\bar x &\pm z_{\,0.0005} \, \frac{s}{\sqrt{n}} \\[0.2cm]
671.23 &\pm 3.29 \, \frac{20}{\sqrt{45}} \\[0.2cm]
671.23 &\pm 3.29 \,(2.981) \\[0.2cm]
671.23 &\pm 9.8089 \\[0.2cm]
[661.42,&\,\,681.04]
\end{align*}
We are 99.9\% confident that $\mu$ is in the above interval. In other words, this particular component is almost certain to last between 661.42 hours and 681.04 hours.
\end{enumerate}
\end{minipage}}\vspace{0.03\textwidth}



\framebox[1.02\textwidth]{
\begin{minipage}[t]{0.98\textwidth}
\begin{minipage}[t]{0.47\textwidth}
\subsection*{Question 4}
\begin{enumerate}[a)]
\item The true difference in proportions is unknown, i.e., $p_1 - p_2 =$ unknown.
\item We will let index 1 correspond to rural and index 2 correspond to urban:\\[0.1cm]
    $\hat p_1 = 0.5263$ and $n_1 = 38$;\\[0.1cm]
    $\hat p_2 = 0.6034$ and $n_2 = 116$.\\[0.3cm]
90\% confidence $\Rightarrow$ $\alpha = 0.1$ remaining $\Rightarrow$ $\alpha/2 = 0.05$ in each tail.
\begin{small}
\begin{align*}
(\hat p_1 - \hat p_2) &\pm z_{\,0.05} \, \sqrt{\frac{\hat p_1(1-\hat p_1)}{n_1}+\frac{\hat p_2(1-\hat p_2)}{n_2}} \\[0.2cm]
\hspace{-0.9cm}(0.5263 - 0.6034) &\pm 1.64 \, \sqrt{\frac{0.5263(0.4737)}{38}+\frac{0.6034(0.3966)}{116}} \\[0.2cm]
-0.0771 &\pm 1.64 \, (0.0929) \\[0.2cm]
-0.0771 &\pm 0.1523 \\[0.2cm]
[-0.2294,&\,\,0.0752]
\end{align*}
\end{small}
We are 90\% confident that the true difference, $p_1 - p_2$, is contained in the calculated interval.
\end{enumerate}
\end{minipage}\hspace{0.01\textwidth}
\begin{minipage}[t]{0.48\textwidth}
\quad\\[-1.4cm]
\begin{enumerate}
\item[] In particular, note that the interval supports the possibility of no difference: $p_1 - p_2 = 0$. Thus, we conclude that there appears to be no difference in opinions.% of people in rural and urban areas in relation to this new policy.
\item[c)] 52.63\% of 38 $\Rightarrow$ $0.5263(38) = 20$ individuals.\\[0.3cm]
60.34\% of 116 $\Rightarrow$\! $0.6034(116) = 70$ individuals.\\[0.3cm]
$\Rightarrow$ $20+90 = 90$ individuals overall from the group of $154$ who are in favour of the policy.
\item[d)] Overall we have $\hat p = \frac{90}{154} = 0.5844.$
    %90\% confidence for the overall $p$:
\begin{align*}
\hat p &\pm z_{\,0.05} \, \sqrt{\frac{\hat p(1- \hat p)}{n}} \\[0.2cm]
0.5844 &\pm 1.64 \, \sqrt{\frac{0.5844(0.4156)}{154}} \\[0.2cm]
0.5844 &\pm 0.0651 \\[0.1cm]
[0.5193,&\,\,0.6495]\\[-0.8cm]
\end{align*}
We are 90\% confident that the true overall proportion, $p$, of individuals in support of the policy is in the above interval.
\end{enumerate}
\end{minipage}
\end{minipage}}\vspace{0.03\textwidth}




\framebox[1.02\textwidth]{
\begin{minipage}[t]{0.98\textwidth}
\begin{minipage}[t]{0.47\textwidth}
\subsection*{Question 5}
\begin{enumerate}[a)]
\item The parameter is the true difference between mean temperatures: $\mu_1 - \mu_2$. Of course, its value is unknown.
\item Note that we have:\\[0.2cm]
$n_1 = 50$, $\bar x_1 = 40.1$, $s_1 = 2.5$.\\[0.2cm]
$n_2 = 50$, $\bar x_2 = 34.8$, $s_2 = 1.1$.\\[0.2cm]
95 \% confidence $\Rightarrow$ $\alpha = 0.05$ remaining = $\alpha/2= 0.025$ in each tail.
\begin{align*}
(\bar x_1 - \bar x_2) &\pm z_{\,0.025} \, \sqrt{\frac{s_1^2}{n_1}+\frac{s_2^2}{n_2}} \\[0.2cm]
(40.1 - 34.8) &\pm 1.96 \, \sqrt{\frac{(2.5)^2}{50}+\frac{(1.1)^2}{50}} \\[0.2cm]
5.3 &\pm 1.96 \, \sqrt{0.125+0.0242} \\[0.2cm]
5.3 &\pm 1.96 \, (0.3863) \\[0.2cm]
5.3 &\pm 0.7571 \\[0.2cm]
[4.54,&\,\,6.06]
\end{align*}
We are 95\% confident that the true difference is between 4.54 and 6.06, i.e., using a Type 1 heat sink leads to a hotter CPU. Thus, we should use Type 2 heat sinks.
\end{enumerate}
\end{minipage}\hspace{0.04\textwidth}
\begin{minipage}[t]{0.47\textwidth}
\quad\\[-1.4cm]
\begin{enumerate}
\item[c)] We want
\begin{align*}
1.96 \, \sqrt{\frac{s_1^2}{n_1}+\frac{s_2^2}{n_2}} = 0.4.
\end{align*}
We assume that $n_1 = n_2 = n$ for simplicity.
\begin{align*}
\Rightarrow 1.96 \, \sqrt{\frac{(2.5)^2}{n}+\frac{(1.1)2}{n}} &= 0.4.\\[0.2cm]
1.96 \, \sqrt{\frac{6.25}{n}+\frac{1.21}{n}} &= 0.4\\[0.2cm]
1.96 \, \sqrt{\frac{6.25+1.21}{n}} &= 0.4\\[0.2cm]
1.96 \, \sqrt{\frac{7.46}{n}} &= 0.4\\[0.2cm]
\frac{1.96\sqrt{7.46}}{\sqrt{n}} &= 0.4
\end{align*}
\begin{align*}
\Rightarrow \frac{1}{\sqrt{n}} &= \frac{0.4}{1.96\sqrt{7.46}}\\[0.2cm]
\sqrt{n} &= \frac{1.96\sqrt{7.46}}{0.4}\\[0.2cm]
n &= \left(\frac{1.96\sqrt{7.46}}{0.4}\right)^2 \\[0.2cm]
&= 179.11.
\end{align*}
Thus, we need about 180 CPUs of each type (i.e., 360 altogether) to reduce the margin of error to within $\pm 0.4$.
\end{enumerate}
\end{minipage}
\end{minipage}}\vspace{0.03\textwidth}


\framebox[1.02\textwidth]{
\begin{minipage}[t]{0.98\textwidth}
\begin{minipage}[t]{0.47\textwidth}
\subsection*{Question 6}
\begin{enumerate}[a)]
\item Here $n = 8$, $\bar x = 6.4$ and $s  =2.2$.\\[0.1cm]
95\% confidence $\Rightarrow$ $\alpha=0.05$ remaining $\Rightarrow$ $\alpha/2=0.025$ in each tail.\\[0.3cm]
Since the sample size is small ($n < 30$), we use a $t$ value rather than a $z$ value. Specifically, the degrees of freedom are $\nu  = n - 1 = 8 -1 =7$.
\begin{align*}
\bar x &\pm t_{\,7,\,0.025} \, \frac{s}{\sqrt{n}} \\[0.2cm]
6.4 &\pm 2.365 \, \frac{2.2}{\sqrt{8}} \\[0.2cm]
6.4 &\pm 1.8395 \\[0.2cm]
[4.56,&\,\,8.24]
\end{align*}
\end{enumerate}
\end{minipage}\hspace{0.04\textwidth}
\begin{minipage}[t]{0.47\textwidth}
\quad\\[-1.4cm]
\begin{enumerate}
\item[] We are 95\% confident that computer science students spend, on average, between 2.56 and 8.24 hours gaming in a week.
\item[b)] 99\% confidence $\Rightarrow$ $\alpha=0.01$ remaining $\Rightarrow$ $\alpha/2=0.005$ in each tail.
    \begin{align*}
\bar x &\pm t_{\,7,\,0.005} \, \frac{s}{\sqrt{n}} \\[0.2cm]
6.4 &\pm 3.499 \, \frac{2.2}{\sqrt{8}} \\[0.2cm]
6.4 &\pm 2.7216 \\[0.2cm]
[3.68,&\,\,9.12]
\end{align*}
Note: to increase our confidence in capturing the true $\mu$, the interval width increases.
\end{enumerate}
\end{minipage}
\end{minipage}}\vspace{0.03\textwidth}





\framebox[1.02\textwidth]{
\begin{minipage}[t]{0.98\textwidth}
\subsection*{Question 7}
\begin{enumerate}[a)]
\item \quad \\[-1.45cm]
\begin{center}
\begin{tabular}{|c|ccccc|c|}
\multicolumn{2}{c}{}&&&& \multicolumn{1}{c}{} & \multicolumn{1}{c}{$\sum$} \\[0.1cm]
\hline
&&&&&&\\[-0.4cm]
$x$ & 34.1 & 33.5 & 32.8 & 33.1  & 32.5 & 166 \\[0.2cm]
$x^2$ & 1162.81 & 1122.25 & 1075.84 & 1095.61 & 1056.25 & 5512.76 \\[0.1cm]
\hline
\multicolumn{7}{c}{}\\[-0.8cm]
\end{tabular}
\end{center}
\begin{align*}
\bar x &= \frac{\sum x}{n} = \frac{166}{5} = 33.2. \\[0.6cm]
s^2 &= \frac{\sum x^2 - n \, \bar x^2 }{n-1} = \frac{5512.76 - 5(33.2^2)}{4} =  0.39. \\[0.6cm]
s &= \sqrt{0.39} = 0.6245.
\end{align*}
\item 95\% confidence $\Rightarrow$ $\alpha=0.05$ remaining $\Rightarrow$ $\alpha/2=0.025$ in each tail.\\[0.3cm]
Since the sample size is small ($n < 30$), we use a $t$ value with $\nu  = n - 1 = 5 -1 =4$.
\begin{align*}
\bar x &\pm t_{\,4,\,0.025} \, \frac{s}{\sqrt{n}} \\[0.2cm]
33.2 &\pm 2.776 \, \frac{0.6245}{\sqrt{5}} \\[0.2cm]
33.2 &\pm 2.776 \, (0.2793) \\[0.2cm]
33.2 &\pm 0.7753 \\[0.2cm]
[32.42,&\,\,33.98]
\end{align*}
\item We are 95\% confident that the true value, $\mu$, lies in the above interval. Note that this interval includes the value $\mu = 33$. Thus, it seems that the machine is working as expected.
\end{enumerate}
\end{minipage}}\vspace{0.03\textwidth}








\end{document} 