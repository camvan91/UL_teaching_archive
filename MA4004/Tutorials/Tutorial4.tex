\documentclass[12pt]{article}
\usepackage{amsmath}
%\usepackage[paperwidth=21cm, paperheight=29.8cm]{geometry}
\usepackage[angle=0,scale=1,color=black,hshift=-0.4cm,vshift=15cm]{background}
\usepackage{multirow}
\usepackage{enumerate}
\usepackage[gen]{eurosym}

%\SetBgScale{1}
%\SetBgAngle{0}
%\SetBgColor{black}
%\SetBgContents{\rule{1pt}{30cm}}
%\SetBgHshift{-8.4cm}
%
%\backgroundsetup{contents={
%\begin{tabular}{c|c}
%\hspace{2cm} & \\[0.7cm]
%& {\bf Statistics for Computing ------ Lecture 1 ------ Solutions} \\[0.3cm]
%%\hline
%\hspace{2cm} & \hspace{18.5cm} \\ [28cm]
%\end{tabular}}}

\backgroundsetup{contents={
{\bf \centering Statistics for Computing ------------------ Tutorial 4 --------------------------- Questions} }}


\setlength{\voffset}{-3cm}
\setlength{\hoffset}{-2cm}
\setlength{\parindent}{0cm}
\setlength{\textheight}{27cm}
\setlength{\textwidth}{17cm}

\pagestyle{empty}



\begin{document}

\subsection*{Question 1}
You develop a random number generater which assigns a value to the random variable $X$ according to the following probability distribution:
\begin{center}
\begin{tabular}{|c|ccccc|}
\hline
&&&&&\\[-0.4cm]
$x$ & 0.0 & 0.5 & 1.0 & 2.0 & 3.0 \\
\hline
&&&&&\\[-0.4cm]
$\Pr(X=x)$ & $0.4$ & $0.2$ & $0.15$ & $0.15$ & $?$ \\[0.1cm]
\hline
\multicolumn{6}{c}{}\\
\end{tabular}
\end{center}

{\bf(a)} What is value the value of $\Pr(X = 3.0)$? \quad {\bf(b)} Calculate $E(X)$ and $Sd(X)$. \quad {\bf(c)} You produce a gambling game where the player wins (in euro) the value of $X$ generated, e.g., if a $2.0$ appears, \euro{2} is won. How much should you charge for a play of this game so that that \emph{you} (the programmer) make a profit of \euro{0.10} on average per game? (i.e., the player \emph{loses} \euro{0.10} on average) \quad {\bf(d)} Using your answer to part (c), determine the probability that you make a profit when somebody plays this game. \quad {\bf(e)} If 10 people play this game, what is the probability that you make a profit 8 times?

\subsection*{Question 2}
You flip three coins. Let $X = $ ``the number of heads'' and $Y = $ ``the number of unique faces''.\\[-0.2cm]

{\bf(a)} What is the sample space for this experiment? \quad {\bf(b)} Construct the \emph{joint distribution} for $X$ and $Y$. \quad {\bf(c)} Based on this joint distribution, construct the \emph{marginal} distribution for $X$ and for $Y$. \quad {\bf(d)} Are $X$ and $Y$ independent? \quad {\bf(e)} Calculate $E(Y)$ and $Sd(Y)$. \quad {\bf(f)} Calculate $\Pr(Y=2\,|\,X=2)$ and interpret its value (compare with $\Pr(Y=2)$).


\subsection*{Question 3}
Let $X =$ ``the attack power of player 1'' and let $Y =$ ``the attack power of player 2''.\\[-0.3cm]

Let the probability distributions for $X$ and $Y$ be:
\begin{center}
\begin{tabular}{|c|ccc|c|c|ccc|}
\cline{1-4}\cline{6-9}
&&&&&&&&\\[-0.4cm]
$x$ & 0 & 100 & 300 & \qquad\qquad & $y$ & 0 & 80 & 200\\
\cline{1-4}\cline{6-9}
&&&&&&&&\\[-0.4cm]
$\Pr(X=x)$ & $0.2$ & $0.75$ & $0.05$ & & $\Pr(Y=y)$ & $0.1$ & $0.6$ & $0.3$ \\[0.1cm]
\cline{1-4}\cline{6-9}
%\multicolumn{9}{c}{}
\end{tabular}
\end{center}
{\footnotesize(e.g., p1 misses 20\% of the time, deals 100 points of damage 75\% of the time and performs a critical blow 5\% of the time.)}\\[-0.2cm]

{\bf(a)} What is the average attack power of each player? \quad {\bf(b)} If both players have 1000 hit-points, how many attacks does it take for player 1 to defeat player 2 and vice versa? Which player will win on average? \quad {\bf(c)} Let's now assume that player 1 uses his/her \emph{first} turn to cast a spell (and therefore does not attack in this turn). The result of the spell is that player 2 can no longer perform a critical blow, i.e., $\Pr(Y=200) = 0$, \emph{from turn two onwards}. Since setting $p(200) = 0$ leads to $\sum p(y) \ne 1$, assume that the remaining probability ($= 0.3$) is distributed evenly between $p(0)$ and $p(80)$. What is the outcome of the battle now?


\subsection*{Question 4}

You flip a coin 10 times - let $X =$ ``the number of heads''. Using the binomial probability function, calculate the following:\\[-0.2cm]

{\bf(a)} $\Pr(X = 2)$. \quad {\bf(b)} $\Pr(X = 0)$. \quad {\bf(c)}  $\Pr(X > 2)$. \quad {\bf(d)} $\Pr(X \le 3)$. \quad {\bf(e)} $\Pr(5 \le X \le 7)$.  \quad {\bf(f)} $E(X)$ and $Sd(X)$. \quad {\bf(g)} Using the binomial tables, calculate $\Pr(X \le10)$ in the case where the coin is flipped 20 times. \quad {\bf(h)} If the coin is flipped 50 times, what is $E(X)$?

\subsection*{Question 5}

Repeat Question 4 (a) - (e) but now using the binomial tables.



\subsection*{Question 6}

Let's assume that a sequence of bits (binary numbers) is transmitted and, at the other end, decoded; the decoder has a 10\% chance reading a bit incorrectly (i.e., reading a 0 as 1 or vice versa). Let $X$ be the number of errors in the sequence received (i.e., the decoded sequence). Calculate the probability that there are: \\[-0.2cm]

{\bf(a)} \emph{No} errors in a 20-bit string. \quad {\bf(b)} Less than three errors in a 10-bit string. \quad {\bf(c)} More than 10 errors in (i) a 50-bit string and (ii) a 100-bit string (hint: use tables). \quad {\bf(d)} Calculate the average number of errors in a 100-bit string. Calculate the standard deviation also.


\subsection*{Question 7}
We follow on from Question 6 but now consider the case where, to reduce the probability of error, each bit is sent \emph{three} times and then a ``majority vote'' approach is used to determine the value of each received bit. The following example explains the situation:\\[-0.5cm]
\begin{center}
\begin{tabular}{ccccc}
\hline
&&&&\\[-0.3cm]
\multirow{2}{*}{Sent} & $0$ & $1$ & $1$ & $0$ \\
& $\overbrace{000}$ & $\overbrace{111}$ & $\overbrace{111}$ & $\overbrace{000}$ \\[0.2cm]
\hline
&&&&\\[-0.3cm]
\multirow{2}{*}{Received} & $\underbrace{001}$ & $\underbrace{111}$ & $\underbrace{010}$ & $\underbrace{000}$ \\
 & $0$ & $1$ & $0$ & $0$ \\[0.2cm]
\hline
%\multicolumn{5}{c}{}
\end{tabular}
\end{center}
$\Rightarrow$ there is one error in decoding the first $000$, but since the majority result is taken, this bit is correctly identified as a $0$. There are two errors in decoding the second $111$, so this bit is misread as a $0$. It is clear that a character is misread if the decoder makes \emph{two or three errors} in these blocks of three replicates.\\[-0.2cm]

{\bf(a)} Show that sending each bit 3 times reduces the error probability from 10\% to 2.8\%. \quad\\ {\bf(b)} Using this reduced value, $p=0.028$, calculate the probability that there are no errors in a 20-bit string. Compare this result to Q6(a). \quad {\bf(c)} Now assume that each bit is sent 5 times and, again, the majority vote approach is used. Calculate the probability that there are no errors in a 20-bit string in this case. %\quad {\bf(d)} Recalculate the two probabilities from part (c) using the Poisson approximation.



\end{document} 