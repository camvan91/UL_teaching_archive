\subsection{Example}
Accident Prone Ltd has recorded the following data. It shows a record of the ages of the people involved in accidents over a three month period.

You are required to construct a box-plot for this data and to summarise your findings in a brief report.


\noindent \textbf{Remarks }


\begin{itemize}
\item The sample size (n) is 40
\item The data is already in ascending order. If it was not in order, you would have to re-arrange it.

\item The minimum and maximum are 16 and 64 respectively. The Range is therefore 48.

\item The median is the average of the 20th (which is 31) and 21st value (which is 32): i.e. 31.5

\item The lower quartile (Q1) is the average of the 10th and 11th values 21.5

\item The upper quartile (Q3) is the average of the 30th and 31st values 37.5

\item The IQR is therefore 16

\item The lower fence is computed as $Q_1 - 1.5IQR$
It makes no sense to use -2.5, so we will use 0 as the value for our lower fence

\item The upper fence is computed as $Q_3 + 1.5IQR$

\item It makes no sense to use -2.5, so we will use 0 as the value for our lower fence

\item We will have one value  64 higher than the upper fence. This is an outlier.
\item Our box-plot looks like this

\[ IMAGE \]

\item We notice that it is skewed. 

\end{itemize}

