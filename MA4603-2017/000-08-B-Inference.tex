\documentclass[]{report}

\voffset=-1.5cm
\oddsidemargin=0.0cm
\textwidth = 480pt

\usepackage{framed}
\usepackage{subfiles}
\usepackage{graphics}
\usepackage{newlfont}
\usepackage{eurosym}
\usepackage{amsmath,amsthm,amsfonts}
\usepackage{amsmath}
\usepackage{color}
\usepackage{amssymb}
\usepackage{multicol}
\usepackage[dvipsnames]{xcolor}
\usepackage{graphicx}
\begin{document}
%1. MA4104
%2. MA4704
%3. StatsLab
%4. Check that variance and standard deviations are specified correctly.











\section{Confidence Intervals for Means}

The structure of a confidence interval for the mean is as follows:
\[ \mbox{Sample Mean} \pm  \left( \mbox{Quantile} \times \mbox{Std. Error}  \right) \]


\noindent \textbf{Quantiles:}\\
For large samples (i.e. greater than 30) where a normal distribution can be assumed, the quantiles 
are as follows\\ \bigskip
\begin{center}
\begin{tabular}{|c|c|}
\hline  $90\%$  &   1.645  \\ 
\hline  $95\%$  &   1.96  \\
\hline  $99\%$  &   2.576 \\ 
\hline 
\end{tabular} 
\end{center}


\noindent \textbf{Sample Mean:}\\
The sample mean $\bar{x}$ is usually given in the question.
\\ \bigskip
(Remark: Sample mean is a type of \textbf{\textit{point estimate}}).


\noindent \textbf{Standard Error :}\\
The standard error is computed using the sample standard deviation ($s$) and the sample size ($n$).

\[ S.E. (\bar{x}) = \frac{s}{\sqrt{n}} \]










%-----------------------------------------------------------%



\textbf{Confidence Intervals}

\begin{itemize}
\item The $95\%$ confidence interval is a range of values which contain the true population parameter (i.e. mean, proportion etc) with a probability of $95\%$.
\item We can expect that a $95\%$ confidence interval will not include the true parameter values $5\%$ of the time.
\item A confidence level of $95\%$ is commonly used for computing confidence interval, but we could also have confidence levels of $90\%$, $99\%$ and $99.9\%$.
\end{itemize}



\begin{itemize}
\item A confidence level for an interval is denoted to $1-\alpha$ (in percentages: $100(1-\alpha)\%$) for some value $\alpha$.
\item A confidence level of $95\%$ corresponds to $\alpha = 0.05$.
\item $100(1-\alpha)\%$ = $100(1-0.05)\%$  = $100(0.95)\%$ = $95\%$
\item For a confidence level of $99\%$, $\alpha = 0.01$.
\item Knowing the correct value for $\alpha$ is important when determining quantiles.
\end{itemize}







%------------------------------------------------------------------------%


% -- Lecture 8B
% -- Revise the Tables
% -- Sample Size Estimation for mean
% -- Example SSE for mean
% -- SSE for Proportion
% -- Example SSE for proportion
% -- Paired Test



%------------------------------------------------------------------------------%



\[ ( \bar{X} - \bar{Y} ) \pm \left[ \mbox{Quantile } \times S.E(\bar{X}-\bar{Y}) \right] \]
\begin{itemize}
	\item If the combined sample size of X and Y is greater than 30, even if the individual sample sizes are less than 30, then we consider it to be a large sample.
	\item The quantile is calculated according to the procedure we met in the previous class.
\end{itemize}

%---------------------------------------------------------%
\begin{itemize}
	\item Assume that the mean ($\mu$) and the variance ($\sigma$) of the distribution 
	of people taking the drug are 50 and 25 respectively and that the mean ($\mu$) 
	and the variance ($\sigma$) of the distribution of people not taking the drug are 
	40 and 24 respectively. 
\end{itemize}










%-----------------------------------------------------------%


\textbf{Standard Error}

\begin{itemize}
\item The standard error measures the dispersion of the sampling distribution.
\item For each type of point estimate, there is a corresponding standard error.
\item A full list of standard error formulae will be attached in your examination paper.
\item The standard error for a  mean is
\[ S.E( \bar{x} )  = {\sigma \over \sqrt{n}} \]
However, we often do not know the value for $\sigma$. For practical purposes, we use the sample standard deviation $s$ as an estimate for $\sigma$ instead.
\[ S.E( \bar{x} )  = {s \over \sqrt{n}} \]
\end{itemize}







\section{Confidence Interval of a Mean of the Small Sample}

If the data have a normal probability distribution and the sample standard deviation $s$ is used to estimate the population standard deviation $\sigma$, the interval estimate is given by:
\begin{equation}
\bar{X} \pm t_{1-\alpha/2,n-1}\frac{s}{\sqrt{n}}
\end{equation}
where $t_{1-\alpha/2,n-1}$ is the value providing an area of $\alpha/2$ in the upper tail of a Student’s t distribution with n - 1 degrees of freedom.


%======================%
\newpage
\section{Introduction to Hypothesis Testing}
\begin{itemize}
\item So far, we have studied two types of confidence interval, a confidence interval for a sample mean and for a sample proportion.\\ (Later we will call these \textbf{\textit{One Sample}} confidence Intervals.
\item There are more types of confidence intervals that we will cover later in this course. \\ (We shall refer to these confidence intervals as the \textbf{\textit{Two Sample}} confidence Intervals.
\item We shall turn our attention to \textit{\textbf{Hypothesis Testing}} in the mean time.
\end{itemize}


\textbf{Determining the Quantile}
The confidence level is $95\%$. The sample size is greater than 30. Therefore the appropriate quantile is 1.96.\\





\end{document}
