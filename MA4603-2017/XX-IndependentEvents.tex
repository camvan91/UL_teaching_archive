


\section{Independent Events}

\begin{itemize}
\item Two events are independent if the occurrence of one does not change the probability of the other occurring.
\item An example would be rolling a 2 on a die and flipping a head on a coin. Rolling the 2 does not affect the probability of flipping the head.
\item If events are independent, then the probability of them both occurring is the product of the probabilities of each occurring.
\end{itemize}
\[P(A \cap B) = P(A) \times P(B)\]

%-------------------------------------------------------%   


}




\section{Independent Events}
Events A and B in a probability space $S$ are said to be independent if the
occurrence of one of them does not influence the occurrence of the other.\\ \bigskip

More specifically, $B$ is independent of A if $P(B)$ is the same as $P(B|A)$. Now
substituting $P(B)$ for $P(B|A)$ in the multiplication theorem from the previous
slide yields.
\[ P(A\cap B) = P(A)\times P(B)\]
We formally use the above equation as our definition of independence.
