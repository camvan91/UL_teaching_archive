 \documentclass[a4paper,12pt]{article}
%%%%%%%%%%%%%%%%%%%%%%%%%%%%%%%%%%%%%%%%%%%%%%%%%%%%%%%%%%%%%%%%%%%%%%%%%%%%%%%%%%%%%%%%%%%%%%%%%%%%%%%%%%%%%%%%%%%%%%%%%%%%%%%%%%%%%%%%%%%%%%%%%%%%%%%%%%%%%%%%%%%%%%%%%%%%%%%%%%%%%%%%%%%%%%%%%%%%%%%%%%%%%%%%%%%%%%%%%%%%%%%%%%%%%%%%%%%%%%%%%%%%%%%%%%%%
\usepackage{eurosym}
\usepackage{vmargin}
\usepackage{amsmath}
\usepackage{multicol}
\usepackage{graphics}
\usepackage{epsfig}
\usepackage{framed}
\usepackage{subfigure}
\usepackage{fancyhdr}

\setcounter{MaxMatrixCols}{10}
%TCIDATA{OutputFilter=LATEX.DLL}
%TCIDATA{Version=5.00.0.2570}
%TCIDATA{<META NAME="SaveForMode" CONTENT="1">}
%TCIDATA{LastRevised=Wednesday, February 23, 2011 13:24:34}
%TCIDATA{<META NAME="GraphicsSave" CONTENT="32">}
%TCIDATA{Language=American English}

%\pagestyle{fancy}
\setmarginsrb{20mm}{0mm}{20mm}{25mm}{12mm}{11mm}{0mm}{11mm}
%\lhead{MA4413 2013} \rhead{Mr. Kevin O'Brien}
%\chead{Midterm Assessment 1 }
%\input{tcilatex}

\begin{document}



%---------------------------------------------------------------- %
\newpage
\noindent {\Large \textbf{MA4413 Weeks 10 and 11 Tutorials}}
\section*{Question 1 (Paired t-test)}
The weight of 10 students was observed before commencement of their studies and after graduation (in kgs). By calculating the realisation of the appropriate test statistic, test the hypothesis that the mean weight of students increases during their studies at a significance level of  5\%. 
%--------------------------------%
\begin{center}
\begin{tabular}{|c|c|c|c|c|c|c|c|c|c|c|}
\hline
Student	&	1	&	2	&	3	&	4	&	5	&	6	&	7	&	8	&	9	&	10	\\ \hline
Weight before	&	68	&	74	&	59	&	65	&	82	&	67	&	57	&	90	&	74	&	77	\\ \hline
Weight after	&	71	&	73	&	61	&	67	&	85	&	66	&	61	&	89	&	77	&	83	\\ \hline
\end{tabular} 
\end{center}

%-------------------------------------------%
\noindent \textbf{[Recall Descriptive Statistics]}\\
\noindent You may be required to carry out these calculations in the exam.
\begin{itemize}
\item Case-wise differences are 
\[ d = \{3, -1,  2,  2,  3, -1,  4, -1,  3,  6  \}\]
\item The sum of case-wises differences and squared case-wise differences are $\sum d_i = 20$ and $\sum d_i^2 = 90$ respectively.
\item Mean of case-wise differences $\bar{d}=2.00$.
\[ \bar{d} = \frac{3 + (-1) + 2 + \ldots + 6}{10} = \frac{20}{10} \]
\item Standard deviation of casewise differences $s_d= 2.36 $\\
(Modified version of standard deviation formula)


\[ s_d = \sqrt{\frac{ \sum(d^2_i) - \frac{(\sum d_i)^2}{n}}{n-1}}\]
\[ s_d = \sqrt{\frac{ 90 - \frac{(20)^2}{10}}{9}} = \sqrt{\frac{50}{9}} = 2.36 \]

\item Standard Error
\[ S.E.(d) = \frac{s_d}{\sqrt{n}} =\frac{2.36}{\sqrt{10}} = 0.745\]
\item From Murdoch Barnes, the CV is 1.812 (small sample,df = 9,one-tailed procedure)
\end{itemize}


\noindent \textbf{Writing the Hypotheses}
\begin{description}
\item[$H_0$] $\mu_d \leq 0$ \\mean of case-wise differences not a positive number. (i.e. no increase in weight)
\item[$H_1$] $\mu_d > 0$ \\mean of case-wise differences is a positive number. (i.e. increase in weight)
\end{description}
%-------------------------------------------%


%--------------------------------%
\subsection*{Question 1 Part B}
Calculate a 95\% confidence interval for the amount of weight that students put on during their studies. Using this confidence interval, test the hypotheses that on average students put on \textbf{3 kilos} during their studies
%ii) students lose 3 kilos during their studies.

%(What assumption was made in order to both carry out the test and calculate the confidence interval?)

\section*{Question 2 (Two Sample Means - One Tailed)}
%good
A pharmaceutical company wants to test, a new medication for blood pressure. Tests
for such products often include a `\textit{treatment group}' of people who use the drug and a `\textit{control group}'of people who did not use the drug. 50 people with high blood pressure are given the new drug and 100 others, also with high blood pressure, are not given the drug. 

The systolic blood pressure is measured for each subject, and the sample statistics
are given below. Using a 0.05 level of significance, test the claim that the new drug \textbf{reduces}
blood pressure. 
%Would you recommend advertising that the new drug does not aect blood pressure?
\begin{center}
\begin{tabular}{|c||c|}
\hline 
Treatment & Control \\ \hline \hline
$n_1$ = 50 & $n_2$ = 100 \\ \hline
$x_1$ = 189.4  & $x_1$ = 203.4  \\ \hline
$s_1$ = 39.0 & $s_1$ = 39.4 \\ \hline
\end{tabular} 
\end{center}
\textbf{\textit{Standard Error Formula}}
\[ S.E.(\bar{x}_1 - \bar{x}_2)  = \sqrt{\frac{s_1^2}{n_1} + \frac{s_2^2}{n_2}} \]

%----------------------------------------------------------------------- %
\section*{Question 5 (Two Sample proportions, one tailed)}

A simple random sample of front-seat occupants involved in car crashes were taken. 
The first sample was on cars with airbags available and it was found that there were 29 occupant fatalities out of a total of 1110 occupants. The second sample was on cars with no airbags available and
there were 62 fatalities out of a total 1553 occupants.
\begin{itemize}
\item[(i)] Using a 5\% significance level, determine whether or not there is a difference in the proportion of fatality rates of occupants in cars with airbags and cars without airbags.
\item[(ii)] Calculate a 95\% confidence interval for the difference between the two proportions of fatality rates.
\end{itemize}

\noindent \textbf{\textit{Standard Error Formula }}\\
Confidence Intervals
\[ S.E.(\hat{p}_1 - \hat{p}_2)  = \sqrt{\frac{\hat{p}_1 \times (100 - \hat{p}_1)}{n_1} + \frac{\hat{p}_2 \times (100 - \hat{p}_2)}{n_2}} \]
Hypothesis testing
\[ S.E.(\pi_1 - \pi_2)  = \sqrt{\bar{p} \times (100 - \bar{p}) \times \left( \frac{1}{n_1} + \frac{1}{n_2}\right)} \]
Aggregate Sample Proportion
\[  \bar{p} = \frac{x_1+x_2}{n_1+n_2} \]


\noindent \textbf{\textit{Confidence Intervals (in terms of percentages) }}\\
95\% confidence interval
\[ (\hat{p}_1 - \hat{p}_2 ) \times (1.96 \times S.E.(\hat{p}_1 - \hat{p}_2))\]
\[ 1.4 \times (1.96 \times 0.683) =  (1.27,1.53)\]

\newpage
\section*{Question 10 - Testing Equality of Variances}
Interpret the output from the following tests of equality of variances. State your conclusion both by referencing the $p-$value and the confidence interval. You may assume the significance level is 5\%.

(Remark : This procedure is a one-tailed procedure. However, we will base our conclusion on whether or not we arbitrarily decide the p-value is large or small )
\begin{framed}
\begin{verbatim}
> var.test(X,Y)

        F test to compare two variances

data:  X and Y
F = ………, num df = 13, denom df = 13, p-value = 0.02725
alternative hypothesis: 
   true ratio of variances is not equal to 1 
95 percent confidence interval:
  1.164437 11.299050 
sample estimates:
ratio of variances 
          ………… 
\end{verbatim}
\end{framed}
\begin{framed}
\begin{verbatim}
> var.test(X,Z)

        F test to compare two variances

data:  X and Z 
F = ………, num df = 13, denom df = 11, p-value = 0.7813
alternative hypothesis: 
   true ratio of variances is not equal to 1 
95 percent confidence interval:
 0.2526643 2.7401535 
sample estimates:
ratio of variances 
         ………………
\end{verbatim}
\end{framed}
%\begin{framed}
%\begin{verbatim}
%> var.test(Y,Z)
%
%        F test to compare two variances
%
%data:  Y and Z
%F = ………, num df = 13, denom df = 11, p-value = 0.01616
%alternative hypothesis: true ratio of variances is not equal to 1 
%95 percent confidence interval:
% 0.06965702 0.75543304 
%sample estimates:
%ratio of variances 
%         …………………
%\end{verbatim}
%\end{framed}
%\end{document}
\newpage

\section*{Question 11 - Shapiro-Wilk Test}
Interpret the output from the three Shapiro-Wilk tests. What is the null and alternative hypotheses? State your conclusion for each of the three tests.
\begin{framed}
\begin{verbatim}
> shapiro.test(X)

        Shapiro-Wilk normality test

data:  X
W = 0.9001, p-value = 0.113
>
\end{verbatim}
\end{framed}
\begin{framed}
\begin{verbatim}
> shapiro.test(Y)

        Shapiro-Wilk normality test

data:  Y 
W = 0.8073, p-value = 0.006145
>
\end{verbatim}
\end{framed}
\begin{framed}
\begin{verbatim}
> shapiro.test(Z)

        Shapiro-Wilk normality test

data:  Z
W = 0.9292, p-value = 0.372
\end{verbatim}
\end{framed}
%------------------------------------------------ %
\newpage
\section*{Question 12 - Classification Metrics}
For each of the following classification tables, calculate the following appraisal metrics.
\begin{multicols}{2}
\begin{itemize}	
\item 	accuracy  
\item 	recall
\item 	precision
\item 	F-measure
\end{itemize}
\end{multicols}
	



\begin{center}
\begin{tabular}{|c|c|c|}
\hline  & Predict Negative & Predict Positive \\ 
\hline Observed Negative &	9500	&	85	\\ 
\hline Observed Positive & 	115	&	300	\\ 
\hline 
\end{tabular} 
\end{center}

\begin{center}
\begin{tabular}{|c|c|c|}
\hline  & Predict Negative & Predict Positive \\ 
\hline Observed Negative &	9700	&	140	\\ 
\hline Observed Positive & 	60	&	100	\\ 
\hline 
\end{tabular} 
\end{center}

\begin{center}
\begin{tabular}{|c|c|c|}
\hline  & Predict Negative & Predict Positive \\ 
\hline Observed Negative &	9530	&	10	\\ 
\hline Observed Positive & 	300	&	160	\\ 
\hline 
\end{tabular} 
\end{center}
			


\end{document}




\end{document}
