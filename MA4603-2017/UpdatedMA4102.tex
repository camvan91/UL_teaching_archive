
\documentclass[a4paper,12pt]{article}
%%%%%%%%%%%%%%%%%%%%%%%%%%%%%%%%%%%%%%%%%%%%%%%%%%%%%%%%%%%%%%%%%%%%%%%%%%%%%%%%%%%%%%%%%%%%%%%%%%%%%%%%%%%%%%%%%%%%%%%%%%%%%%%%%%%%%%%%%%%%%%%%%%%%%%%%%%%%%%%%%%%%%%%%%%%%%%%%%%%%%%%%%%%%%%%%%%%%%%%%%%%%%%%%%%%%%%%%%%%%%%%%%%%%%%%%%%%%%%%%%%%%%%%%%%%%
\usepackage{eurosym}
\usepackage{vmargin}
\usepackage{amsmath}
\usepackage{graphics}
\usepackage{epsfig}
\usepackage{subfigure}
\usepackage{fancyhdr}

\setcounter{MaxMatrixCols}{10}
%TCIDATA{OutputFilter=LATEX.DLL}
%TCIDATA{Version=5.00.0.2570}
%TCIDATA{<META NAME="SaveForMode" CONTENT="1">}
%TCIDATA{LastRevised=Wednesday, February 23, 2011 13:24:34}
%TCIDATA{<META NAME="GraphicsSave" CONTENT="32">}
%TCIDATA{Language=American English}

\pagestyle{fancy}
\setmarginsrb{20mm}{0mm}{20mm}{25mm}{12mm}{11mm}{0mm}{11mm}
\lhead{MA4102} \rhead{Dr. Se$\mathrm{\acute{a}}$n Lacey}
\chead{Repeat Assessment Paper 2011}
%\input{tcilatex}

\begin{document}


\section{Mathematics: Attempt ALL questions}

\begin{enumerate}
\item
\begin{enumerate}
\item (5 marks) How many months would it take an initial investment of \euro
3,500 to accumulate \euro 5,700, if the interest rate is 5\% compounded
every six months?



\item (5 $\times$ 3 marks) Consider the matrices%
\begin{equation*}
A=\left[
\begin{array}{cc}
-1 & 0 \\
2 & -4 \\
\end{array}%
\right] ,\quad B=\left[
\begin{array}{ccc}
2 & -3 \\
4 & 5
\end{array}%
\right] ,\quad C=\left[
\begin{array}{ccc}
1 & 4 & -2 \\
3 & 1 & 4%
\end{array}%
\right] .
\end{equation*}%
Compute (where possible):%
\begin{equation*}
\text{i.}\quad 3A+2B,\qquad \text{ii.}\quad B^{-1},\qquad \text{iii.}\quad \left(
AB\right) ^{T}, \qquad \text{iv.}\quad BC,\qquad \text{v.}\quad CB.
\end{equation*}

\item A company can use two types of machine X and Y in a manufacturing
plant. The number of operators for each machine and the running cost per day
are given as%
\begin{equation*}
\begin{array}{ccc}
& \text{Cost per day} & \text{Available operators} \\
\text{Machine X} & 4 & 4 \\
\text{Machine Y} & 5 & 3 \\
\text{Maximum available} & 320 & 240%
\end{array}%
\end{equation*}%
The profit per good is 30 for machine X and 25 for machine Y.

\begin{enumerate}
\item (2 marks) What is the objective function for this problem?

\item (6 marks) What are the constraints for this problem?

\item (6 marks) Sketch the feasible region.

\item (6 marks) Determine the number of goods  that should be
produced by X and Y respectively to maximise profit. What is the maximum profit?
\end{enumerate}
\end{enumerate}

\item

\begin{enumerate}
\item (5 marks) A principal of \euro 5,000 is invested with interest
compounded continuously. What is the required interest rate if the
investment is to grow to \euro 7,500 after 4 years?


%%%%%%%%%%%%%%%%%%%%%%%%%%%%%%%%%%%%%%%%%%%%%%%%%%%%%%%%%%%%%%%%%%%%%%%%%%%%%%%%%%


\item The supply and demand functions for a particular market are given by the equations:

\begin{equation*}
P_{S}=Q^{2}-6Q+9,\qquad P_{D}=Q^{2}-10Q+25.
\end{equation*}


\begin{enumerate}
\item (6 marks) Sketch the graph of each function over the interval Q = 0 to Q = 6.

\item (2 marks) What are the maximum/minimum values for each quadratic?

\item (5 marks) Find algebraically the point of intersection of the two quadratic functions (i.e., find
the equilibrium price and quantity).

\end{enumerate}





%%%%%%%%%%%%%%%%%%%%%%%%%%%%%%%%%%%%%%%%%%%%%%%%%%%%%%%%%%%%%%%%%%%%%%%%%%%%%%%%%%%%
\item The demand and supply functions of a good are given by%
\begin{equation*}
P=-8Q_{D}+50,\qquad P=5Q_{S}+11.
\end{equation*}%
Find the equilibrium price and quantity using

\begin{enumerate}
\item (6 marks) graphical methods,

\item (8 marks) the inverse matrix method, and

\item (8 marks) using Cramer's rule.
\end{enumerate}
\end{enumerate}
\end{enumerate}

\newpage

\section{Statistics: Attempt ALL questions}


\begin{enumerate}
\item
\begin{enumerate}
\item The heights for a group of forty rowing club members are tabulated as follows;

    \begin{table}[ht]
\begin{center}
\begin{tabular}{|rrrrrrrrrr|}

  \hline
141 & 148 & 149 & 149 & 155 & 156 & 167 & 169 & 169 & 170 \\
170 & 173 & 175 & 176 & 177 & 179 & 182 & 182 & 183 & 183 \\
183 & 184 & 184 & 184 & 185 & 185 & 185 & 186 & 186 & 189 \\
191 & 191 & 191 & 191 & 192 & 192 & 192 & 193 & 194 & 199 \\
   \hline
\end{tabular}
\end{center}
\end{table}
\vspace{-0.5cm}
\begin{enumerate}
\item (6 marks) Summarize the data in the above table using a frequency table. Use 6 class intervals, with 140 as the lower limit of the first interval.
\item (6 marks) Draw a histogram for the above data.
\item (4 marks) Comment on the shape of the histogram. Based on the shape of the histogram, what is the best measure of centrality and variability?
\item (12 marks) Construct a box plot for the above data. Clearly demonstrate how all of the necessary values were computed.
\end{enumerate}
\vspace{0.25cm}
\item Data on the construction durations (measured in months) were collected for a random sample of similar infrastructure projects in two neighbouring countries: $A$ and $B$.\\
    The durations for country $A$ were collected and tabulated as follows;

\begin{table}[ht]
\begin{center}
\begin{tabular}{|rrrrrrr|}

\hline
24 & 17 & 12 & 23 & 18 & 20 & 19 \\

%41 & 44 & 44 & 43 & 37 & 37 & 34  \\
\hline
\end{tabular}
\end{center}
\end{table}
\vspace{-0.5cm}
\noindent For this sample, compute the following descriptive statistics:
\begin{itemize}
\item[a.] (1 Mark) The median,
\item[b.] (1 Mark) The mean,
\item[c.] (1 Mark) The variance,
\item[d.] (1 Mark) The standard deviation.
\end{itemize}

%\begin{enumerate}
%\item (2 marks) Calculate the mean of the durations for country $A$.
%\item (4 marks) Calculate the variance for country $A$.
%\item (2 marks) Calculate the standard deviation for country $A$.
%\item (2 marks) Calculate the coefficient of variation for country $A$.
%
%\item (2 marks) For the sample in country $B$, the mean of the durations was found to be 36 weeks, with a standard deviation of 6 weeks. In which country do the durations show a more dispersed distribution?
%
%\end{enumerate}


\end{enumerate}
\newpage
%---------------------------------------%
\item
\begin{enumerate}

\item An electronics assembly subcontractor receives resistors from two suppliers: Deltatech provides
70\% of the subcontractors's resistors while another company, Echelon, supplies the remainder.
1\% of the resistors provided by Deltatech fail the quality control test, while 2\% of the
chips from Echelon also fail the quality control test.

\begin{enumerate}
\item (5 marks)What is the probability that a resistor will fail the quality control test?


\item (4 marks)What is the probability that a resistor that fails the quality control test was supplied by Echelon?
\end{enumerate}


\vspace{0.25cm}


\item It is estimated by a particular bank that 25\% of credit card customers pay only the minimum amount due on their monthly credit card bill and do not pay the total amount due. 50 credit card customers are randomly selected.
\begin{enumerate}
\item (3 marks)	What is the probability that 9 or more of the selected customers pay only the minimum amount due?
\item (3 marks) What is the probability that less than 6 of the selected customers pay only the minimum amount due?
\item (3 marks)	What is the probability that more than 5 but less than 10 of the selected customers pay only the minimum amount due?
\end{enumerate}



\vspace{0.25cm}
\item The average lifespan of a PC monitor is 6 years. You may assume that the lifespan of monitors follows an exponential probability distribution.
    \begin{enumerate}
    \item (3 marks) What is the probability that the lifespan of the monitor will be at least 5 years?
    \item (3 marks) What is the probability that the lifespan of the monitor will not exceed 4 years?
    \item (3 marks) What is the probability of the lifespan being between 5 years and 7 years?
    \end{enumerate}
\vspace{0.25cm}
\item A machine is used to package bags of potato chips.  Records of the packaging machine indicate that its fill weights are normally distributed with a mean of 455 grams per bag and a standard deviation of 10 grams.

    \begin{enumerate}
    \item (5 marks) What proportion of bags filled by this machine will contain more than 470 grams in the long run?
    \item (5 marks)	What proportion of bags filled by this machine will contain less than 445 grams in the long run?
    \item (3 marks)	What proportion of bags filled by this machine will be between 465 grams and 475 grams in the long run?
    \end{enumerate}
\end{enumerate}



\end{enumerate}
\newpage
\section{Useful formulae}

\subsection{Mathematics}

\begin{enumerate}
\item Logarithms: If $N=b^{n}$, then $\log _{b}N=n.$

\item Compound interest:%
\begin{equation*}
P_{t}=P_{0}\left( 1+i\right) ^{t},\qquad P_{t}=P_{0}\left( 1+\frac{i}{m}%
\right) ^{mt},\qquad P_{t}=P_{0}\mathrm{e}^{it}.
\end{equation*}

\item Matrices:

\begin{enumerate}
\item Inverse of a 2*2 matrix:
\begin{equation*}
\left[
\begin{array}{cc}
a_{11} & a_{12} \\
a_{21} & a_{22}%
\end{array}%
\right] ^{-1}=\frac{1}{a_{11}a_{22}-a_{12}a_{21}}\left[
\begin{array}{cc}
a_{22} & -a_{12} \\
-a_{21} & a_{11}%
\end{array}%
\right] .
\end{equation*}

\item Deteminant of a 2*2 matrix:
\begin{equation*}
\left\vert
\begin{array}{cc}
a_{11} & a_{12} \\
a_{21} & a_{22}%
\end{array}%
\right\vert =a_{11}a_{22}-a_{12}a_{21}.
\end{equation*}

\item Cramer's Rule: If
\begin{eqnarray*}
a_{1}x+b_{1}y &=&d_{1}, \\
a_{2}x+b_{2}y &=&d_{2},
\end{eqnarray*}%
then%
\begin{equation*}
x=\frac{\left\vert
\begin{array}{cc}
d_{1} & b_{1} \\
d_{2} & b_{2}%
\end{array}%
\right\vert }{\left\vert
\begin{array}{cc}
a_{1} & b_{1} \\
a_{2} & b_{2}%
\end{array}%
\right\vert },\qquad y=\frac{\left\vert
\begin{array}{cc}
a_{1} & d_{1} \\
a_{2} & d_{2}%
\end{array}%
\right\vert }{\left\vert
\begin{array}{cc}
a_{1} & b_{1} \\
a_{2} & b_{2}%
\end{array}%
\right\vert }.
\end{equation*}
\end{enumerate}
\end{enumerate}

\subsection{Statistics}

\begin{enumerate}
\item Sample mean
\begin{equation*}
\bar{x}=\frac{\sum x_{i}}{n}.
\end{equation*}

\item Sample standard deviation
\begin{equation*}
s=\sqrt{\frac{\sum \left( x_{i}-\bar{x}\right) ^{2}}{%
n-1}}.
\end{equation*}

\item Conditional probability:
\begin{equation*}
P(B|A)=\frac{P\left( A\text{ and }B\right) }{P\left( A\right) }.
\end{equation*}

\item Binomial probability function
\begin{equation*}
f\left( x\right) =\left(
\begin{array}{c}
n \\
x%
\end{array}%
\right) p^{x}\left( 1-p\right) ^{n-x}\qquad \text{where}\qquad \left(
\begin{array}{c}
n \\
x%
\end{array}%
\right) =\frac{n!}{x!\left( n-x\right) !}.
\end{equation*}

\item Poisson probability function
\begin{equation*}
f\left( x\right) =\frac{m^{x}\mathrm{e}^{-m}}{x!}.
\end{equation*}

\item Exponential probability distribution
\begin{equation*}
P\left( X \leq k \right) = 1 - e^{-k/\mu}
\end{equation*}


\end{enumerate}

\end{document}
