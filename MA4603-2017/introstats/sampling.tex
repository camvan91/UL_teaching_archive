
%=====================================================%
\subsection{1.5 Types of Sampling}
\subsubsection{1. Simple Random Sampling}
A sample is a simple random sample (SRS) if the probability of picking any collection of n units from the sampling frame as a
sample does not depend on the collection of units, i.e. individuals
are picked completely ”at random” from the population. Advantages: 1. Simple, 2. If the sampling frame is appropriate,
then there is no sampling bias.

Disadvantages: If the population can be divided into units called strata, which have different characteristics, more precision can be
obtained by using Stratified Random Sampling (see below).

\subsubsection{2. Stratified Random Sampling}
The population is divided into strata according to various
characteristics, e.g. sex, occupation, age. A simple random sample is taken from each stratum, so that the
proportions of individuals taken from each stratum equals the proportion of individuals in the population as a whole within that
stratum.
%=====================================================%
%\subsection{Stratified Random Sampling}
Advantages:
\begin{enumerate} 
\item If the sampling frame is appropriate and the proportions of individuals in each stratum known, then there is no
sampling bias.
\item If the strata are different in their characteristics, this method
leads to increased precision.
\end{enumerate}

Disadvantages:
\begin{enumerate} 
\item  Complexity, increased time required to collect
samples.
\item The proportions of individuals in each stratum should be known
accurately for the method to be effective.
\end{enumerate}
%=====================================================%
\subsubsection{3. Systematic sampling}
\begin{itemize}
\item Suppose we wish to sample a fraction 1/k of the population. One
individual is chosen from the first k units listed in the sampling
frame and each k-th individual is chosen from then onwards, i.e. if
\item I want to choose 1\% of the population, I may choose number 53
from the first 100, then numbers 153, 253,. . . are also picked.
\end{itemize}
Advantages and Disadvantages: 
\begin{itemize}
\item Same as simple random
strategies, except that it is more difficult to control the precision if
the list is not ordered randomly with respect to the variables of
interest.
e.g. If one wanted to study earnings in a given population, it would
be reasonable to choose subjects from an alphabetic list using
systematic sampling.
\item However, it would not be reasonable to choose the subjects in such
a way from a list of highest earners (i.e. with the highest earners
at the top).
\end{itemize}
%=====================================================%
\subsubsection{4. Convenience Sampling}
The sample is chosen so as to minimise the costs of obtaining a
sample, i.e. questioning friends and colleagues, inviting volunteers
etc.

Advantages: 
\begin{enumerate}
\item Simplicity, 
\item Low sampling costs.
\end{enumerate}
Disadvantages: 1. The sample may not be representative (i.e.
there is sampling bias). For example, when we invite responses on
some political issue (e.g. immigration), we are more likely to get
responses only from those who have rather extreme views on that
issue. 

If we invite respondents to a study on the mathematical
abilities of students, mathematically gifted individuals are more
likely to volunteer than non-gifted individuals.

%=====================================================%
\subsubsection{5. Expert Sampling}
An expert chooses a sample that he feels is representative of the
population as a whole.
Advantages: 1. If the method employed by the expert is
appropriate, this may well result in increased precision.
Disadvantages: 1. An inappropriate method will lead to bias.

\subsubsection{Probability Sampling}
\begin{itemize}
\item These methods can be more broadly defined into probability and
non-probability sampling.
\item Probability sampling occurs when the probability that an individual
is picked to be in a sample does not depend on the individual.
\item 
The first 3 methods described above are forms of probability
sampling. Note: Stratified random sampling is a form of
probability sampling as long as the correct proportions of
individuals in the various strata are known.
\item Such methods are advantageous as much can be said about the
distribution of the statistics obtained from such a sample (see
below).
\end{itemize}
