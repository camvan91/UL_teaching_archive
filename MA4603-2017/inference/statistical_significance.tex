\subsection*{Significance )}

\begin{itemize}
\item In hypothesis testing, the significance level $\alpha$ is the threshold used for rejecting the null hypothesis. \item The significance level is used in hypothesis testing as follows: First, the difference between the result (i.e. observed statistic or point estimate)  of the experiment and the \textbf{null value} \item The null value is the expected value of this statistic, assuming that the null hypothesis is true), is determined.(i.e.we denote this  \textbf{\textit{Observed - Null}}). \item Then, assuming the null hypothesis is true, the probability of a difference that large or larger is computed . \item Finally, this probability is compared to the significance level.\item  If the probability is less than or equal to the significance level, then the null hypothesis is rejected and the outcome is said to be statistically significant.
\end{itemize}

\newpage
\section{Statistical significance}
\begin{itemize}
\item Statistical significance is a mathematical tool used to determine whether the outcome of an experiment is the result of a relationship between specific factors or due to chance. 
\item Statistical significance is commonly used in the medical field to test drugs and vaccines and to determine causal factors of disease. Statistical significance is also used in the fields of psychology, environmental biology, and any other discipline that conducts research through experimentation.

\item Statistics are the mathematical calculations of numeric sets or populations that are manipulated to produce a probability of the occurrence of an event. Statistics use a numeric sample and apply that number to an entire population. 
\item For the sake of example, we might say that $80\%$ of all Americans drive a car. It would be difficult to question every American about whether or not they drive a car, so a random number of people would be questioned and then the data would be statistically analyzed and generalized to account for everyone.

\item In a scientific study, a hypothesis is proposed, then data is collected and analyzed. The statistical analysis of the data will produce a number that is statistically significant if it falls below $5\%$, which is called the confidence level. In other words, if the likelihood of an event is statistically significant, the researcher can be $95\%$ confident that the result did not happen by chance.

\item Sometimes, when the statistical significance of an experiment is very important, such as the safety of a drug meant for humans, the statistical significance must fall below $3\%$. In this case, a researcher could be $97\%$ sure that a particular drug is safe for human use. This number can be lowered or raised to accommodate the importance and desired certainty of the result being correct.
\item Statistical significance is used to reject or accept what is called the null hypothesis. A hypothesis is an explanation that a researcher is trying to prove. The null hypothesis holds that the factors a researcher is looking at have no effect on differences in the data. \item Statistical significance is usually written, for example, $t=.02, p<.05$. Here, "t" stands for the statistic test score and "p<.05" means that the probability of an event occurring by chance is less than $5\%$. These numbers would cause the null hypothesis to be rejected, therefore affirming that the alternative hypothesis is true.

\item Here is an example of a psychological hypothesis using statistical significance: It is hypothesized that baby girls smile more than baby boys. In order to test this hypothesis, a researcher would observe a certain number of baby girls and boys and count how many times they smile. At the end of the observation, the numbers of smiles would be statistically analyzed.
\item Every experiment comes with a certain degree of error. It is possible that on the day of observation all the boys were abnormally grumpy. 

\item The statistical significance found by the analysis of the data would rule out this possibility by 95\% if t=.03. In this case, the null hypothesis that baby girls do not smile more than baby boys would be rejected, and with 95\% certainty, the researcher could say that girls smile more than boys.

\end{itemize}

\subsection{Significance Level}

\begin{itemize}
\item In everyday spoken English, ``\textit{significant}" means important, while in Statistics ``\textit{significant}" means probably true (not due to chance). A research finding may be true without being important.\item When statisticians say a result is "highly significant" they mean it is very probably true. 

\item 95\% is commonly used as the accepted level in statistical college classes. In Medical research, it is likely to be much higher, e.g. 99.9\%.

\item
The significance level of a statistical hypothesis test is a fixed probability of wrongly rejecting the null hypothesis $H_0$, if it is in fact true.

\item It is the probability of a \textit{\textbf{type I error}} and is set by the investigator in relation to the consequences of such an error. That is, we want to make the significance level as small as possible in order to protect the null hypothesis and to prevent, as far as possible, the investigator from inadvertently making false claims.

\item The significance level is usually denoted by $\alpha$ (alpha):

\[\mbox{Significance Level} = \mbox{P(type I error)} = \alpha\]

\item Usually, the significance level is chosen to be 0.05 (or equivalently, 5\%).

\end{itemize}


\end{document}
