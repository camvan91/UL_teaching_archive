If X is a continuous random variable then we can say that the probability of obtaining a \textbf{precise} value $x$ is infinitely small, i.e. close to zero.

\[P(X=x) \approx 0 \]

Consequently, for continuous random variables (only),  $P(X \leq x)$ and $P(X < x)$ can be used interchangeably.

\[P(X \leq x) \approx P(X < x) \]

%==========================================================================%
\subsection{Expected Value for Continuous Random Variables}
The expected value of a random variable X is symbolised by E(X) or $mu$.


If X is a discrete random variable with possible values $x_1, x_2, x_3, \ldots, x_n$, and p(xi) denotes P(X = xi), then the expected value of X is defined by: 

DEFINITION

where the elements are summed over all values of the random variable X. 

%==========================================================================%

\section{Quantiles for Probability Distributions}
\begin{itemize}
\item The quantile (this term was first used by Kendall, 1940) of a distribution of values is a number xp such that a proportion p of the population values are less than or equal to xp. 

\item For example, the .25 quantile (also referred to as the 25th percentile or lower quartile) of a variable is a value (xp) such that $25\%$ (p) of the values of the variable fall below that value.

\item Similarly, the $0.75$ quantile (also referred to as the 75th percentile or upper quartile) is a value such that $75\%$ of the values of the variable fall below that value and is calculated accordingly.
\end{itemize}

%===========================================================================%
