

\documentclass[]{report}

\voffset=-1.5cm
\oddsidemargin=0.0cm
\textwidth = 480pt

\usepackage{framed}
\usepackage{subfiles}
\usepackage{graphics}
\usepackage{newlfont}
\usepackage{eurosym}
\usepackage{amsmath,amsthm,amsfonts}
\usepackage{amsmath}
\usepackage{color}
\usepackage{amssymb}
\usepackage{multicol}
\usepackage[dvipsnames]{xcolor}
\usepackage{graphicx}
\begin{document}

%========================================================%

\subsection{Complement Events}

\textbf{Important!}
\begin{itemize}
	
	\item The \textbf{complement} of any event \textbf{A} is the event ``\textbf{not A}", i.e. the event that A does not happen.
	\item The event A and its complement \textbf{not A} are mutually exclusive and exhaustive. 
	\item Generally, there is only one event B such that A and B are both mutually exclusive and exhaustive; that event is the complement of A. 
	\item The complement of an event A is usually denoted as $A^{c}$ or $A^{\prime}$.
	
\end{itemize}

\[P(A^{c})=1-P(A).\]


\section{The Complement Rule}
\begin{itemize}
	\item 
	The probability of an event not occurring is one minus the probability of it occurring.
	
	\[P(E^{C}) = 1 - P(E)\]
\end{itemize}
\section*{Complement Rule}


%---------------------------------------------- %

\[P(A |B) = \frac{P(A \cap B)}{P(B)})\]

The complement rule in Probability

$P(C^{\prime}) = 1- P(C)$


If the probability of C is $70 \%$ then the probability of $C^{\prime}$ is $30\%$

	%--------------------------------------------------------------------------------%
	{
		\textbf{The Complement Event}
		
		\begin{itemize} 
			
			\item The \textbf{complement} of an event $A$ is the set of all outcomes in the sample
			space that are not included in the outcomes of event $A$.
			\item We call the complement event of $A$ as $A^c$.
			\item The complement event of a die throw resulting in an even number is the
			die throwing an odd number.
			\item Question: if there is a $40\%$ chance of a randomly selected student being male, what is the probability of the selected student being female?
		\end{itemize}
	}

%------------------------------------------------------------%
\section*{Complement Rule}



The complement rule in Probability

$P(C^{\prime}) = 1- P(C)$



If the probability of C is $70 \%$ then the probability of $C^{\prime}$ is $30\%$



\subsection{Complement Events}


\begin{itemize}
	\item Suppose we are interested in whether or not a particular person is a dog owner
	\item We denote the event that a randomly selected person owning a dog as ``D"
	\item Necessarily the complement event is not owning a dog. We will denote this comeplement event as $D^c$
	\item Suppose the probability of owning a dog is 0.3
	\[ P(D) = 0.3\]
	\item The probability of not owning a dog is therefore:
	\[ P(D^c) = 1- P(D) = 1-0.3 =0.7.\]
\end{itemize}


\end{document}