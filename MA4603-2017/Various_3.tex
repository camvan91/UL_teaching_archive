\documentclass[]{report}

\voffset=-1.5cm
\oddsidemargin=0.0cm
\textwidth = 480pt

\usepackage{framed}
\usepackage{subfiles}
\usepackage{graphics}
\usepackage{newlfont}
\usepackage{eurosym}
\usepackage{amsmath,amsthm,amsfonts}
\usepackage{amsmath}
\usepackage{color}
\usepackage{amssymb}
\usepackage{multicol}
\usepackage[dvipsnames]{xcolor}
\usepackage{graphicx}
\begin{document}




\subsection*{Confidence Intervals}


\begin{itemize}
	\item A researcher takes a random sample of 500 urban residents and finds that
	122 have fibre-optic broadband access. 
	\item Calculate a 90\% Confidence Interval for
	the true percentage of residents who have fibre-optic broadband access.
\end{itemize}	
\subsection*{Question 10. } % 10 Marks

\begin{itemize}
	\item[a.] (1 Mark) The observation of the air pressure at a volume of 5 cubic metres was 19.87 bars.
	Calculate the residual from the regression model corresponding to this observation.
	\item[b.] (3 marks) Using the table above to justify your conclusion, test the null hypothesis that there
	is no monotonic (systematic) relationship between volume and pressure. State the null
	and alternative hypotheses clearly.
	\item[c.] (2 marks) Briefly explain why the use of linear regression to describe pressure as a function
	of volume is inappropriate.
\end{itemize}







% - Section 1 Probability
% - Section 2 Discrete Probability Distributions
% - Section 3 Normal Distribution and Sampling Distributions
% - Section 4 Single Sample Inference
% - Section 5 Two Sample Inference
% - Section 6 Information Theory
% - Section 7 Data Compression







\end{document}
