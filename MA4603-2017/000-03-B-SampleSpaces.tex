
\documentclass[]{report}

\voffset=-1.5cm
\oddsidemargin=0.0cm
\textwidth = 480pt

\usepackage{framed}
\usepackage{subfiles}
\usepackage{graphics}
\usepackage{newlfont}
\usepackage{eurosym}
\usepackage{amsmath,amsthm,amsfonts}
\usepackage{amsmath}
\usepackage{color}
\usepackage{amssymb}
\usepackage{multicol}
\usepackage[dvipsnames]{xcolor}
\usepackage{graphicx}
\begin{document}

\subsection{Set Theory : Union and Intersection}

Set theory is used to represent relationships among events.\\ \bigskip

\noindent \textbf{Union of two events:}\\
The union of events A and B is the event containing all the sample points
belonging to A or B or both. This is denoted $A\cup B$, (pronounce as ``A union
B").\\ 



\section{The Complement Event}

\begin{itemize} 

\item The complement of an event $A$ is the set of all outcomes in the sample
space that are not included in the outcomes of event $A$.
\item We call the complement event of $A$ as $A^c$.
\item The complement event of a die throw resulting in an even number is the
die throwing an odd number.
\item Question: if there is a $40\%$ chance of a randomly selected student being male, what is the probability of the selected student being female?
\end{itemize}

\section{Set Theory : Union and Intersection}

Set theory is used to represent relationships among events.\\ \bigskip

\noindent \textbf{Union of two events:}\\
The union of events A and B is the event containing all the sample points
belonging to A or B or both. This is denoted $A\cup B$, (pronounce as ``A union
B").\\ \bigskip
\noindent \textbf{Intersection of two events:}\\
The intersection of events A and B is the event containing all the sample
points common to both A and B. This is denoted $A\cap B$, (pronounce as ``A intersection
B").


In general, if A and B are two events in the sample space S, then
\begin{itemize} 
\item $A \subseteq B$ (A is a subset of B) = `if A occurs, so does B’
\item $\varnothing$ (the empty set) = an impossible event
\item $S$ (the sample space) = an event that is certain to occur
\end{itemize}



\section{Random Experiments and  Events}

%\frametitle{Random experiment}
\begin{itemize}
\item \textbf{Sample Space}, S. For a given experiment the sample space, S, is the set of all
possible outcomes.
\item \textbf{Event}, E. This is a subset of S. If an event E occurs, the outcome of the experiment is contained in E.
\end{itemize}




Sample spaces abound and are infinite in number. But there are a few that are frequently used for examples in introductory statistics. Below are the experiments and their corresponding sample spaces:

\begin{itemize}
\item For the experiment of flipping a coin, the sample space is {Heads, Tails} and has two elements.

\item For the experiment of flipping two coins, the sample space is {(Heads, Heads), (Heads, Tails), (Tails, Heads), (Tails, Tails) } and has four elements.
\end{itemize}

%==================================================================================== %
{ 
\begin{itemize}

\item For the experiment of flipping three coins, the sample space is {(Heads, Heads, Heads), (Heads, Heads, Tails), (Heads, Tails, Heads), (Heads, Tails, Tails), (Tails, Heads, Heads), (Tails, Heads, Tails), (Tails, Tails, Heads), (Tails, Tails, Tails) } and has eight elements.


\item For the experiment of flipping n coins, where n is a positive whole number, the sample space consists of 2n elements. There are a total of $C(n, k)$ ways to obtain k heads and $n - k$ tails for each number k from 0 to n.

\item For the experiment consisting of rolling a single six-sided die, the sample space is 
\[\{1, 2, 3, 4, 5, 6\} \]
\end{itemize}
\smallskip
\begin{itemize}
\item For the experiment of rolling two six-sided dice, the sample space consists of the set of the 36 possible pairings of the numbers 1, 2, 3, 4, 5 and 6.
\item For an experiment of drawing from a standard deck of cards, the sample space is the set that lists all 52 cards in a deck. For this example the sample space could only consider certain features of the cards, such as rank or suit.
\end{itemize}

\noindent \textbf{Basics of Probability}
%====================================%
\begin{itemize}
\item \textbf{Random Experiment}
A random experiment is one whose outime is determined by chance.

\item \textbf{Sample Space}
In a probabilistic experiment, the sample space is the set of all possible outcomes of the experiment. 
Suppose the probabilistic experiment is the toss of a dice. The six numbers that can appear face up, 
from 1 to 6, are the 6 possible outcomes of the experiment. Hence, the sample space is:
\[S={1,2,3,4,5,6}\]

\item \textbf{Sample Points}
In a probabilistic experiment, a sample point is one of the possible outcomes of the experiment. The set of all sample points is called sample space.

\item \textbf{Events}
An event *A* is merely collection of outcomes, or in other words, a subset of the sample space.
\end{itemize}

\textbf{Sample Spaces and Events}
\begin{itemize}
\item  The set of all possible outcomes of a probability experiment is called a \textbf{\emph{sample space}}, which is usually denoted by $\boldsymbol{S}$.
\item  The sample space is an exhaustive list of all the possible outcomes of an experiment. We call individual elements of this list \textbf{\emph{sample points}}.
\item  Each possible outcome is represented by one and only one sample point in the sample space.
\end{itemize}

\section{Sample Spaces}

% [http://cnx.org/content/m16845/latest/]
\begin{itemize}
\itemA complete list of all possible outcomes of a random experiment is called \textbf{sample space }or possibility space and is denoted by $\mathcal{S}$.


\itemA sample space is a set or collection of outcome of a particular random experiment.

\item For example, imagine a dart board. You are trying to find the probability of getting a bullseye. The dart board is the sample space. The probability of a dart hitting the dart board is 1.0.
\item For another example, imagine rolling a six sided die. The sample space is $\{1, 2, 3, 4, 5, 6\}$.
\end{itemize}

\section{Sample Spaces}
Consider a couple that have two children. Treating the gender of the children as an \textit{\textbf{ordered pair}} outcome of a random experiment, the sample space is 
\[\boldsymbol{S} = \{ (b,b), (b,g), (g,b), (g,g)\}.\]

Let us assume that each sample point is \textit{\textbf{equiprobable}}, with probability 0.25 for each sample point.



%% {-1cm}
Find the probability that both children are girls if it is known that: 

\begin{itemize}
\item[(a)] at least one of the children is a girl,
\item[(b)] the older child is a girl. 
\end{itemize}




%% {-1.5cm}
\textbf{Part a} \\
Find the probability that both children are girls if it is known that at least one of the children is a girl.
\[\boldsymbol{S} = \{ (b,b), (b,g), (g,b), (g,g)\}.\]




%% {-1.5cm}
\textbf{Part b} \\
Find the probability that both children are girls if it is known that the older child is a girl.
\[\boldsymbol{S} = \{ (b,b), (b,g), (g,b), (g,g)\}.\]


\section{Definitions}

%------------------------------------------------------------------%
\textbf{Experiments and Outcomes}
\begin{itemize}
\item  In the study of probability any process of observation is referred to as an \textbf{\emph{experiment}}.
\item  The results of an experiment (or other situation involving uncertainty) are called the \textbf{\emph{outcomes }}of the experiment.
\item  An experiment is called a \textbf{\emph{random experiment}} if the outcome can not be predicted.
\item  Typical examples of a random experiment are \begin{itemize} \item  a role of a die, \item  a toss of a coin, \item  drawing a card from a deck.
\end{itemize}
\item  If the experiment is yet to be performed we refer to `possible outcomes' or `possibilities' for short. If the experiment has been performed, we refer to `realized outcomes' or `realizations'.
\end{itemize}




\subsection*{Random experiment}
\begin{itemize}
\item \textbf{Sample Space, S}. For a given experiment the sample space, S, is the set of all
possible outcomes.
\item \textbf{Event, E}. This is a subset of S. If an event E occurs, the outcome of the
experiment is contained in E.
\end{itemize}




%--------------------------------------------------------%
\begin{framed}
\textbf{Example} 
For each of the following experiments, write out the sample space.
\begin{itemize}
\item  Experiment: Rolling a die once
\begin{itemize}
\item  Sample space $\boldsymbol{S} = \{1,2,3,4,5,6 \}$ \end{itemize}
\item  Experiment:  Tossing a coin
\begin{itemize}
\item  Sample space $\boldsymbol{S} = \{\mbox{ Heads },\mbox{ Tails} \}$ \end{itemize}
\item  Experiment:  Measuring a randomly selected person's height (cms)
\begin{itemize} \item  Sample space $\boldsymbol{S} = \mbox{ The set of all possible real numbers }$
\end{itemize}\end{itemize}
\end{framed}




\section{Sample Spaces and Events}

\begin{itemize}
\item The set of all possible outcomes of a probability experiment is called a
\textbf{\emph{sample space}}, which is usually denoted by \textbf{\emph{S}}.
\item The sample space is an exhaustive list of all the possible outcomes of an
experiment. We call individual elements of this list \textbf{\emph{sample points}}.
\item Each possible outcome is represented by one and only one sample point
in the sample space.
\end{itemize}


%--------------------------------------------------------------------------------%
{
\subsection{Events}

\textbf{Events}
\begin{itemize}
\item 
An \textbf{\emph{event}} is a specific outcome, or any collection of outcomes of an experiment.
\item  Formally, any subset of the sample space is an event.
\item  Any event which consists of a single outcome in the sample space is called an \textbf{\emph{elementary}}  or \textbf{\emph{simple event}}.
\item  Events which consist of more than one outcome are called \textbf{\emph{compound events}}.
\item  For example, an elementary event associated with the die example could be the ``die shows $3$".
\item  An compound event associated with the die example could be the ``die shows an even number".
\end{itemize}






\begin{framed}
\noindent \textbf{Examples of Events}

\noindent Consider the experiment of rolling a die once. Here the sample space
is given as $S = \{ 1,2,3,4,5,6\}$. \\ The following are examples of possible events.


\begin{multicols}{2}
\begin{itemize} 
\item A = score $< 4$ = $\{ 1,2,3\}$.
\item B = `score is even' = $\{ 2,4,6\}$.
\item C = `score is 7' = 0
\item $A\cup B$ = `the score is $< 4$  or even or both' = $\{ 1,2,3,4,6\}$
\item $A\cap B$ = `the score is $< 4$  and even’ = $ \{ 2 \}$
\item $A^c$ =`event A does not occur' = $ \{ 4,5,6\}$
\end{itemize}
\end{multicols}
\end{framed}
{
\textbf{Examples of Events}
Consider the experiment of rolling a die once. From before, the sample space is given as $S = \{1,2,3,4,5,6\}$. The following are examples of possible events.

\begin{itemize}
\item  $A$ = score $< 4$ = $\{1,2,3\}$.
\item  $B$ = `score is even' $= \{2,4,6\}$
\item  $C$ = `score is 7' = $\boldsymbol{\oslash}$\\
\item  $A \cup B$ = `the score is $< 4$ or even or both' = $\{1,2,3,4,6\}$\\
\item  $A \cap B$ = `the score is $< 4$ and even' = $\{2\}$\\
\item  $A^C$  = 'event $A$ does not occur' = $\{4,5,6\}$
\end{itemize}

%%%%%%%%%%%%%%%%%%%%%%%%%%%%%%%%%%%%%%%%%%%%%%%%%%%%%%%

\section{Probability}
If there are n possible outcomes to an experiment, and $m$ ways in which event $A$ can happen, then the probability of event $A$ ( which we write as $P(A)$) is
\[P(A) = \frac{m}{n}.\]

The probability of the event $A$ may be interpreted as the proportion of times that event $A$ will occur if we repeat the random experiment an infinite number of times.\\ \bigskip

%%%%%%%%%%%%%%%%%%%%%%%%%%%%%%%%%%%%%%%%%%%%%%%%%%%%%%%

\section{Rules}
\begin{enumerate}
\item  $0 \leq P(A) \leq 1$ ; the probability of any event lies between 0 and 1 inclusive.
\item  P(S) = 1 ; the probability of the sample space is always equal to 1.
\item  $P(A^c) = 1- P(A)$; how to compute the probability of the complement.
\end{enumerate}

%%%%%%%%%%%%%%%%%%%%%%%%%%%%%%%%%%%%%%%%%%%%%%%%%%%%%%%

\subsection{Random experiment}
\begin{itemize}
\item \textbf{Sample Space}, S. For a given experiment the sample space, S, is the set of all
possible outcomes.
\item \textbf{Event}, E. This is a subset of S. If an event E occurs, the outcome of the experiment is contained in E.
\end{itemize}


\subsection{The Complement Event}

\begin{itemize}

\item The complement of an event $A$ is the set of all outcomes in the sample
space that are not included in the outcomes of event $A$.
\item We call the complement event of $A$ as $A^c$.
\item The complement event of a die throw resulting in an even number is the
die throwing an odd number.
\item Question: if there is a $40\%$ chance of a randomly selected student being male, what is the probability of the selected student being female?
\end{itemize}


\subsection{Probability}
If there are n possible outcomes to an experiment, and m ways in which event
A can happen, then the probability of event A ( which we write as P(A)) is
\[ P(A) = \frac{m}{n} \]

The probability of the event A may be interpreted as the proportion of times
that event A will occur if we repeat the random experiment an infinite number
of times.\\ \bigskip

\textbf{Rules}:\\
\begin{itemize}
\item[1] $0 \leq P(A) \leq 1 $: the probability of any event lies between 0 and 1
inclusive.
\item[2] $P(S) = 1$ : the probability of the sample space is always equal to 1.
\item[3] $P(A^c) = 1-P(A)$ : how to compute the probability of the complement.
\end{itemize}

\section{Sample Spaces and Events}



%==================================================================%


\subsection{Random experiment}
\begin{itemize}
\item \textbf{Sample Space}, S. For a given experiment the sample space, S, is the set of all
possible outcomes.
\item \textbf{Event}, E. This is a subset of S. If an event E occurs, the outcome of the experiment is contained in E.
\end{itemize}
%==================================================================%


\section{Common Sample Spaces}



\subsection{Sample Space (2)}
Consider a random experiment in which a coin is tossed once, and a number between 1 and 4 is selected at random.
Write out the sample space $S$ for this experiment.

\bigskip

\[ S = \{(H,1),(H,2),(H,3),(H,4),(T,1),(T,2),(T,3),(T,4)\} \]

( $H$ and $T$ denoted `Heads' and `Tails' respectively. )

\newpage


\section{Sample Spaces - V2}
%--------------------------------------------------------%
{


\begin{itemize}

\item The sample space for a given set of events is the set of all possible outcomes of that event. 

\item The sample space is commonly denoted $S$.

\item 
For example, the sample space of a toss of two coins, each of which may land heads (H) or tails (T), is the set of all possible outcomes, i.e. $\{HH, HT, TH,TT\}$.

\item Members of the sample space set are commonly called \textbf{sample points.}


\item Suppose one urn contains three balls; one red, one blue and one green, and a second urn contain three balls; numbered 1, 2, and 3. \item An experiment consists of two balls being drawn at random (i.e. one from each urn).

\item Write out the sample space for this experiment.

\end{itemize}


\begin{itemize}

\item Where $R$,$B$ and $G$ are red, blue and green respectively.

\item Answer: 

\[S = \{(R,1),(R,2),(R,3),(B,1),(B,2),(B,3),(G,1),(G,2),(G,3) \}\]

\end{itemize}


%--------------------------------------------------------%
\newpage 

\subsection{Sample Space}

\begin{itemize}
\item Consider a coin toss game played by players A and B. \item Player A tosses a fair coin three times, with the number of heads being the score for player A. \item Player B tosses a coin four times, with the number of heads being the score for player B.
\item Questions
\begin{itemize}
\item[a.] Write out the sample space for the scores for both players A and B.
\item[b.] Write out the sample space for the scores of C, where C is the difference of the two scores (i.e. B-A)
\item[c.] Are the sample points for the sample space of A and B equally probable? Provide a brief justification for your answer.
\end{itemize}


\end{itemize}

\subsection{Sample Space}

\begin{itemize}
\item $S_A$ : $\{0,1,2,3\}$ 
\item $S_B$ : $\{0,1,2,3,4\}$ 
\item $S_C$ : $\{-3,-2,\ldots,3,4\}$
\item Events are not equiprobable. Probability of 1 head is different than for 3 heads for A. Recall Probability Trees.
\end{itemize}

%===============================================%

\subsection*{Sample Question: Probability of Two Dice Rolls}

Two dice A and B are cast. What is the probability of getting
\begin{itemize}
\item[1] a total of 2 or a total of 6?
\item[2] a total greater than 9?
\item[3] a total which is three times as great as other possible totals?
\end{itemize}

\bigskip
\noindent \textbf{Sample Space for Ordered Pair of Dice Rolls}


{

\begin{center}
\begin{tabular}{|c||c|c|c|c|c|c|}
\hline
\phantom{space}& \phantom{sp} \textbf{1}\phantom{sp}&\phantom{sp} \textbf{2}\phantom{sp}&\phantom{sp} \textbf{3}\phantom{sp}&\phantom{sp} \textbf{4}\phantom{sp}&\phantom{sp} \textbf{5} \phantom{sp}&\phantom{sp}\textbf{6}\phantom{sp}\\ \hline\hline
\textbf{1}&(1,1)&3(1,2)&(1,3)&(1,4)&(1,5)&(1,6) \\ \hline
\textbf{2}&(2,1)&(2,2)&(2,3)&(2,4)&(2,5)&(2,6) \\ \hline
\textbf{3}&(3,1)&(3,2)&(3,3)&(3,4)&(3,5)&(3,6) \\ \hline
\textbf{4}&(4,1)&(4,2)&(4,3)&(4,4)&(4,5)&(4,6) \\ \hline
\textbf{5}&(5,1)&(5,2)&(5,3)&(5,4)&(5,5)&(5,6) \\ \hline
\textbf{6}&(6,1)&(6,2)&(6,3)&(6,4)&(6,5)&(6,6) \\ \hline
\end{tabular}
\end{center}

\noindent \textbf{Sample Space}: $\{ (1,1),(1,2),(1,3), \ldots (6,6)\}$
}
\bigskip

\noindent \textbf{Sample Space for Sum of Dice Rolls}

{

\begin{center}
\begin{tabular}{|c||c|c|c|c|c|c|}
\hline
\phantom{space}& \phantom{sp} \textbf{1}\phantom{sp}&\phantom{sp} \textbf{2}\phantom{sp}&\phantom{sp} \textbf{3}\phantom{sp}&\phantom{sp} \textbf{4}\phantom{sp}&\phantom{sp} \textbf{5} \phantom{sp}&\phantom{sp}\textbf{6}\phantom{sp}\\ \hline\hline
\textbf{1}&2&3&4&5&6&7 \\ \hline
\textbf{2}&3&4&5&6&7&8 \\ \hline
\textbf{3}&4&5&6&7&8&9 \\ \hline
\textbf{4}&5&6&7&8&9&10 \\ \hline
\textbf{5}&6&7&8&9&10&11 \\ \hline
\textbf{6}&7&8&9&10&11&12 \\ \hline
\end{tabular}
\end{center}


\noindent \textbf{Sample Space}: $\{ 1,2,3,4,5,6,7,8,9,10,11,12\}$
}
%----------------------------------------%


\noindent \textbf{Probability}

\begin{itemize}
\item A total of 2 or 6

\[ E_1 = \{ (1,1) \} \qquad \therefore  P(E_1)  = \frac{1}{36} \]
(Remark : how many outcomes have a sum of "2" out of the 36 outcomes?)
\medskip
\item A total of  6

\[ E_1 = \{ (1,5), (5,1), (4,2), (2,4), (3,3) \} \qquad \therefore  P(E_2)  = \frac{5}{36} \]
(Remark : how many outcomes have a sum of "6" out of the 36 outcomes?)
\medskip
\item A total greater than 9 

\[ E_3 = \{ (4,6) , (5,5), (6,4), (6,5), (6,6), (3,3)\} \qquad \therefore P(E_2)  = \frac{6}{36} \]
\medskip
\item  A total which is three times as great as other possible totals. \\

These totals are 6, 9 and 12.

\[ E_4 = \{ (1,5) , (2,4), (4,2), (1,5), (3,3), (6,3), (4,5) (5,4) , (6,6) \} \qquad \therefore  P(E_3)  = \frac{10}{36} \]

\end{itemize}












\begin{verbatim}


%---------------------------------------------------------------------------------------------------------------%
%----R Code ----
%---------------------------------------------------------------------------------------------------------------%
n=60000
Y=numeric(n)
for ( i in 1:n){

X=floor(runif(100,1,7))
Y[i]=sum(X)
}

Y
hist(Y,breaks=seq(300,400,by=10),main=c("Totals of 100 Die Throws"),cex.lab=1.4,font.lab=2,xlab=c("Total Score"))

hist(Y,breaks=seq(300,400,by=20),main=c("Totals of 100 Die Throws"),cex.lab=1.4,font.lab=2,xlab=c("Total Score"))



Z=seq(1:n)
Y/Z

plot(Y/Z,type="l",col="red",main=c("Die Rolls: Running Average"),font.lab=2,ylab="Average Value", xlab=
" Number of Throws")
abline(h=3.5,col="green")


#####################################################

plot(Z,Z.y,pch=16,col="red",ylim=c(2.5,5.5),main=c("Variance"),font.lab=2,ylab=" ", xlab="X: Green  Y: Blue  Z: Red" )

points(Y,Y.y,pch=16,col="blue" )
points(X,X.y,pch=16,col="green" )
points(c(1000,1000,1000),c(3,4,5),pch=18,cex=1.2)
lines(c(1000,1000),c(2.75,5.25),lty=3)



n=100000
Y=numeric(n)
for ( i in 1:n){

X=floor(runif(100,1,7))
Y[i]=sum(X)
}

Y
hist(Y,breaks=seq(270,430,by=2),main=c("Totals of 100 Die Throws (n= 100,000)"),cex.lab=1.4,font.lab=2,xlab=c("Total Score")) 


\end{verbatim}


%=================================================================================%







\subsection{Examples of Sample Spaces}
\begin{itemize}
\itemThe following list consists of sample spaces of examples of random experiments and their respective outcomes.

\itemThe tossing of a coin, sample space is $\{Heads, Tails\}$.

\item The roll of a die, sample space is {1, 2, 3, 4, 5, 6}

\item The selection of a numbered ball (1-50) in an urn, sample space is $\{1, 2, 3, 4, 5, ...., 50\}$

\item Percentage of calls dropped due to errors over a particular time period, sample space is $\{2\%, 14\%, 23\%, ......\}$

\item The time difference between two messages arriving at a message centre, sample space is {0, ...., infinity}

\item The time difference between two different voice calls over a particular network, sample space is {0, ...., infinity}
\end{itemize}







\end{document}
