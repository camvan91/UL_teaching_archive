\documentclass[a4paper,12pt]{article}
%%%%%%%%%%%%%%%%%%%%%%%%%%%%%%%%%%%%%%%%%%%%%%%%%%%%%%%%%%%%%%%%%%%%%%%%%%%%%%%%%%%%%%%%%%%%%%%%%%%%%%%%%%%%%%%%%%%%%%%%%%%%%%%%%%%%%%%%%%%%%%%%%%%%%%%%%%%%%%%%%%%%%%%%%%%%%%%%%%%%%%%%%%%%%%%%%%%%%%%%%%%%%%%%%%%%%%%%%%%%%%%%%%%%%%%%%%%%%%%%%%%%%%%%%%%%
\usepackage{eurosym}
\usepackage{vmargin}
\usepackage{amsmath}
\usepackage{graphics}
\usepackage{epsfig}
\usepackage{subfigure}
\usepackage{fancyhdr}

\setcounter{MaxMatrixCols}{10}
%TCIDATA{OutputFilter=LATEX.DLL}
%TCIDATA{Version=5.00.0.2570}
%TCIDATA{<META NAME="SaveForMode" CONTENT="1">}
%TCIDATA{LastRevised=Wednesday, February 23, 2011 13:24:34}
%TCIDATA{<META NAME="GraphicsSave" CONTENT="32">}
%TCIDATA{Language=American English}

\pagestyle{fancy}
\setmarginsrb{20mm}{0mm}{20mm}{25mm}{12mm}{11mm}{0mm}{11mm}
\lhead{MA4413} \rhead{Mr. Kevin O'Brien}
\chead{Version 1 - 16th October 2013}
%\input{tcilatex}

\begin{document}

\section*{Formulae}
\subsection*{Remarks}
Information theory and data compression formulas to be added.
%-------------------------------------------------%
\subsection*{Descriptive Statistics}
\begin{itemize}
\item Sample Variance
\begin{equation*}
s^2 = \frac{\sum (x-\bar{x})^2}{n-1}
\end{equation*}
\end{itemize}
%-------------------------------------------------%
\subsection*{Probability}
\begin{itemize}

\item Conditional probability:
\begin{equation*}
P(B|A)=\frac{P\left( A\text{ and }B\right) }{P\left( A\right) }.
\end{equation*}


\item Bayes' Theorem:
\begin{equation*}
P(B|A)=\frac{P\left(A|B\right) \times P(B) }{P\left( A\right) }.
\end{equation*}





\item Binomial probability distribution:
\begin{equation*}
P(X = k) = ^{n}C_{k} \times p^{k} \times \left( 1-p\right) ^{n-k}\qquad \left( \text{where  }
^{n}C_{k} =\frac{n!}{k!\left(n-k\right) !}. \right)
\end{equation*}

\item Poisson probability distribution:
\begin{equation*}
P(X = k) =\frac{m^{k}\mathrm{e}^{-m}}{k!}.
\end{equation*}

\item Exponential probability distribution:
\begin{equation*}
P(X \leq k) = \begin{cases}
1-e^{- k/\mu}, & k \ge 0, \\
0, & k < 0.
\end{cases}\qquad \left( \text{where  }
\mu = {1\over \lambda}\right)
\end{equation*}
\end{itemize}

\subsection*{Confidence Intervals}
{\bf One sample}
\begin{eqnarray*} S.E.(\bar{X})&=&\frac{\sigma}{\sqrt{n}}.\\\\
S.E.(\hat{P})&=&\sqrt{\frac{\hat{p}\times(100-\hat{p})}{n}}.\\
\end{eqnarray*}
{\bf Two samples}
\begin{eqnarray*}
S.E.(\bar{X}_1-\bar{X}_2)&=&\sqrt{\frac{\sigma^2_1}{n_1}+\frac{\sigma_2^2}{n_2}}.\\\\
S.E.(\hat{P_1}-\hat{P_2})&=&\sqrt{\frac{\hat{p}_1\times(100-\hat{p}_1)}{n_1}+\frac{\hat{p}_2\times(100-\hat{p}_2)}{n_2}}.\\\\
\end{eqnarray*}
\subsection*{Hypothesis tests}
{\bf One sample}
\begin{eqnarray*}
S.E.(\bar{X})&=&\frac{\sigma}{\sqrt{n}}.\\\\
S.E.(\pi)&=&\sqrt{\frac{\pi\times(100-\pi)}{n}}
\end{eqnarray*}
{\bf Two large independent samples}
\begin{eqnarray*}
S.E.(\bar{X}_1-\bar{X}_2)&=&\sqrt{\frac{\sigma^2_1}{n_1}+\frac{\sigma_2^2}{n_2}}.\\\\
S.E.(\hat{P_1}-\hat{P_2})&=&\sqrt{\left(\bar{p}\times(100-\bar{p})\right)\left(\frac{1}{n_1}+\frac{1}{n_2}\right)}.\\
\end{eqnarray*}
{\bf Two small independent samples}
\begin{eqnarray*}
S.E.(\bar{X}_1-\bar{X}_2)&=&\sqrt{s_p^2\left(\frac{1}{n_1}+\frac{1}{n_2}\right)}.\\\\
s_p^2&=&\frac{s_1^2(n_1-1)+s_2^2(n_2-1)}{n_1+n_2-2}.\\
\end{eqnarray*}
{\bf Paired sample}
\begin{eqnarray*}
S.E.(\bar{d})&=&\frac{s_d}{\sqrt{n}}.\\\\
\end{eqnarray*}
{\bf Standard Deviation of case-wise differences (computational formula)}
\begin{eqnarray*}
s_d = \sqrt{ {\sum d_i^2 - n\bar{d}^2 \over n-1}}.\\\\
\end{eqnarray*}

\newpage

\begin{itemize}
	\item Sample Mean
	\[\bar{x} = \frac{\sum x_i}{n}\]
	
	
	\item Inter-Quartile Range
	\[IQR = Q_3 - Q_1 \]
	
	\item Trimean
	\[ {Q_3 + Q_1 \over 2 } \]
\end{itemize}


\section{Formulas for Statistics}

\begin{enumerate}
	\item Sample mean
	\begin{equation*}
	\bar{x}=\frac{\sum x_{i}}{n}.
	\end{equation*}
	
	\item Sample standard deviation
	\begin{equation*}
	s=\sqrt{\frac{\sum \left( x_{i}-\bar{x}\right) ^{2}}{%
			n-1}}.
	\end{equation*}
	
	\item Conditional probability:
	\begin{equation*}
	P(B|A)=\frac{P\left( A\text{ and }B\right) }{P\left( A\right) }.
	\end{equation*}
	
\end{enumerate}


\section{Formulas for Statistics}

\begin{enumerate}
	\item Sample mean
	\begin{equation*}
	\bar{x}=\frac{\sum x_{i}}{n}.
	\end{equation*}
	
	\item Sample standard deviation
	\begin{equation*}
	s=\sqrt{\frac{\sum \left( x_{i}-\bar{x}\right) ^{2}}{%
			n-1}}.
	\end{equation*}
	
	\item Conditional probability:
	\begin{equation*}
	P(B|A)=\frac{P\left( A\text{ and }B\right) }{P\left( A\right) }.
	\end{equation*}
	
\end{enumerate}



%--------------------------------------------------------------------%


\section{Formulas}
%=======================================================%
\subsection*{Confidence Intervals}
{\bf One sample}
\begin{eqnarray*} S.E.(\bar{X})&=&\frac{\sigma}{\sqrt{n}}.\\\\
	S.E.(\hat{P})&=&\sqrt{\frac{\hat{p}\times(100-\hat{p})}{n}}.\\
\end{eqnarray*}
{\bf Two samples}
\begin{eqnarray*}
	S.E.(\bar{X}_1-\bar{X}_2)&=&\sqrt{\frac{\sigma^2_1}{n_1}+\frac{\sigma_2^2}{n_2}}.\\\\
	S.E.(\hat{P_1}-\hat{P_2})&=&\sqrt{\frac{\hat{p}_1\times(100-\hat{p}_1)}{n_1}+\frac{\hat{p}_2\times(100-\hat{p}_2)}{n_2}}.\\\\
\end{eqnarray*}

%=======================================================%

\begin{equation}
\bar{X} \pm t_{\nu,\alpha/2}\mbox{S.E.}(\bar{X})
\end{equation}

\begin{equation}
\hat{P} \pm t_{\nu,\alpha/2}\mbox{S.E.}(\hat{P})
\end{equation}

\subsection{Hypothesis Testing}
% Inference: Two samples
\begin{equation}
\frac{(\hat{P}_{1}-\hat{P}_{2})-(P_{1}-P_{2})}{S.E.(\hat{P}_{1}-\hat{P}_{2})}
\end{equation}

\begin{equation}
\frac{(\bar{X}-\bar{Y})-(\mu_{x}-\mu_{y})}{S.E.(\bar{X}-\bar{Y})}
\end{equation}

\subsection*{Hypothesis tests}
{\bf One sample}
\begin{eqnarray*}
	S.E.(\bar{X})&=&\frac{\sigma}{\sqrt{n}}.\\\\
	S.E.(\pi)&=&\sqrt{\frac{\pi\times(100-\pi)}{n}}
\end{eqnarray*}
{\bf Two large independent samples}
\begin{eqnarray*}
	S.E.(\bar{X}_1-\bar{X}_2)&=&\sqrt{\frac{\sigma^2_1}{n_1}+\frac{\sigma_2^2}{n_2}}.\\\\
	S.E.(\hat{P_1}-\hat{P_2})&=&\sqrt{\left(\bar{p}\times(100-\bar{p})\right)\left(\frac{1}{n_1}+\frac{1}{n_2}\right)}.\\
\end{eqnarray*}
{\bf Two small independent samples}
\begin{eqnarray*}
	S.E.(\bar{X}_1-\bar{X}_2)&=&\sqrt{s_p^2\left(\frac{1}{n_1}+\frac{1}{n_2}\right)}.\\\\
	s_p^2&=&\frac{s_1^2(n_1-1)+s_2^2(n_2-1)}{n_1+n_2-2}.\\
\end{eqnarray*}
{\bf Paired sample}
\begin{eqnarray*}
	S.E.(\bar{d})&=&\frac{s_d}{\sqrt{n}}.\\\\
\end{eqnarray*}
{\bf Standard deviation of case-wise differences}
\begin{eqnarray*}
	s_d = \sqrt{ {\sum d_i^2 - n\bar{d}^2 \over n-1}}.\\\\
\end{eqnarray*}

\section*{Formulae}
\subsection*{Probability}


\begin{itemize}
	
	\item Conditional probability:
	\begin{equation*}
	P(B|A)=\frac{P\left( A\text{ and }B\right) }{P\left( A\right) }.
	\end{equation*}
	
\end{itemize}
\begin{itemize}
	
	\item Conditional probability:
	\begin{equation*}
	P(B|A)=\frac{P\left( A\text{ and }B\right) }{P\left( A\right) }.
	\end{equation*}
	
	
	\item Bayes' Theorem:
	\begin{equation*}
	P(B|A)=\frac{P\left(A|B\right) \times P(B) }{P\left( A\right) }.
	\end{equation*}
	
	
	\item Sample Variance
	\begin{equation*}
	s^2 = \frac{\sum (x-\bar{x})^2}{n-1}
	\end{equation*}
	
	
	\item Binomial probability distribution:
	\begin{equation*}
	P(X = k) = ^{n}C_{k} \times p^{k} \times \left( 1-p\right) ^{n-k}\qquad \left( \text{where}\qquad
	^{n}C_{k} =\frac{n!}{k!\left(n-k\right) !}. \right)
	\end{equation*}
	
	\item Poisson probability distribution:
	\begin{equation*}
	P(X = k) =\frac{m^{k}\mathrm{e}^{-m}}{k!}.
	\end{equation*}
\end{itemize}



\subsection*{Confidence Intervals}
{ One sample}
\begin{eqnarray*} S.E.(\bar{X})&=&\frac{\sigma}{\sqrt{n}}.\\\\
	S.E.(\hat{P})&=&\sqrt{\frac{\hat{p}\times(100-\hat{p})}{n}}.\\
\end{eqnarray*}
{ Two samples}
\begin{eqnarray*}
	S.E.(\bar{X}_1-\bar{X}_2)&=&\sqrt{\frac{\sigma^2_1}{n_1}+\frac{\sigma_2^2}{n_2}}.\\\\
	S.E.(\hat{P_1}-\hat{P_2})&=&\sqrt{\frac{\hat{p}_1\times(100-\hat{p}_1)}{n_1}+\frac{\hat{p}_2\times(100-\hat{p}_2)}{n_2}}.\\\\
\end{eqnarray*}
\subsection*{Hypothesis tests}
{ One sample}
\begin{eqnarray*}
	S.E.(\bar{X})&=&\frac{\sigma}{\sqrt{n}}.\\\\
	S.E.(\pi)&=&\sqrt{\frac{\pi\times(100-\pi)}{n}}
\end{eqnarray*}
{ Two large independent samples}
\begin{eqnarray*}
	S.E.(\bar{X}_1-\bar{X}_2)&=&\sqrt{\frac{\sigma^2_1}{n_1}+\frac{\sigma_2^2}{n_2}}.\\\\
	S.E.(\hat{P_1}-\hat{P_2})&=&\sqrt{\left(\bar{p}\times(100-\bar{p})\right)\left(\frac{1}{n_1}+\frac{1}{n_2}\right)}.\\
\end{eqnarray*}
{ Two small independent samples}
\begin{eqnarray*}
	S.E.(\bar{X}_1-\bar{X}_2)&=&\sqrt{s_p^2\left(\frac{1}{n_1}+\frac{1}{n_2}\right)}.\\\\
	s_p^2&=&\frac{s_1^2(n_1-1)+s_2^2(n_2-1)}{n_1+n_2-2}.\\
\end{eqnarray*}
{\bf Paired sample}
\begin{eqnarray*}
	S.E.(\bar{d})&=&\frac{s_d}{\sqrt{n}}.\\\\
\end{eqnarray*}
{ Standard Deviation of case-wise differences (computational formula)}
\begin{eqnarray*}
	s_d = \sqrt{ {\sum d_i^2 - n\bar{d}^2 \over n-1}}.\\\\
\end{eqnarray*}

\end{document}

