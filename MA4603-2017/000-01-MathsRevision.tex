
\documentclass[]{report}

\voffset=-1.5cm
\oddsidemargin=0.0cm
\textwidth = 480pt

\usepackage{framed}
\usepackage{subfiles}
\usepackage{graphics}
\usepackage{newlfont}
\usepackage{eurosym}
\usepackage{amsmath,amsthm,amsfonts}
\usepackage{amsmath}
\usepackage{color}
\usepackage{amssymb}
\usepackage{multicol}
\usepackage[dvipsnames]{xcolor}
\usepackage{graphicx}
\begin{document}
% \tableofcontents




\section{Notation for Data Set}

\begin{itemize}
	\item Suppose that we have a data set with $n$ observations. For each observation, a measure is recorded. Conventionally the measures are denoted $x$ unless a more suitable notation is available. 
	\item A subscript can be used to indicate which observation the measure is for.
	\item Hence we would write a data set as follows; $(x_{1}, x_{2},x_{3} , x_{1} \dots x_{n})$ (i.e. the first, second, third ... $n$th observation).
\end{itemize}


\section{Relational operators}
\begin{itemize}
	\item $>$  means `is greater than’
	\item $\geq$ means `is greater than or equal to’
	\item $<$ means `is less than’
	\item $\leq$ means `is less than or equal to’
	\item $\neq$ means `is not equal to’
	\item $\approx$ or $\simeq$ means `is approximately equal to’
\end{itemize}





\section{Factorials Numbers}

A factorial is a positive whole number, based on a number $n$ , and which is written as $``n!"$. The factorial $n!$ is defined as follows:

\[n!  =n \times (n-1) \times (n-2) \times \ldots \times 2 \times 1 \]
The factorial function (symbol: !) just means to multiply a series of descending natural numbers. \\ \bigskip

Remark $n!  =n \times (n-1)!$\\ \bigskip

	
\noindent \textbf{ Example: }
	
	\begin{itemize}
		\item $3!  = 3 \times 2  \times 1 = 6 $
		
		\item $4!  = 4 \times 3! = 4 \times 3 \times 2 \times 1 = 24$
	\end{itemize}
	Remark $0! = 1$ not $0$.
	
\textbf{ Example: }

\begin{multicols}{2}
\begin{itemize}
\item $4! = 4 \times 3 \times 2 \times 1 = 24$
\item $7! = 7 \times 6 \times 5 \times 4 \times 3 \times 2 \times 1 = 5,040$
\item $3!  = 3 \times 2  \times 1 = 6 $
\item $1! = 1$
\item $0! = 1 $
\end{itemize}
\end{multicols}

\[ (n+1)! = (n+1) \times n! \] 
Equivalently

\[n! = n \times (n-1)!  = n \times (n-1) \times (n-2)! = \ldots \]

\begin{itemize}
\item $4!  = 4 \times 3! = 4 \times 3 \times 2 \times 1 = 24$
\item $5! = 5 x 4!$
\item $6! = 6 x 5!$  and so on
\item $6! = 6 \times 5!  = 6 \times 5 \times 4!$
\end{itemize}


\begin{itemize}
	\item  $n!$ and $k!$ are the coefficients of $n$ and $k$ respectively.
	\item  $n! = n \times (n-1) \times (n-2) \times \ldots \times 2 \times 1$
	\item  For example $5! = 5\times4\times3\times2\times1 = 120$
	\item  $n! = n \times (n-1)!$
	\item  Importantly $0! = 1$ not 0.
\end{itemize}


\begin{itemize}

	
	\item factorials 
	\[ n! = (n)\times (n-1)\times(n-2) \times \ldots \times 1 \]
	\begin{itemize}
		\item $5! = 5 \times 4 \times 3 \times 2 \times 1 = 120 $
		\item $3! = 3 \times 2 \times 1$
	\end{itemize}
\end{itemize}	


\newpage
	\subsection{Example: Factorials }
	
	\noindent \textbf{Examples:}
	
	\begin{itemize}
		\item $4! = 4 \times 3 \times 2 \times 1 = 24$
		\item $7! = 7 \times 6 \times 5 \times 4 \times 3 \times 2 \times 1 = 5,040$
		\item $1! = 1$
		\item $0! = 1 $
	\end{itemize}
	Importantly 
	\[n! = n \times (n-1)!  = n \times (n-1) \times (n-2)! \]
	For Example
	\[6! = 6 \times 5!  = 6 \times 5 \times 4! \]
	
	%%- \textbf{Factorials Numbers}
	
	
	\begin{itemize}
		\item $3!  = 3 \times 2  \times 1 = 6 $
		
		\item $4!  = 4 \times 3! = 4 \times 3 \times 2 \times 1 = 24$
		
		\item factorials 
		\[ n! = (n)\times (n-1)\times(n-2) \times \ldots \times 1 \]
		\begin{itemize}
			\item $5! = 5 \times 4 \times 3 \times 2 \times 1 = 120 $
			\item $3! = 3 \times 2 \times 1$
		\end{itemize}
		\item Zero factorial : Remark $0! = 1$ not $0$.
		\[ 0! =  1 \]
	\end{itemize}
	

\section{Choose Operator \ Binomial Coefficients}

For the positive integer $n$ and non-negative integer $k$ ( with $k\leq n$), the choose operater is calculated as follows:

\[ {n \choose k} = \frac{n!}{k! \times (n-k)!} \]

\begin{framed}
	
\[ {n \choose r} = \frac{n!}{(n-r)! r!} \]
\smallskip
\[ {6 \choose 3} = \frac{6!}{(6-3)! 3!} = \frac{6!}{3! \times 3!}\]
\smallskip
\[ \frac{6!}{3! \times 3!} = \frac{6 \times 5 \times 4 \times 3!}{3! \times 3!} = \frac{120}{6} = 120\]
\smallskip
\[ {6 \choose 2} = \frac{6!}{2! \times (6-2)!} = \frac{6!}{2! \times 4!}  \]\[\mbox{   } = \frac{6 \times 5 \times 4!}{2! \times 4!} 
= 30/2 =15 \]

\end{framed}


More examples of Binomial coefficients on blackboard.



\noindent \textbf{Example}

\[{3\choose 1}  = { 3! \over 1! \times(3-1)!} { 3 \times 2! \over 1! \times 2!} = \frac{3}{1} = 3 \]


%--------------------------------------------------------------------------------------%

\textbf{The Choose Operator}
(Remember $0!$  is always equal to 1)
\begin{multicol}{2}
\begin{itemize}
	\item ${3 \choose 0} = 1$
	\item ${3 \choose 1} = 3$
	\item ${3 \choose 2} = 3$
	\item ${3 \choose 3} = 1$
\end{itemize}
\end{multicol}



Exercises: Evaluate the following:

\begin{itemize}
	\item[1] ${5 \choose 2}$
	\item[2] ${5 \choose 0}$
	\item[3] ${10 \choose 1}$
	\item[4] ${10 \choose 9}$
\end{itemize}







\section{Binomial Coefficients}

For the positive integer $n$ and non-negative integer $k$ ( with $k\leq n$), the choose operater is calculated as follows:

\[ {n \choose k} = \frac{n!}{k! \times (n-k)!} \]





Evaluate the following:
		\subsection*{More Exercises}				
		Evaluate the following:
		\begin{multicols}{3}
			\begin{enumerate}
				\item[(i)] ${5 \choose 2}$
				\item[(ii)] ${5 \choose 0}$
				\item[(iii)] ${6 \choose 3}$
				\item[(iv)] ${6 \choose 6}$
				\item[(v)] ${10 \choose 1}$
				\item[(vi)] ${10 \choose 9}$
			\end{enumerate}        
		\end{multicols}	



	{
		
		\begin{framed}
			
			\[ {n \choose k} = \frac{n!}{k! \times (n-k)!} \]
		\end{framed}		
		\begin{multicols}{2}
			\begin{itemize}
				\item ${6 \choose 2} = 15$
				\item ${5 \choose 2} = 10$  
				\item ${4 \choose 0} = 1$  
				\item ${4 \choose 3} = 4$  
			\end{itemize}
		\end{multicols}	
		
		
		
		\subsection*{Formula}
		\[ \binom nk  = \frac{n!}{k!(n-k)!} = \frac{n(n-1)\ldots(n-k+1)}{k(k-1)\dots 1},\]
		which can be written using factorials as  whenever $k\leq n$
		
{
	\textbf{Binomial Coefficients}
	
	In the last class, we came across binomial coefficients. Informally, binomial coefficients are the number of ways $k$ items can be selected from a group of $n$ items. 
	The binomial coefficient indexed by n and k is usually written as $^nC_k$ or
	\[ {n \choose k}\].
	$C$ is colloqially known as the ``choose operator".
	
	\[ {n \choose k} = \frac{n!}{k! \times (n-k)!} \]
	
	(We call the operator the choose operator. We will use both notations interchangeably.)
	
	

}
%---------------------------------------------------------------------------%
{
	\textbf{Probability Mass Function}
	(Formally defining something mentioned previously)
	\begin{itemize} \item a probability mass function (pmf) is a \textbf{\emph{function}}
		that gives the probability that a discrete random variable is exactly equal to some
		value.
		\[P(X=k)\]
		\item The probability mass function is often the primary means of defining a discrete
		probability distribution
		\item It is conventional to present the probability mass function in the form of a table.
		\item The p.m.f of a value $k$ is often denoted $f(k)$.
	\end{itemize}
}
%--------------------------------------------------------------------------------------%
{
	\textbf{ Binomial Example 1 }
	(Revision from Last Class)\\
	Suppose a die is tossed 5 times. What is the probability of getting exactly 2 fours?
	
	\textbf{Solution:} This is a binomial experiment in which the number of trials is equal to 5, the number of successes is equal to 2, and the probability of success on a single trial is 1/6 or about 0.167. 
	\\
	\bigskip
	Therefore, the binomial probability is:
	
	\[P(X=2) = ^5C_2 \times (1/6)^2 \times (5/6)^3 = 0.161\]
}

%--------------------------------------------------------------------------------------%
{
	\textbf{ Binomial Example 2 }
	Suppose there is a container that contains 6 items.  The probability that any one of these items is defective is 0.3. Suppose all six items are inspected. 
	\begin{itemize}
		\item What is the probability of 3 defective components?
		\item What is the probability of 4 defective components?
	\end{itemize}
	
	\[ P(3\text{ defects}) = f(3) = P(X = 3) = {6\choose 3}0.3^3 (1-0.3)^{6-3} = 0.1852 \]
	\[ P(4\text{ defects}) = f(4) = P(X = 4) = {6\choose 4}0.3^4 (1-0.3)^{6-4} = 0.0595 \]
}
		

\subsection{Binomial Coefficients}



\[ \binom nk  = \frac{n!}{k!(n-k)!} = \frac{n(n-1)\ldots(n-k+1)}{k(k-1)\dots 1},\]
which can be written using factorials as  whenever $k\leq n$


Evaluat the following
\begin{itemize}
	\item $^{10}C_0$ 
	\item $^{10}C_1$
	\item $^6C_3$
\end{itemize}
Solutions
\begin{itemize}
	\item $^{10}C_0  = 10! / (10! \times 0!) = 1$
	\item $^{10}C_1 =  10! / (9! \times 1!) =  = 1$
	\item $^6C_3$
\end{itemize}



\subsection{Binomial Coefficients}

\[ \binom 5 2  = \frac{5!}{2!\;(5-2)!} = \frac{5.4.3!}{2! .3!} = \frac{5.4}{2.1} = 10\]


\[ \binom 5 0   = \frac{5!}{0!\;(5-0)!} = \frac{5!}{0! .5!} = \frac{5!}{2!} = 1\]
Recall $0! =1$

\begin{itemize}
	\item In the last class, we came across binomial coefficients. Informally, binomial coefficients are the number of ways $k$ items can be selected from a group of $n$ items. 
	
	\item The binomial coefficient indexed by n and k is usually written as $^nC_k$ or
	\[ {n \choose k}\].
	
	\item $C$ is colloqially known as the ``choose operator".
	
	\[ {n \choose k} = \frac{n!}{k! \times (n-k)!} \]
	
	\item (We call the operator the choose operator. We will use both notations interchangeably.)
	
	
\end{itemize}



\begin{itemize}
	\item $n!$ and $k!$ are the coefficients of $n$ and $k$ respectively.
	\item $n! = n \times (n-1) \times (n-2) \times \ldots \times 2 \times 1$
	\item For example $5! = 5\times4\times3\times2\times1 = 120$
	\item $n! = n \times (n-1)!$
	\item Importantly $0! = 1$ not 0.
\end{itemize}
\[ {6 \choose 2} = \frac{6!}{2! \times (6-2)!} = \frac{6!}{2! \times 4!}  \]\[\mbox{   } = \frac{6 \times 5 \times 4!}{2! \times 4!} 
= 30/2 =15 \]
More examples of Binomial coefficients on blackboard.



\[ { 52 \choose 5 } = 2598960 \]



Easier example

\[ { 8 \choose 3 } = \frac{8!}{3! \times 5!} \]


We can divide above and below by (5!), leaving us with

\[ \frac{8 \times 7 \times 6}{3 \times 2 \times 1} = 56 \]   

There are 56 way to pick 3 objects at random from group of 8 objects.

	
		
		

\section{Maximum, Minimum and Range}

\subsection{Range}
\begin{itemize}
\item The Range is simply the difference between the maximum value
\end{itemize}

\begin{framed}
	\begin{itemize}
		\item The \textbf{range} of a set of data is the difference between the highest and lowest values in the data set.
		\item Consider the following data set
		\[  \{ 	 39,  23,  34,  41,  37,  27,  44 \}\]
		\item The highest value (i.e. the maximum) is 44.
		\item The lowest value (i.e. the minimum) is 23.
		\item The range is the difference is between these two numbers: 
		\[ \mbox{Range } = 44 - 23 = 21. \] 
	\end{itemize}
\end{framed}









In this frst homework, we will focus on some simple calculations, and simple concepts. The learning outcome is to familiarise yourself with the SULIS homework system

\begin{itemize}
	\item Calculate the range of the following data set
	\[3, 7, 5, 13, 20, 23, 39, 23, 40, 23, 14, 12, 56, 23, 29\]
	%% 330/15 =22
	\item Calculate the range of these numbers:
	\[3, -7, 5, 13, -2\]
	%% 2.4
	\item For each of these data sets, determine the sample size.
\end{itemize}



\section{Summation}
The summation sign $\sum$ is commonly used in most areas of statistics.
Given $x_1 = 3, x_2= 1, x_3 = 4, x_4 = 6, x_5= 8 $ find:

\[
(i) \displaystyle\sum_{i=1}^{i=n} x_{i}  \hspace{3cm}
(ii) \displaystyle\sum_{i=3}^{i=4} x_{i}^2
\]
\begin{eqnarray*}(i) \displaystyle\sum_{i=1}^{i=n} x_{i} &=& x_1 + x_2 +  x_3 +  x_4 + x_5 \\  &=& 3 +1 +4 +6 + 8 \\ &=& \textbf{22} \end{eqnarray*}

\[ (ii) \displaystyle\sum_{i=1}^{i=n} x_{i}^2 = x_3^2 + x_4^2  = 9 + 16 = \textbf{25} \]

\noindent When all elements of a data set are used, a simple version of the summation notation can be used.
$\displaystyle\sum_{i=1}^{i=n} x_{i}$  can simply be written as $\sum x$


\subsection*{Example}
Given that $p_1= 1/4, p_2= 1/8, p_3= 1/8,p_4= 1/3, p_5 = 1/6$ find:

\begin{itemize}
	\item $\displaystyle\sum_{i=1}^{i=n} p_{i} \times x_{i}$
	\item $\displaystyle\sum_{i=1}^{i=n} p_{1} \times x_{i}^2$
\end{itemize}








\section{Summation}
The summation sign $\sum$ is commonly used in most areas of statistics.
Given $x_1 = 3, x_2= 1, x_3 = 4, x_4 = 6, x_5= 8 $ find:

\[
(i) \displaystyle\sum_{i=1}^{i=n} x_{i}  \hspace{3cm}
(ii) \displaystyle\sum_{i=3}^{i=4} x_{i}^2
\]
\begin{eqnarray*}(i) \displaystyle\sum_{i=1}^{i=n} x_{i} &=& x_1 + x_2 +  x_3 +  x_4 + x_5 \\  &=& 3 +1 +4 +6 + 8 \\ &=& \textbf{22} \end{eqnarray*}

\[ (ii) \displaystyle\sum_{i=1}^{i=n} x_{i}^2 = x_3^2 + x_4^2  = 9 + 16 = \textbf{25} \]

\noindent When all elements of a data set are used, a simple version of the summation notation can be used.
$\displaystyle\sum_{i=1}^{i=n} x_{i}$  can simply be written as $\sum x$




\subsection*{Example}
Given that $p_1= 1/4, p_2= 1/8, p_3= 1/8,p_4= 1/3, p_5 = 1/6$ find:

\begin{itemize}
	\item $\displaystyle\sum_{i=1}^{i=n} p_{i} \times x_{i}$
	\item $\displaystyle\sum_{i=1}^{i=n} p_{1} \times x_{i}^2$
\end{itemize}




\section{The Exponential Function }

\[ f(x) = e^{x} \]

For most statistical analyses that you are likely to encounter, the value of x is likely to be negative or less than one.

\[ f(x) = e{-1} \]

\[ f(x) = e^{0.5} \]


\newpage

\begin{framed}
	{
		\large
Combinations formula
\[ ^{n}C_k  = {n! \over k!  \times (n-k)!} \]

\begin{itemize}
	\item Remark $n! = n \times (n-1)! $
	\item 0! = 1
\end{itemize}

\medskip
Show that
\[ ^{n}C_0  = 1 \]

\textbf{Solution: }
\[ ^{n}C_0  = {n! \over 0!  \times (n-0)!} =  {n! \over n!} = 1 \]


Show that
\[ ^{n}C_1  = n \]

\textbf{Solution: }
\[ ^{n}C_1  = {n! \over 1!  \times (n-1)!} =  {n \times (n-1)! \over (n-1)!} = n \]

\bigskip

Compute $ ^{7}C_2  $\\

\textbf{Solution: }
\[ ^{7}C_2  = {7! \over 2!  \times (7-2)!} =  {7 \times 6 \times 5! \over 2! \times 5!} = {42 \over 2} =21  \]



Compute $ ^{11}C_1  $\\

\textbf{Solution: }
\[ ^{11}C_1  = {11! \over 1!  \times 10!} =  {11 \times 10! \over 1 \times 10!} = 11 \]
}
\end{framed}

	\subsection{Example 1}
	
	\[ \binom 5 2  = \frac{5!}{2!\;(5-2)!} = \frac{5.4.3!}{2! .3!} = \frac{5.4}{2.1} = 10\]
	
	\subsection*{Example 2}
	
	\[ \binom 5 0   = \frac{5!}{0!\;(5-0)!} = \frac{5!}{0! .5!} = \frac{5!}{2!} = 1\]
	Recall $0! =1$

\end{document}
