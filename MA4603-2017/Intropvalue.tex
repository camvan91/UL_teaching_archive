%--------------------------------------------------%
The level of marginal significance within a statistical hypothesis test, representing the probability of the occurrence of a given event. The p-value is used as an alternative to rejection points to provide the smallest level of significance at which the null hypothesis would be rejected. The smaller the p-value, the stronger the evidence is in favor of the alternative hypothesis.

P-values are calculated using p-value tables, or spreadsheet/statistical software.

\begin{itemize}
	\item Because different researchers use different levels of significance when examining a question, a reader may sometimes have difficulty comparing results from two different tests. 
	
	\item For example, if two studies of returns from two particular assets were done using two different significance levels, a reader could not compare the probability of returns for the two assets easily. 
	
	\item For ease of comparison, researchers will often feature the p-value in the hypothesis test and allow the reader to interpret the statistical significance themselves. This is called a \textbf{\textit{p-value approach}} to hypothesis testing. 
\end{itemize}

