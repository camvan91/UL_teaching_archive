\documentclass{beamer}

\usepackage{amsmath}
\usepackage{amssymb}

\begin{document}

%------------------------------------------------------------------------------%
\begin{frame}
	% http://www.wyzant.com/resources/lessons/math/statistics_and_probability/introduction/data
	\frametitle{Data}
	\Large
	\begin{itemize}
		\item Data can be defined as groups of information that represent the qualitative or quantitative attributes of a variable or set of variables, which is the same as saying that data can be any set of information that describes a given entity. Data in statistics can be classified into grouped data and ungrouped data.
		
		\item Any data that you first gather is ungrouped data. 
		\item Ungrouped data is data in the raw. An example of ungrouped data is a any list of numbers that you can think of.
	\end{itemize}
\end{frame}
%------------------------------------------------------------------------------%
\begin{frame}
	\frametitle{Grouped Data}
	\Large
	\begin{itemize}
		\item Grouped data is data that has been organized into groups known as classes. 
		\item Grouped data has been 'classified' and thus some level of 
		data analysis has taken place, which means that the data is no longer raw.
		\item A data class is group of data which is related by some user defined property. 
		\item 
		For example, if you were collecting the ages of the people you met as you walked down the street, you could group them into classes as those in 
		their teens, twenties, thirties, forties and so on. Each of those groups is called a class.
	\end{itemize}
\end{frame}
%------------------------------------------------------------------------------%
\begin{frame}
	% http://www.wyzant.com/resources/lessons/math/statistics_and_probability/introduction/data
	\frametitle{Class Intervals}
	\Large
	\begin{itemize}
		\item Each of those classes is of a certain width and this is referred to as the \textbf{Class Interval}. 
		\item This class interval is very important when it comes to drawing Histograms and Frequency diagrams. 
		\item All the classes may have the same class size or they may have different classes sizes depending on how you group your data. 
		\item The class interval is always a whole number.
	\end{itemize}
	
\end{frame}
%------------------------------------------------------------------------------%
\begin{frame}
	% http://www.wyzant.com/resources/lessons/math/statistics_and_probability/introduction/data
	Below is an example of grouped data where the classes have the same class interval.
	\begin{center}
		\begin{tabular}{|c|c|}
			\hline Age (years)	&	Frequency	\\ \hline
			0 - 9	&	12	\\ \hline
			10 - 19	&	30	\\ \hline
			20 - 29	&	18	\\ \hline
			30 - 39	&	12	\\ \hline
			40 - 49	&	9	\\ \hline
			50 - 59	&	6	\\ \hline
			60 - 69	&	0	\\ \hline
		\end{tabular} 
	\end{center}
	
\end{frame}
%------------------------------------------------------------------------------%
\begin{frame}
	Solution:
	
	Below is an example of grouped data where the classes have different class interval.
	
	Age (years)	 Frequency		 Class Interval
	0 - 9	 	15	 	10
	10 - 19	 	18	 	10
	20 - 29	 	17	 	10
	30 - 49	 	35	 	20
	50 - 79	 	20	 	30
	
\end{frame}
%------------------------------------------------------------------------------%
\begin{frame}
	\frametitle{Calculating Class Interval}
	Given a set of raw or ungrouped data, how would you group that data into suitable classes that are easy to work with and at the same time meaningful?
	
	The first step is to determine how many classes you want to have. 
	Next, you subtract the lowest value in the data set from the highest value in the data set and then you divide by the number of classes that you want to have:
	
\end{frame}

%------------------------------------------------------------------------------%
\begin{frame}
	
	Example 1:
	
	Group the following raw data into ten classes.
	
	
	
	Solution:
	
	The first step is to identify the highest and lowest number
	
	
	
	
	
\end{frame}
%------------------------------------------------------------------------------%
\begin{frame}
	
	
	Class interval should always be a whole number and yet in this case we have a decimal number. The solution to this problem is to round off to the nearest whole number.
	
	In this example, 2.8 gets rounded up to 3. So now our class width will be 3; meaning that we group the above data into groups of 3 as in the table below.
	
	
\end{frame}
%------------------------------------------------------------------------------%
\begin{frame}
	
	Number	 Frequency
	1 - 3	 7
	4 - 6	 6
	7 - 9	 4
	10 - 12	 2
	13 - 15	 2
	16 - 18	 8
	19 - 21	 1
	22 - 24	 2
	25 - 27	 3
	28 - 30	 2
	
	
\end{frame}
%------------------------------------------------------------------------------%
\begin{frame}
	\frametitle{Class Limits and Class Boundaries}
	\Large
	\vspace{-1cm}
	\begin{itemize}
		\item 
		Class limits refer to the actual values that you see in the table. 
		\item Taking an example of the table above, 1 and 3 would be the class limits of the first class. 
		\item 
		Class limits are divided into two categories: lower class limit and upper class limit. \item In the table above, for the first class, 1 is the lower class limit while 3 is the upper class limit.
	\end{itemize}
\end{frame}
%------------------------------------------------------------------------------%
\begin{frame}
	\frametitle{Class Limits and Class Boundaries}
	\Large
	\vspace{-1cm}
	\begin{itemize}
		\item On the other hand, class boundaries are not always observed in the frequency table.
		\item  Class boundaries give the true class interval, and similar to class limits, are also divided into lower and upper class boundaries.
		
		\item The relationship between the class boundaries and the class interval is given as follows:
	\end{itemize}
	
\end{frame}
%------------------------------------------------------------------------------%
\begin{frame}
	
	Class boundaries are related to class limits by the given relationships:
	
	
	
	
	
	As a result of the above, the lower class boundary of one class is equal to the upper class boundary of the previous class.
	
	Class limits and class boundaries play separate roles when it comes to representing statistical data diagrammatically as we shall see in a moment.
\end{frame}
%---------------------------------------------------------------%
%-----------------------------------------------------------------------------------------------------------%
\section*{Statistics for grouped data}

\begin{frame}
\frametitle{Statistics for grouped data}
grouped data refers to the arrangement of raw data with a wide range of values into groups. This process makes the data more manageable. Graphs and frequency diagrams can then be drawn showing the class intervals chosen instead of individual values.


\noindent An estimate, $\bar{x}$, of the mean of the population from which the data are drawn can be calculated from the grouped data as:
\[ \bar{x} = \frac{\sum f x }{\sum f}\]
In this formula, $x$ refers to the mid-point of the class intervals, and $f$ is the class frequency. Note that the result of this will be different from the sample mean of the ungrouped data.

\end{frame}
\begin{frame}
\begin{tabular}{|c|c|c|}
\hline
Class limits& Class midpoint & frequency \\
  \hline
\$240 - 259.99 & \$250 &7\\
\$260 - 279.99 & \$270 &20\\
\$280 - 299.99 & \$290 &33\\
\$300 - 319.99 & \$310 &25\\
\$320 - 339.99 & \$330 &11\\
\$340 - 359.99 & \$350 &4\\
\hline
& & Total = 100\\
  \hline
\end{tabular}
\end{frame}

\begin{frame}
\frametitle{Cumulative frequency}
The graph of a cumulative frequency distribution is called an ogive (pronounced ``o-jive"). For the less-than
type of cumulative distribution, this graph indicates the cumulative frequency below each exact class limit of
the frequency distribution. When such a line graph is smoothed, it is called an ogive curve.
\end{frame}

\begin{frame}
\frametitle{Relative frequency}
A relative frequency distribution is one in which the number of observations associated with each class has
been converted into a relative frequency by dividing by the total number of observations in the entire
distribution. Each relative frequency is thus a proportion, and can be converted into a percentage by multiplying
by 100.\\

\noindent One of the advantages associated with preparing a relative frequency distribution is that the cumulative
distribution and the ogive for such a distribution indicate the cumulative proportion (or percentage) of
observations up to the various possible values of the variable. A percentile value is the cumulative percentage of
observations up to a designated value of a variable.
\end{frame}

\end{document}