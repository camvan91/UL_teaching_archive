{
	{Multiplication Rule}

\vspace{-0.5cm}
\begin{itemize}
%\item The multiplication rule is a result used to determine the probability that two events, $A$ and $B$, both occur.
\item This useful multiplication rule follows from the definition of conditional probability.
\item First we algebraically re-arrange the conditional probability equation.
\[ P(A \cap B) \;=\; P(A|B)\times P(B). \]
\item Equivalently $ P(A \cap B) \;=\; P(B|A)\times P(A). $
\item Therefore we can say:
\end{itemize}
\[ P(A|B)\times P(B) \;=\; P(B|A)\times P(A). \]
}
%-------------------------------------------------------%
{
	{Multiplication Rule}

\vspace{-1cm}
As an aside, for \t{independent events}, (events which have no influence on one another), the multiplication rule simplifies to:
\[P(A \cap B)  = P(A)\times P(B) \]
}
%-------------------------------------------------------%
{
	{Multiplication Rule}

\vspace{-1cm}
\t{Going back to our example:}\\
From the first year intake example, check that
\[ P(E|F)\times P(F) = P(F|E)\times P(E)\]
\begin{itemize}
\item $P(E|F)\times P(F) = 0.58 \times 0.38  = 0.22$
\item $P(F|E)\times P(E) = 0.55 \times 0.40  = 0.22$
\end{itemize}
}