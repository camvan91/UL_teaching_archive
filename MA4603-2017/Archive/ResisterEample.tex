


\section*{Question 2}

\begin{itemize}
\item Ohm's Law: $U = IR$

\begin{itemize}
\item U: Potential (also known as Voltage)
\item I: Current
\item R: Resistance
\end{itemize}

\item Suppose that the resistance R is a random
variable having uniform distribution over interval $[1.5, 2.25]$, i.e. $R \sim U(1.5; 2.25)$ \item $U$ has normal
distribution with the mean 12[V] and standard deviation $1.2[V]$, i.e. $U \sim N(12, 1.2^2)$.\item Assume that
R and U are independent and solve the following problems.
\end{itemize}



\noindent\textbf{Part 1}Find the expected values for R, I. $E[R]$ and $E[I]$

\[ \int^b_a {1 \over x} dx =log(b) - log(a) \]

\begin{itemize}
\item $R \sim U(a,b)$
\item From Formulae:  $E(R) = \frac{a+b}{2} = = \frac{1.5 +2.25 }{2} =1.875 $
\item $E(R) = 1.875[ \Omega]$
\bigskip
\item We are told that $E(U) = 12 [V] $

\item \[ E[I] = E \left[{U \over R} \right] =  E[U] \times E[1/R] \]
\item We need to compute $E[1/R]$

\item Recall \[E[R] =  \int \left[ r f(r) \right] dr \]
\item Recall \[E[R^2] = \int \left[ r^2 f(r)\right] dr\]
\item Therefore \[ E[1/R] =  \int \left[1/r f(r)\right] dr  \]
\item In this case, the density function $f(r)$ is a constant.
\[f(r) = {1 \over b-a}  = {1 \over 2.25 - 1.5} = {1\over 0.75}\]

\end{itemize}



\begin{itemize}
\item \[ E[1/R] =  \int \left[1/r f(fr)\right] dr = f(r) \times \int \left[1/r \right] dr   \]
\item Density function is a constant
\[f(r) = {1 \over b-a}  = {1 \over 2.25 - 1.5} = {1\over 0.75}\]
\item Using the hint

\[ \int^b_a {1 \over r} dr =log(b) - log(a) = log(2.25) - log(1.5)  \]

\item \[ E(1/R) = {log(2.25) - log(1.5)  \over 0.75}  = 0.54063 \]

\item So the answer is
\[ E(I) = 12 \times 0.54063 = 6.4876 [A] \]
\end{itemize}






\noindent\textbf{Part 3} Find the variances for R and I\\


hint:
\[ \int^b_a {1 \over x^2} dx = 1/a - 1/b \]

\begin{itemize}

\item We use a very similar approach to the last part
\item $R \sim U(a,b)$
\item From Formulae:  $Var(R) = \frac{(b-a)^2}{12} = \frac{(2.25 - 1.5)^2 }{12}  $
\item $Var(R) = 0.046875$
\item We are also told that $Var(U) = 1.2^2 = 1.44 $
\item Using formulae \[Var(I) = E(I^2) - E(I)^2 \]
\item We need to compute $E(I^2)$
\[ E(I^2)= E \left[ \frac{U^2}{R^2} \right] = E[U^2] \times E \left[ \frac{1}{R^2} \right] \]



\end{itemize}







\begin{itemize}

\item We can say: $Var(U) = E[U^2] - E[U]^2$
\item We know $Var(U)$ and can compute $E[U]^2$ easily
\item $1.44 = E[U^2] - 12^2$
\item Necessarily $ E[U^2] = 145.44$

\item To find $E[1/R^2]$ we use a very similar approach to the last question.

\[ E[1/R^2] =  \int \left[{1 \over r^2} f(r)\right] dr  = f(r) \int {1 \over r^2}  dr  \]

\[ E[1/R^2] = {(1/1.5) - (1/2.25) \over 0.75} = 0.29627 \]

\item Therefore $E[I^2]  = 145.44 \times 0.29627 = 43.09$

\item $Var[I] = E[I] - E[I]^2  = 43.09 - (6.4876)^2 = 1.001[A]$

\end{itemize}



\noindent\textbf{Part 3} Find the standard deviations for R and I?\\



\begin{itemize}
\item Simply compute the square roots of the respective variances.
\item $\sigma_R = 0.21651 [\Omega]$
\item $\sigma_I = 1.0005 [A]$
\end{itemize}



\noindent\textbf{Part 4} If you observe I=7[A], would yo reconsider this a rather unusual value for current in the view of the obtained values for the mean and the standard deviation of I? Justify your answer\\


\begin{itemize}
\item The expected value of $I$ is 6.4876 [A]
\item The standard deviation is 1.005
\item We expect most values to be within $6.48 \pm 1 [A]$

\item The observed value of the current seems to be: close to the mean, and within the range of typical values for current I
\end{itemize}

\noindent\textbf{Part 5} Find the covariance between R and I\\



\begin{itemize}
\item $Cov(R,I) = E(R\times I) - \left(\mu_R \times \mu_I \right)$
\item Alternatively :$Cov(R,I) = E(R\times I) - \left(E(R) \times E(I) \right)$
\item Using Ohm's law: $E (I \times R) = E(U)  =12$
\item $E(I) \times E(R) = 1.875 \times 6.4876 = 12.164$
\item $Cov(R,I) = -0.164$
\end{itemize}




\noindent\textbf{Part 6} Evaluate the correlation coefficients between R and I.\\

\begin{itemize}
\item See Formula
\[ \rho = \frac{Cov(X,Y)}{\sigma_x \sigma_y} \]
\item From before $\sigma_I =1.005 [A]$ and $\sigma_R =0.21651 [\Omega]$
\item
The correlation coefficient is defined as the ratio between the covariance and the product
standard deviation
\[\rho_{R,I} = { -0.164 \over (0.21651 \times 1.005)} = -0.7537 \]

\end{itemize}



\noindent\textbf{Part 7}Based on the obtained values of characteristics, if you observe R = 2.1[$\Omega
$], then out of two
possible values 7.5[A] and 5.2[A] for the current I which you would consider more plausible and
why?

\begin{itemize}
\item From the correlation coefficient, we see a fairly strong negative relationship
\item If we see high values of one variable, we expect lower values of the other.
\item In this instance, we have a reasonably high value of resistance 2.1, where the max is 2.25.
\item Therefore we expected a low value for current.
\item We expect that the value of 5.2 [A] is more likely.

\end{itemize}
%---------------------------------------------------- %
