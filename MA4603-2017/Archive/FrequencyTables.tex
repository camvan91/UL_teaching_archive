	
\section{ Frequency Table }
			\begin{itemize}
				\item A frequency table is a way of summarising a set of data. It is a record of how often each value (or set of values) of the variable in question occurs. \item  It may be enhanced by the addition of percentages that fall into each category.
				
				\item A frequency table is used to summarise categorical, nominal, and ordinal data. 
				
				\item 
				It may also be used to summarise continuous data once the data set has been divided up into sensible groups.
				\item
				When we have more than one categorical variable in our data set, a frequency table is sometimes called a contingency table because the figures found in the rows are contingent upon (dependent upon) those found in the columns.
			\end{itemize}
		
		
		{
			Example 
			Suppose that in thirty shots at a target, a marksman makes the following scores: 
			5 2 2 3 4 4 3 2 0 3 0 3 2 1 5 
			1 3 1 5 5 2 4 0 0 4 5 4 4 5 5 
			
			The frequencies of the different scores can be summarised as: 
			Score  Frequency Relativa Frequency (%)  
			0 4 13% 
			1 3 10% 
			2 5 17% 
			3 5 17% 
			4 6 20% 
			5 7 23% 
			
			
