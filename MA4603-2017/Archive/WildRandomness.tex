Wild Randomness

What is wild randomness? Simply put, it is an environment in which a single observation or a particular number can impact the total in a disproportionate way. The bell curve has �thin tails� in the sense that large events are considered possible but far too rare to be consequential.

But many fundamental quantities follow distributions that have �fat tails� � namely, a higher probability of extreme values that can have a significant impact on the total.

Ignorance probably played the larger role, he thinks. Rating agencies, like investors and regulators, rely on relatively simple models to forecast the risk associated with future market movements. Those models often assume a "mild randomness" of market fluctuations.

In reality, Cont argues, what visionary mathematician Beno�t Mandelbrot calls "wild randomness" prevails: Risk is concentrated in a few rare, hard-to-predict, but extreme, market events. 