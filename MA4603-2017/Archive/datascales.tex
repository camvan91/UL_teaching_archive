Qualitative Data
Quantitative Data

Discrete v Continuous



1.2   Types of Data

Qualitative data: labels or names; usually non-numeric e.g. gender, social class 

Quantitative data: indicate how much or how many; numeric data e.g. age, height

Quantitative

DiscreteContinuous


%==============================================================================================%

Discrete: values change by whole numbers or steps e.g. family size

Continuous: can take all values in a given range including decimal places e.g. height





%==============================================================================================%
%==============================================================================================%
	
	1.3  Scales of Data Measurement
	
	As well as categorising data into certain types, we can also categorise data by levels or scales of measurement. Again, some levels of measurement are more useful than others.
	
	%==============================================================================================%
	
	Why do we need to know the scale of measurement of our data? Not only do we use different methods of analysis for different types of data but some methods of analysis require the data to be measured at a certain level or scale as well.
	
	%==============================================================================================%
	
	Scales of measurement
	\begin{itemize}
		\item Nominal
		\item Ordinal
		\item Interval
		\item Ratio
	\end{itemize}
	
	

	
	%==============================================================================================%
	
	Classify the following by data type and scale of measurement:
	
	1.Sector of business e.g. manufacturing or service
	2.Weight
	3.Cigarette consumption e.g. light, heavy, moderate
	4.Income
	5.Number of visits to the doctor per year
	6.Profit (euros)
	
	
	
	%==============================================================================================%
	
	
	\textbf{Answers}
	
	
	1.  Qual.      Nominal
	2.  Quant.    Ratio
	3.  Qual.      Ordinal
	4.  Quant.    Ratio
	5.  Quant.    Ratio
	6.  Quant. Ratio
	

	%==============================================================================================%
	
	Qualitative data is usually measured at a nominal or ordinal scale of measurement.
	
	Nominal: the data gives the person a label or tells us what category a person or object falls into e.g. colour of the car.
	
	%==============================================================================================%
	
	Ordinal: the data has all the properties of nominal data but we also get more information since the order of the data is meaningful e.g. do you think this module is
	
	very easy easy challengingvery challenging
	
	
	Notice that there is a ranking system present.
	
	Whatever you answer is still just a word i.e. non-numeric but it gives more information than the colour of a car. All students can be ranked depending on what they think of the module.
	
	
	%==============================================================================================%
	
	Quantitative data is usually measured at an interval or ratio scale of measurement.
	
	Interval: the data is numeric and has order in the same way as ordinal data has order. We can also measure the difference between two observations e.g. we can measure the difference between two temperatures of 10 degrees and 20 degrees i.e. 10 degrees.
	
	We cannot measure the difference between two students where one thinks the module is easy and the other finds it challenging i.e. we will not get a meaningful number.
	
	%==============================================================================================%
	
	Ratio: the data is numeric and has all the properties of interval data. The ratio of the data is also meaningful e.g. we can say things like I�m half her age or twice as tall. The data has a meaningful and unique zero point which allows us to say if for example you have zero age, you do not exist.
	
	Temperature does not have a unique and meaningful zero point. For example, if the temperature is zero degrees Celsius it does not mean temperature does not exist. Temperature does not have a unique zero point - it depends on which temperature scale you are using i.e. zero degrees Fahrenheit also exists
	
	
%============================================================ %
%============================================================ %

	
	Ratio v Interval
	
	The difference between ratio and interval is most evident in how a value of zero is treated.
	
	For Ratio values � a value of zero means the complete absence.
	For interval values
	
	Consider that 0 degrees centigrade is 32 degrees Fahrenheit.
	
	ordinal
	nominal (Categorical)
	
	For example consider a variable�origin� where the outcomes are Limerick /Rest of Munster/Rest of Ireland/Rest of World.
	
	ordinal � contains a hierarchy
	
	Discrete:
	
	Number of Children in a class
	Number of faulty components per batch
	
	Continuous
	
	Length of time between cars arriving at a petrol station pump.
	
	
	It  is not possible have 24.5 children in any class, but it is possible to have an average of 24.5 per class.

%============================================================ %
	
	Ratio v Interval
	
	The difference between ratio and interval is most evident in how a value of zero is treated.
	
	For Ratio values � a value of zero means the complete absence.
	For interval values
	
	Consider that 0 degrees centigrade is 32 degrees Fahrenheit.
	
	ordinal
	nominal (Categorical)
	
	For example consider a variable�origin� where the outcomes are Limerick /Rest of Munster/Rest of Ireland/Rest of World.
	
	ordinal � contains a hierarchy
	
	Discrete:
	
	Number of Children in a class
	Number of faulty components per batch
	
	Continuous
	
	Length of time between cars arriving at a petrol station pump.
	
	
	It  is not possible have 24.5 children in any class, but it is possible to have an average of 24.5 per class.