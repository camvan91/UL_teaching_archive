

%%--------------------------------------------------------%
%{
%	{Permutations and Combinations}
%
%
%Often we are concerned with computing the number of ways of selecting and arranging groups of items. \begin{itemize} \item  A \t{\emph{combination}} describes the selection of items from a larger group of items.  \item A \t{\emph{permutation}} is a combination that is arranged in a particular way.
%\end{itemize}
%
%\bigskip
%\begin{itemize}
%\item Suppose we have items A,B,C and D to choose two items from.
%\item AB is one possible selection, BD is another. AB and BD are both combinations.
%\item More importantly, AB is one combination, for which there are two distinct permutations: AB and BA.
%\end{itemize}
%}

%%--------------------------------------------------------%
%{
%	{Combinations}
%
%\t{Combinations: }
%The number of ways of selecting $k$ objects from $n$ unique objects is:
%
%\[ ^n C_k = {n!  \over k! \times (n-k)!} \]
%
%In some texts, the notation for finding the number of possible combination is written
%
%\[ ^n C_k =  {n \choose k} \]
%
%}
%
%%--------------------------------------------------------%
%{
%	{Example of Combinations}
%How many ways are there of selecting two items from possible 5?
%
%\[ ^5 C_2   \left( \mbox{ also }  {5 \choose 2}  \right) =  {5!  \over 2! \times 3!} =  {5 \times 4 \times 3!  \over 2 \times 1 \times 3!} = 10  \]
%
%\bigskip
%Discuss how combinations can be used to compute the number of rugby matches for each group in the Rugby World Cup.
%
%}
%%--------------------------------------------------------%
%{
%	{The Permutation Formula}
%The number of different permutations of r items from n unique items is written as $^n P_k$
%
%
%\[ ^n P_k = \frac{n!}{(n-k)!}\]
%}
%
%%--------------------------------------------------------%
%{
%	{Permutations}
%\t{Example:}
%How many ways are there of arranging 3 different jobs, between 5 workers, where each worker can only do one job?
%
%
%\[ ^5 P_3 = \frac{5!}{(5-3)!}  = {5! \over 2!} = 60\]
%
%}
%
%




