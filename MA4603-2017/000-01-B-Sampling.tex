
\textbf{Sampling}
In medical trials (which are an
important statistical application) the population may be those
patients who arrive for treatment at the hospital carrying out the
trial, and this may be very different from one hospital to
another. If you look at any collection of official statistics
(which are most important for state planning) you will be struck
by the great attention that is given to defining the population
that is surveyed or sampled, and to the definition of terms. For
instance, it has proved difficult to get consensus on the meaning
of ‘unemployed’ in recent years but a statistician must be
prepared to investigate the population of unemployed. Think about
this carefully. It should help you with your Sociology and
Marketing and Market Research modules as well as this one:

%--------------------------------------------------------------------------------------%
%--------------------------------------------------------------------------------------%

\textbf{Bias}

In addition to the common-sense meaning of bias, there is also a
more technical meaning for the word in statistics. This will be
found both in Chapter 10 of this guide and in the work on
estimators in Statistics 2. It seems natural enough to wish to
avoid bias, but it is not helpful to be swayed by the value
judgements inherited from the use of a word outside the limits of
academic discussion.

%--------------------------------------------------------------------------------------%
%--------------------------------------------------------------------------------------%

\textbf{Sampling}



Explain the difference between sampling error and sampling bias. Explain
briefly which is taking place in the following situations, if the population under study consists of all the pupils in a certain school:

i. Your sample is a list of all pupils at the school except those who arrived
at the school in the last school year.

ii. You take a random sample of names from the school register of all pupils at the school.

%--------------------------------------------------------------------------------------%
%--------------------------------------------------------------------------------------%

\textbf{Sampling}
Define quota sampling. In what circumstances would you use it?
In what circumstances would you use stratified random sampling?
Give two ways in which stratified random sampling differs from quota sampling.

%--------------------------------------------------------------------------------------%
%--------------------------------------------------------------------------------------%

\textbf{Sampling}


Although it seems sensible to sample a population to avoid the
cost of a total enumeration (or census) of that population, it is
possible to make a strong argument against the practice. One might
well consider that sampling is fundamentally unfair because a
sample will not accurately represent the whole population, and it
allows the units selected for the sample to have more importance
than those not selected. This might be thought undemocratic. Many
countries continue to take a full census of their population, even
though sampling might be cheaper. It is less obvious, but true,
that sampling might well be more accurate, because more time can
be spent verifying the information collected for a sample.


%--------------------------------------------------------------------------------------%
\end{document}
