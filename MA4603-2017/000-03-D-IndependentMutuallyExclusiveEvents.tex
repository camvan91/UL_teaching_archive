\documentclass[]{report}

\voffset=-1.5cm
\oddsidemargin=0.0cm
\textwidth = 480pt

\usepackage{framed}
\usepackage{subfiles}
\usepackage{graphics}
\usepackage{newlfont}
\usepackage{eurosym}
\usepackage{amsmath,amsthm,amsfonts}
\usepackage{amsmath}
\usepackage{color}
\usepackage{amssymb}
\usepackage{multicol}
\usepackage[dvipsnames]{xcolor}
\usepackage{graphicx}
\begin{document}

%%\chapter{12. Inference Procedures for Single Samples }


\section{Independent Events}


Two events, A and B, are independent if the fact that A occurs does not affect the probability of B occurring.\\

Some other examples of independent events are:

\begin{itemize}
\item Landing on heads after tossing a coin AND rolling a 5 on a single 6-sided die.
\item Choosing a marble from a jar AND landing on heads after tossing a coin.
\item Choosing a 3 from a deck of cards, replacing it, AND then choosing an ace as the second card.
\item Rolling a 4 on a single 6-sided die, AND then rolling a 1 on a second roll of the die.
\end{itemize}



\noindent \textbf{Multiplication Rules}   

When two events, A and B, are independent, the probability of both occurring is:
\[ P(A \cap B) = P(A) \times P(B)\]

Similarly for events A,B and C, the probability of all three events occurring is:
\[ P(A \cap B \cap C) = P(A) \times P(B) \times P(C) \]


\section{Independent Events}
Events A and B in a probability space $S$ are said to be independent if the
occurrence of one of them does not influence the occurrence of the other.\\ \bigskip

More specifically, $B$ is independent of A if $P(B)$ is the same as $P(B|A)$. Now
substituting $P(B)$ for $P(B|A)$ in the multiplication theorem from the previous
slide yields.
\[ P(A\cap B) = P(A)\times P(B)\]
We formally use the above equation as our definition of independence.





\noindent \textbf{Independent Events}

\begin{itemize}
\item Two events are independent if the occurrence of one does not change the probability of the other occurring.
\item An example would be rolling a 2 on a die and flipping a head on a coin. Rolling the 2 does not affect the probability of flipping the head.
\item If events are independent, then the probability of them both occurring is the product of the probabilities of each occurring.
\end{itemize}
\[P(A \cap B) = P(A) \times P(B)\]

%---------------------------------------------------------------------------------------------------%   


\section*{Independent events}

\begin{itemize}
\item Two events are independent when the occurrence or nonoccurrence of one event has no effect on the
probability of occurrence of the other event. 

\item Two events are dependent when the occurrence or nonoccurrence
of one event does affect the probability of occurrence of the other event.
\end{itemize}



%------------------------------------------------------------------------------------------------%



\subsection{Independent Events}
\begin{itemize}
\item Suppose that a man and a woman each have a pack of 52 playing cards.
\item Each draws a card from his/her pack. Find the probability that they each draw a Queen.
\item We define the events:
\begin{itemize} \normalsize \item A = probability that man draws a Queen = 4/52  = 1/13
\item B = probability that woman draws a Queen = 1/13
\end{itemize} \item Clearly events A and B are independent so:
\[ P(A \cap B) = 1/13 \times 1/13 = 0.005917 \]
\end{itemize}




%------------------------------------------------------------------------%
{

\subsection{Independent Events}
Events A and B in a probability space $S$ are said to be independent if the
occurrence of one of them does not influence the occurrence of the other.\\ \bigskip

More specifically, $B$ is independent of A if $P(B)$ is the same as $P(B|A)$. Now
substituting $P(B)$ for $P(B|A)$ in the multiplication theorem from the previous
slide yields.
\[ P(A\cap B) = P(A)\times P(B)\]
We formally use the above equation as our definition of independence.

}
%--------------------------------------------------------------------------------%
\section{Mutually exclusive events}
Two or more events are mutually exclusive, or disjoint, if they cannot occur together. That is, the occurrence
of one event automatically precludes the occurrence of the other event (or events). For instance, suppose we
consider the two possible events ``ace" and ``king" with respect to a card being drawn from a deck of playing
cards. These two events are mutually exclusive, because any given card cannot be both an ace and a king.
Two or more events are nonexclusive when it is possible for them to occur together. 
\\
Note that this definition does not indicate that such events must necessarily always occur jointly. For instance, suppose we consider the two possible events ``ace" and ``spade". These events are not mutually exclusive, because a given card can be both an ace and a spade; however, it does not follow that every ace is a spade or every spade is an ace.



\begin{itemize}
\item Two events are mutually exclusive if they cannot both happen. 
\item A simple example is the set of outcomes of a single coin toss, which can result in either heads or tails, but not both.
\end{itemize}

\begin{itemize}
\item Two events are mutually exclusive (or disjoint) if it is impossible for
them to occur together.
\item Formally, two events $A$ and $B$ are mutually exclusive if and only if
$A\cap B$ = $\varnothing$ \end{itemize}\bigskip
Consider our die example
\begin{itemize}
\item Event A = `observe an odd number' = $\{1,3,5\}$
\item Event B = `observe an even number' = $\{2,4,6\}$

\item $A\cap B$ = $\varnothing$ (i.e. the empty set), so $A$ and $B$ are mutually exclusive.
\end{itemize}



\section{Mutually Exclusive Events}
Mutually exclusive events are events that cannot happen at the same time.
\[ P(A and B) = P(A) + P(B) \]
\noindent \textbf{Mutually Exclusive Events}
\begin{itemize}
\item Two events are mutually exclusive (or disjoint) if it is impossible for
them to occur together.
\item Formally, two events $A$ and $B$ are mutually exclusive if and only if
$A\cap B$ = $\varnothing$ \end{itemize}\bigskip
Consider our die example
\begin{itemize} 
\item Event A = `observe an odd number' = $\{1,3,5\}$
\item Event B = `observe an even number' = $\{2,4,6\}$

\item $A\cap B$ = $\varnothing$ (i.e. the empty set), so $A$ and $B$ are mutually exclusive.
\end{itemize}



\begin{itemize}
\item In a given observation or experiment, an event must either occur or not occur. \item Therefore, the sum of the
probability of occurrence plus the probability of nonoccurrence always equals 1. \item Thus, where $A^{\prime}$ indicates the nonoccurrence of event A, we have
$P(A) + P(A^{c}) =  1$

\item 
Two or more events are mutually exclusive, or disjoint, if they cannot occur together. That is, the occurrence
of one event automatically precludes the occurrence of the other event (or events). 
\item For instance, suppose we
consider the two possible events ``ace" and ``king" with respect to a card being drawn from a deck of playing
cards. 
\item These two events are mutually exclusive, because any given card cannot be both an ace and a king.
\item Two or more events are nonexclusive when it is possible for them to occur together.

\item Note that this definition does not indicate that such events must necessarily always occur jointly. For instance, suppose we consider the two possible events ``ace" and ``spade". 
\item These events are not mutually exclusive, because a given card can be both an ace and a spade; however, it does not follow that every ace is a spade or every spade is an ace.
\end{itemize}
\noindent \textbf{Mutually Exclusive Events}

\begin{itemize}
\item Suppose you roll a 6 sided die. \item Let \textbf{A} be the event that the number is odd and \textbf{B} be the event that the number is even. 
\item Since the die is only rolled once, it is impossible for the number that lands face up is both odd and even. \item The events \textbf{A} and \textbf{B} are said to be mutually exclusive events. \item If two events cannot happen at the same time, they are said to be mutually exclusive.


\end{itemize}

%---------------------------------------------------------------------------------------------------%

\noindent \textbf{Mutually Exclusive Events}

\begin{itemize}
\item Two events are \textbf{mutually exclusive} if they cannot occur together. 
\item Another way of expressing mutually events is \textbf{disjoint} events.
\item If two events are mutually exclusive, then the probability of them both occurring at the same time is 0.
Disjoint:  \[P(A \cap B) = 0\]
\end{itemize}


\begin{itemize}
\item If two events are mutually exclusive, then the probability of either occurring is the sum of the probabilities of each occurring.
\item Specific Addition Rule: Only valid when the events are mutually exclusive.
\[P(A \cup B) = P(A) + P(B)\]
\end{itemize}   


\section{Independence Vs Mutual Exclusion}
{\bf \textbf{Independence Vs Mutual Exclusion}}

{\bf Do not mix up the ideas of independence and mutual exclusion.\\[0.3cm]}
\begin{itemize}\itemsep0.3cm
\item {\bf Independent events}
\begin{itemize}\itemsep0.2cm
\item Have \emph{no effect} on each other.
\item \emph{Can} happen at the same time (but work \emph{independently} of each other).
\item Allow us to simplify the {\bf multiplication rule}.
\end{itemize}
\item {\bf Mutually exclusive events}
\begin{itemize}\itemsep0.2cm
\item \emph{Cannot} happen at the same time.
\item Certainly \emph{affect} each other since the presence of one excludes the presence of the other.
\item Allow us to simplify the {\bf addition rule}.
\end{itemize}
\end{itemize}

Bottom line: If events are independent they are not mutually exclusive. If events are mutually exclusive they are not independent.





Classify the following pairs of events as being mutually exclusive, independent or dependent (but not mutually exclusive).\\

\begin{tabular}{ccc}
& Event $A$ & Event $B$ \\
\textbf{a)} &  A coin shows a head   & The same coin shows a tail \\
\textbf{b)} &  You work hard   & You get promoted \\
\textbf{c)} &  You are Irish  & It rains in Japan \\
\textbf{d)} &  Anti-virus out of date  & Laptop is virus-free \\
\textbf{e)} &  You are in this lecture hall   & You are in  Scholars \\
\textbf{f)} &  You are in this lecture   & You are on Facebook \\
\textbf{g)} &  An individual is not wealthy   & He/she drives an expensive car  \\
\textbf{h)} &  One coin shows a head   & Another coin shows a head
\end{tabular}

%--------------------------------------------------------------------------------%
{
\textbf{Addition Rule (Continued)}
For mutually exclusive events, that is events which cannot occur together:
$P(A\cap B) = 0$.\\ \medskip The addition rule therefore reduces to
\[ P(A\cup B) = P(A) + P(B)\]
}
%--------------------------------------------------------------------------------%

\end{document}
%--------------------------------------------------------------------------------%

