\documentclass[]{report}

\voffset=-1.5cm
\oddsidemargin=0.0cm
\textwidth = 480pt

\usepackage{framed}
\usepackage{subfiles}
\usepackage{graphics}
\usepackage{newlfont}
\usepackage{eurosym}
\usepackage{amsmath,amsthm,amsfonts}
\usepackage{amsmath}
\usepackage{color}
\usepackage{amssymb}
\usepackage{multicol}
\usepackage[dvipsnames]{xcolor}
\usepackage{graphicx}
\begin{document}
\chapter{6. Discrete Probability Distributions}

\noindent \textbf{Discrete probability distributions}

The discrete probability distributions that described in this course are
\begin{itemize}
\item the binomial distribution, 
\item the geometric distribution,
\item the hypergeometric distribution, 
\item the Poisson distributions.
\end{itemize}


\section{Discrete Probability Distributions}
\begin{itemize}
\item Over the next set of lectures, we are now going to look at two important discrete probability distributions

\item The first is the \textbf{\emph{binomial}} probability distribution.

\item The second is the Poisson probability distribution.

\item In \texttt{R}, calculations are performed using the \texttt{binom} family of functions and \texttt{pois} family of functions respectively.
\begin{multicols}{2}
\begin{itemize}
\item[$\ast$] Poisson
\item[$\ast$] Binomial
\item[$\ast$] Geometric
\item[$\ast$] Hypergeometric 
\end{itemize}
\end{multicols}
\end{itemize}
%---------------------------------------------------------------------%



\subsection{The Cumulative Distribution Function}
\begin{itemize}
\item The Cumulative Distribution Function, denoted $F(x)$, is a common way that the probabilities
of a random variable (both discrete and continuous) can be summarized.
\item The Cumulative Distribution Function, which also can be
described by a formula or summarized in a table, is defined as:
\[F(x) = P(X \leq x) \]
\item The notation for a cumulative distribution function, F(x), entails using a capital
"F".  (The notation for a probability mass or density function, f(x), i.e. using a lowercase "f". The notation is not interchangeable.
\end{itemize}


%---------------------------------------------------------------------%

\begin{framed}
\textbf{Useful Results for Discrete Random Variables}

\begin{itemize}
\item $P(X \leq 1) = P(X=0) + P(X=1)$
\item $P(X \leq r) = P(X=0)+ P(X=1) + \ldots P(X= r)$
\item $P(X \leq 0) = P(X=0)$
\item $P(X = r) = P(X \geq r ) - P(X \geq r + 1)$
\item \textbf{Complement Rule}: $P(X \leq r-1) = P(X < r) = 1 - P(X \geq r)$
\item \textbf{Interval Rule}:$ P(a \leq X \leq  b)= P(X \geq a) - P(X \geq b + 1).$
\end{itemize}
\end{framed}



\section{The Cumulative Distribution Function}
\begin{itemize}
\item The Cumulative Distribution Function, denoted $F(x)$, is a common way that the probabilities
of a random variable (both discrete and continuous) can be summarized.
\item The Cumulative Distribution Function, which also can be
described by a formula or summarized in a table, is defined as:
\[F(x) = P(X \leq x) \]
\item The notation for a cumulative distribution function, F(x), entails using a capital
"F".  (The notation for a probability mass or density function, f(x), i.e. using a lowercase "f". The notation is not interchangeable.
\end{itemize}


\end{document}
